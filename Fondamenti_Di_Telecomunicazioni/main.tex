\documentclass{article}
\usepackage{amsmath, amssymb, tikz, geometry, graphicx, natbib, mwe, xcolor,
 listings, tabularx, pdfpages, blindtext, mathtools, stackengine, amsthm, pgfplots,bigints, relsize, upgreek, esint, array}
\usepackage{hyperref}
\usepackage{slashed, enumitem}

\pgfplotsset{width=10cm,compat=1.9}

\colorlet{myWhite}{white!35!gray}
\definecolor{shadeofgray}{HTML}{181818}
\definecolor{shadeofviolet}{HTML}{0f022c}

\hypersetup{
    colorlinks=true,
    linkcolor=violet,
    filecolor=magenta,      
    urlcolor=cyan,
    pdftitle={Overleaf Example},
    pdfpagemode=FullScreen,
}

\geometry{ 
 a4paper,
 total={170mm,257mm},
 left=20mm,
 top=10mm,
 }
 
\lstdefinestyle{mystyle}{ 
bracketsstyle=\color{red}
}

\title{Formulario di Fondamenti di Telecomunicazioni}
\author{Giuseppe Bumma}


%----------------------------------------------------------------------
%use this for a total black background
%\pagecolor{black}
%\color{myWhite}
%----------------------------------------------------------------------


\pagecolor{shadeofgray}
\color{myWhite}



\begin{document}

%Commands
\newcommand{\R}{\mathbb{R}}
\newcommand{\bb}[1]{\mathbb{#1}}
\newcommand{\cc}[1]{\mathcal{#1}}
\newcommand{ \lognormal }{\text{Lognormal} }
\newcommand{\tb}[1]{\textbf{#1}}
\newcommand*\circled[1]{\tikz[baseline=(char.base)]{%
            \node[shape=circle,draw,inner sep=2pt] (char) {#1};}}
%for using circled number in enumerate use:
%\begin{enumerate}[label=\protect\circled{\arabic*}]


\tableofcontents

\maketitle

\section{Numeri complessi}
\begin{center}
    \renewcommand{\arraystretch}{2.5}
    \begin{tabular}{c c}
        Unità immaginaria & $j^2=-1$\\
    \end{tabular}
    \begin{tabular}{c c}
        Forma classica & $z = a + jb$\\
    \end{tabular}
\end{center}
\subsubsection*{Coordinate Polari}
\begin{align*}
    a &= r \cos(\phi) & r &= \sqrt{a^2 + b^2}\\
    b &= r \sin (\phi) & \phi &= 
    \begin{cases}
        \arctan\left( \dfrac{b}{a} \right) &a > 0\\
        \\
        \arctan \left( \dfrac{b}{a} \right) + \pi &a<0
    \end{cases}
\end{align*}
\subsubsection*{}



















\end{document}