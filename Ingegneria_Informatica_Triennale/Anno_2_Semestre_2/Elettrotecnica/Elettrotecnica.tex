\documentclass{article}
\usepackage{amsmath, amssymb, tikz, geometry, graphicx, natbib, mwe, color, xcolor,
 listings, tabularx, pdfpages, blindtext, mathtools, stackengine, amsthm, pgfplots,bigints, relsize, upgreek, esint, array, multirow, schemata, wrapfig, cancel, comment, animate, scalerel, steinmetz}
\usepackage{hyperref}
\usepackage{slashed, enumitem}
\usepackage{titlesec}
\usetikzlibrary{positioning}

\begin{comment}
code to write section, subsection and subsubsection title in a specific color
\titleformat{\section}
{\color{synthwave_text}\normalfont\Large\bfseries}
{\color{synthwave_text}\thesection}{1em}{} 

\titleformat{\subsection}
{\color{synthwave_text}\normalfont\large\bfseries}
{\color{synthwave_text}\thesubsection}{1em}{} 

\titleformat{\subsubsection}
{\color{synthwave_text}\normalfont\normalfont\bfseries}
{\color{synthwave_text}\thesubsubsection}{1em}{}
\end{comment}


\pgfplotsset{compat=1.9}

\colorlet{myWhite}{white!35!gray}
\definecolor{shadeofgray}{HTML}{181818}
\definecolor{shadeofviolet}{HTML}{0f022c}
\definecolor{synthwave_beckground}{HTML}{252334}
\definecolor{synthwave_text}{HTML}{e148aa}


\hypersetup{
    colorlinks=true,
    linkcolor=black,
    filecolor=magenta,      
    urlcolor=cyan,
    pdftitle={Elettotecnica},
    pdfpagemode=FullScreen,
}


\geometry{ 
 a4paper,
 left=10mm,
 right=10mm,
 top=10mm
 }
 
\lstdefinestyle{mystyle}{ 
bracketsstyle=\color{red}
}

\title{Elettrotecnica}
\author{Giuseppe Bumma}


% color option
%\pagecolor{synthwave_beckground} %{shadeofgray}
%\color{myWhite}

\renewcommand{\CancelColor}{\color{synthwave_text}}


\begin{document}

%Commands
\newcommand{\R}{\mathbb{R}}
\newcommand{\Varepsilon}{\mathcal{E}}
\newcommand{\rad}{\text{rad}}
\newcommand{\bb}[1]{\mathbb{#1}}
\newcommand{\cc}[1]{\mathcal{#1}}
\newcommand{ \lognormal }{\text{Lognormal} }
\newcommand{\T}[1]{\text{#1}}
\newcommand*\circled[1]{\tikz[baseline=(char.base)]{%
            \node[shape=circle,draw,inner sep=2pt] (char) {#1};}}
%for using circled number in enumerate use:
%\begin{enumerate}[label=\protect\circled{\arabic*}]


\tableofcontents

\maketitle

\section{Introduzione}

\subsection{La carica elettrica e la forza di Coulomb}
Se due particelle cariche, supposte puntiformi, di carica $q_0$ e $q_1$, siano a una distanza finita fra loro nel vuoto, la \textbf{legge di Coulomb} descrive la forza elettrostatica interagente fra loro:
\[
    |F_C| \propto \frac{q_1q_2}{e^2}
\]
con $r$ distanza tra le due cariche.

\begin{center}
    \includegraphics[scale=0.3]{Image/Forza di Coulomb.png}
\end{center}

La forza di Coulomb $F_C$ è diretta nella direzione di $r$. Quando $q_1$ e $q_2$ hanno lo
stesso segno la forza di Coulomb è repulsiva. Quando sono di segno opposto la
forza è attrattiva.\\
L'unità di misura, nel Sistema Internazionale (SI), della forza di Coulomb è il
newton [N] ed il coefficiente di proporzionalità è $1/(4\pi \epsilon_0)$ dove $\epsilon_0$ è la costante dielettrica del vuoto [$\epsilon_0 = 8,854x10^{-12} C^2/(Nm^2)$].
\vspace*{0.2cm}\\
L'unità di misura della carica elettrica nel sistema di misura SI è il \textbf{coulomb} [C]. La carica elementare nel SI è $e$ ove
\[
    e = 1,6021 x 10^{-19} C
\]
Protone ed elettrone hanno carica di valore assoluto e. Due protoni o
due elettroni si respingono. Un protone ed un elettrone si attraggono.
Per convenzione la carica del protone è positiva ($+e$) e quella dell'elettrone negativa (-$e$).\\
In natura esistano solamente cariche multiple di e. Non può esistere
una carica sottomultiplo di $e$.



\subsubsection{La cariche elettriche ed il loro moto}
Forza che agisce su una particella carica:
\[
    \vec F = q(\vec E + \vec u \times \vec B)
\]
\begin{align*}
    &\vec F: \text{forza [N]} &
    &\vec q: \text{carica elettrica [C]} &
    &\vec u: \text{velocità della carica [m/s]}
\end{align*}
\begin{align*}
    &\vec E: \text{campo elettrico} &
    &\vec B: \text{vettore induzione magnetica}
\end{align*}
\begin{itemize}
    \item Se $\vec B=0$ si ha la cosiddetta \textbf{Forza elettrostatica}
    \[
        \vec F = q \vec E
    \]
    Quindi il campo elettrico $\vec E = \frac{\vec F}{q} $ è una forza per unità di carica [N/C].
    \vspace*{0.1cm}\\
    Campo elettrico e forza elettrostatica da cui esso deriva hanno la stessa
    direzione. Perciò il campo produce un'accelerazione della carica lungo la
    propria direzione.
    \vspace*{0.1cm}\\
    Nel SI l'unità di misura di $\vec E$ è: $N/C = V/ m = m \ kg \ s^{-2} C^{-1}$.
    \vspace*{0.2cm}\\
    \item Se $\vec E=0$ si ha la \textbf{Forza di Lorentz}
    \[
        \vec F = q(\vec u \times \vec B)
    \]
    Quindi il vettore induzione magnetica $\vec B$ è una forza per unità di carica e di velocità [$N s / C m$].
    Campo elettrico e forza elettrostatica da cui esso deriva hanno la stessa
    direzione. Perciò il campo produce un'accelerazione della carica lungo la
    propria direzione.
    Nel SI l'unità di misura di $\vec E$ è: $N/C = V/ m = m \cdot kg \cdot s^{-2} \cdot C^{-1}$.
\end{itemize}
Una particella carica induce una forza sulle cariche
che la circondano. Tale forza può essere attrattiva o
repulsiva. Essa è la forza Coulombiana $F_C$ (o forza
elettrostatica). In ogni punto della regione attorno
alla carica o in presenza ad una distribuzione di
cariche vi è un campo elettrico $\vec E(x,y,z)$ definito dalla
forza indotta su una carica di prova puntiforme
unitaria posta nel punto considerato.
\begin{center}
    \includegraphics[scale=0.27]{Image/Forza elettrostatica.png}
\end{center}
Qualora su una carica in moto si induca una forza
deviante perpendicolare al moto, tale forza è la
forza magnetica o forza di Lorentz 
$F_L$. Il campo
di induzione magnetica $\vec B(x,y,z)$, legato a $\vec F_L$, è
dato dalla forza indotta su una carica unitaria in
moto per unità di velocità della carica stessa. La
direzione del campo $\vec B$ è perpendicolare alla
velocità ed alla forza $\vec FL$ . Il campo $\vec B$ è
perpendicolare alla velocità della carica ed alla
forza indotta.

\subsection{Densità volumetrica di carica}
La carica elettrica non può essere creata o distrutta (legge della
conservazione della carica elettrica). Può solo essere trasferita. Pertanto, la
carica elettrica totale di un sistema isolato non può variare.\\
La densità volumetrica di carica (o distribuzione di carica) è definita da:
\[
    \rho_C (x,y,z) = \lim_{\Delta t \rightarrow 0} \frac{\Delta q}{\Delta \tau} = \frac{dt}{dq}  
\]
dove $d \tau$ è l'elemento infinitesimo di volume.

\subsubsection{Densità di corrente \texorpdfstring{$J$}{J}}
La densità di corrente elettrica $\vec J$ è il vettore il cui modulo è la quantità di
carica che attraversa una superficie unitaria perpendicolare alla velocità $\vec u$
delle cariche. La direzione ed il verso di $\vec J$ sono la direzione ed il verso di $\vec u$:
\[
    \vec J \cdot \hat n = \lim_{\Delta S \rightarrow 0} \lim_{\Delta t \rightarrow 0} \frac{\Delta Q}{\Delta S \Delta t}
\]
$\vec J$: densità di corrente $\left[\frac{C}{m^2\cdot s}\right] = \left[\frac{A}{m^2} \right]$

\begin{center}
    \includegraphics[scale=0.3]{Image/Densità di corrente.png}
\end{center}

$\vec J(x,y,z)$ definisce un campo vettoriale ed è la densità di flusso delle
cariche. La corrente elettrica $i$ è il flusso di carica attraverso una
superficie $S$:
\[
    i = \iint\limits_S \vec J \cdot \hat n \ dS
\]




\subsection{Corrente elettrica}
La corrente elettrica i che attraversa una superficie è la quantità di carica
che attraversa la superficie nell’unità di tempo:
\[
     i = \frac{\Delta q}{\Delta t} 
\]
Se si considera un cavo conduttore, ad esempio, la corrente nel conduttore è la
quantità di carica che attraversa una sezione del cavo nell'unità di tempo.

\begin{center}
    \includegraphics[scale=0.3]{Image/Corrente conduttore.png}
\end{center}

L'unità di misura SI è l'ampere [$A$] dove $A = \frac{C}{s}$
\vspace*{0.1cm}\\
La \textbf{corrente elettrica istantanea} è:
\[
    i(t) = \lim_{\Delta t \rightarrow 0} \frac{\Delta q}{\Delta t} = \frac{dq}{dt}
\]



\subsection{Tensione elettrica e differenza di potenziale elettrico}
La \textit{tensione elettrica} $e_{12}$ fra i punti 1 e 2 lungo il
percorso $l$, è il lavoro $L^{1\rightarrow 2,l}_{q=1}$ che il campo elettrico
$\vec E(x,y,z)$ compie per portare una carica unitaria dl
punto 1 al punto 2 lungo $l$:
\[
    e_{12} = \int_{1,l}^2 \vec E \cdot d \vec l
\]

\begin{center}
    \includegraphics[scale=0.3]{Image/Tensione.png}
\end{center}

Per spostare la carica $q$ dal punto 1 al 2 il lavoro è:
\[
    L^{1 \rightarrow 2,l}_q = q \cdot e_{12}
\]
L'unità di misura SI di $e_{12}$ è il volt [$V$] dove $V = \frac{J}{C} = m^2 \cdot kg \cdot s^{-2} \cdot C^{-1}$.
Qualora la tensione e 12 dipenda dai valori di una
funzione $v(x,y,z)$ definita in una regione che contiene la linea $l$ essa diviene:
\[
    e_{12} = \int_{1,l}^{2}\vec E \ d\vec l = - \int_{1,l}^{2} dv = v_1 - v_2 = v_{12}
\]
dove $v(x,y,z)$ è la \textbf{funzione potenziale elettrico} e $v_{12}$
è la \textbf{differenza di potenziale elettrico}.
\vspace*{0.1cm}\\
Poiché $v_{12}$ è la differenza fra i valori che la funzione $v(x,y,z)$ assume nel punto
iniziale e nel punto finale di $l$, $v_{12}$ non dipende dal percorso che unisce i due
punti. Quindi il campo $\vec E$ è un \textbf{vettore conservativo} \footnote{un campo conservativo è un campo il cui integrale lineare è indipendente dalla traiettoria} con $\vec E = \vec \nabla \cdot v(x,y,z)$.
\vspace*{0.1cm}\\
Per un percorso chiuso $l_c$ contenuto nella regione ove $\vec E$ è conservativo, si ha:
\[
    e_l = \oint _{l_c} \vec E \cdot d \vec l_c = - \oint _{l_c} \vec \nabla \cdot v  \ d\vec l_c = 0
\]



\subsection{Legge di Ampere-Prima legge di Maxwell}
La grandezza vettoriale campo magnetico H è definito dalla legge di Ampere (prima legge di Maxwell)
\[
    \oint _{l_c} \vec H \ d \vec l_c = i_t
\]
dove la corrente totale $i_t = i + i_s$.\\
In questo caso la corrente totale $i_t$ è il flusso del vettore $J_t$ i ovunque solenoidale ($J_t = J + \partial D/ \partial t$). Perciò $i_t$ è il flusso concatenato con la linea chiusa $l_C$ contorno della superficie che attraversa. Il verso di percorrenza di $l$ è determinato con regola della vite destrogira.
\begin{center}
    \includegraphics[scale=0.3]{Image/Campo magnetico.png}
    \includegraphics[scale=0.3]{Image/Campo magnetico-2.png}
\end{center}
L'unità di misura SI di $\vec H$ è l'ampere su metro [$\frac{A}{m}$].
Per materiali lineari: $\vec B = \mu \vec H$ ove $mu$ è la permeabilità magnetica del materiale. Per mezzi non lineari
$\vec B = f(\vec H)$. Solitamente per i materiali magnetici non lineari $f$ è una funzione \textbf{isteretica} (materiali
ferromagnetici).
\begin{align*}
    \oint_{l_C}\vec H \ d \vec l &= \iint\limits_{S} \left( \vec J + \frac{\partial
    D}{\partial t} \right) \ \hat n \ dS\\
    &= \underbrace{\iint\limits_{S} \vec J \ \hat n \ dS}_{\text{corrente di conduzione } I} + \underbrace{\iint\limits_{S} \frac{\partial D}{\partial t} \ \hat n \ dS}_{\iint\limits_{S} \partial D \ \hat n \ dS =\vec \Phi(D)} =\\
    &= I + \underbrace{\frac{\partial \vec \Phi(D)}{\partial t}}_{\text{corrente di spostamento}}
\end{align*}
Immaginiamo di descrivere due superfici $S_1$ e $S_2$ sulla linea chiusa $l_C$
\[
    \oint_{l_C} \vec H \ d\vec l = \iint\limits_{S_1}\left( \vec J + \frac{\partial D}{\partial t} \right) \hat n_1 \ dS_1 = \iint\limits_{S_2}\left( \vec J + \frac{\partial D}{\partial t} \right) \hat n_2 \ dS_2
\]
Prendiamo una superficie chiusa $S_C$ su $S_2$, allora
\[
    \oint_{S_C}\underbrace{\bigg( \vec J + \frac{\partial D}{\partial t}\bigg)}_{\text{vettore solenoidale}}\hat n_C \ dS_C = \iint\limits_{S_2}\left( \vec J + \frac{\partial D}{\partial t} \right) \hat n_2 \ dS_2 - \iint\limits_{S_1}\left( \vec J + \frac{\partial D}{\partial t} \right) \hat n_1 \ dS_1 = 0
\]



\subsection{Legge dell'induzione di Faraday-Seconda legge di Maxwell}
La legge dell'induzione (o legge di Faraday
od anche seconda legge di Maxwell)
stabilisce che:
\[
    e_{l_C} = \oint_{l_C}
    \vec E \ d\vec l_C = -\frac{d \Phi}{dt}
\]
\begin{center}
    \includegraphics[scale=0.3]{Image/Legge di Faraday.png}
\end{center}
ove $\Phi$ è il flusso magnetico concatenato con
la linea chiusa $l_c$. (direzione di $l_C$ data dalla
regola della vite destrogira).\\
$e_{l_c}$ è la tensione elettrica indotta sulla
linea chiusa dalla variazione del flusso
magnetico concatenato con $l_c$; essa è detta
\textbf{forza elettromotrice} (f.e.m.).
\vspace*{0.1cm}\\
\textbf{N.B.} In questo caso $\vec E$ non à conservativo.



\subsection{Conservazione della carica elettrica}
La carica elettrica non si crea né si distrugge. Perciò la diminuzione della carica elettrica
all'interno di un volume $\tau$ corrisponde alle
cariche che lasciano $\tau$ fluendo attraverso la
superficie chiusa $S$, superficie esterna di $\tau$.
\begin{center}
    \includegraphics[scale=0.3]{Image/Conservazione della carica.png}
\end{center}
La \textbf{\textit{
legge di conservazione della carica elettrica}} afferma questo ed è espressa
dall'espressione: 
\[
    \oiint_{S} \vec J \ \hat n \ dS = - \frac{dQ}{dt}
\]
si ha variazione di cariche solo se c'è passaggio di corrente.




\subsection{Legge di Gauss}
Il campo di induzione elettrica o campo spostamento elettrico è definito dalla
legge di Gauss.
Considerando una superficie chiusa $S$,
che delimita il volume $V$; sia $\hat n$ il versore
normale alla superficie. La legge di Gauss
afferma che:
\[
    \oiint_{S} \vec D \ \hat n \ dS = \iiint_{V}\rho \ dV = Q
\]



\subsection{Forza elettromotrice}
$\vec E$ e $\vec B$ descrivono le forze prodotte dal fenomeno elettromagnetico sulle
cariche (forza elettrica per unità di carica e forza magnetica per unità di
carica e di velocità della carica). Esse descrivono ciò che viene prodotto dal
fenomeno EM. Ne descrivono l'\textbf{effetto}.
\vspace*{0.1cm}\\
$\vec D$ ed $\vec H$ descrivono ciò che produce il fenomeno EM (la carica elettrica nel
primo caso e la corrente totale nel secondo). Ne descrivono la \textbf{causa}.

\subsection{Leggi dell'Elettromagnetismo in forma integrale}
\renewcommand{\arraystretch}{2.5}
\begin{center}
    \begin{tabular}{|c|c|}
        \hline
        $\oint_{l_c} \vec H \ d \vec l_c = i_t$ & $1^o$ legge di Maxwell\\
        \hline 
        $\oint_{l_c} \vec E \ d \vec l_c = \dfrac{d\Phi}{dt}$ & $2^o$ legge di Maxwell\\
        \hline
        $\oiint \vec J \ \hat n dS = - \dfrac{dq}{dt}$ & legge di conservazione della carica \\
        \hline
        $\oiint \vec D \ \hat n dS = q $ & legge di Gauss\\
        \hline 
        $\oiint \vec J_t \ \hat n dS = 0 $ & $\vec Jt$ ovunque solenoidale\\
        \hline t
        $\oiint _S \vec B \ \hat n dS=0$ & $\vec B$ ovunque solenoidale\\
        \hline
    \end{tabular}    
\end{center}
Tre di queste sei equazioni sono linearmente indipendenti, le altre tre si ottengono dalle prime tre.



\subsection{Leggi dell'Elettromagnetismo in forma locale}
\begin{center}
    \begin{tabular}{|c|c|}
        \hline
        $\nabla \times \vec H = \vec J + \dfrac{\partial
        \vec D}{\partial t}$ & 1° legge di Maxwell (dal teorema di Stokes) \\
        \hline 
        $\nabla \times \vec E = - \dfrac{\partial
        \vec B}{\partial t}$ & 2° legge di Maxwell
        (dal teorema di Stokes)\\
        \hline
        $\nabla \cdot \vec J = - \dfrac{\partial \rho_c}{\partial t} $ & legge di conservazione della carica (teor. divergenza)\\
        \hline
        $\nabla \cdot \vec D = \rho_c$ & legge di Gauss (dal teorema della divergenza)\\
        \hline
        $\nabla \cdot \vec J_t = 0$ & $\vec J_t$ ovunque solenoidale (dal teorema della divergenza)\\
        \hline 
        $\nabla \cdot \vec B =0$ & $\vec B$ ovunque solenoidale (dal teorema della divergenza)\\
        \hline
    \end{tabular}    
\end{center}
Tre di queste sei equazioni sono linearmente indipendenti, le altre tre si ottengono dalle prime tre.



\subsection{Relazioni materiale}
$\vec E$ e D, $\vec B$  ed $\vec H$  descrivono i fenomeni dell'EM in modo diverso. $\vec E$ e $\vec D$ si
riferiscono al fenomeno Elettrico, $\vec B$  ed $\vec H$  al fenomeno magnetico. $\vec D$ ed $\vec H$ 
descrivono i due fenomeni misurando ciò che li origina: la carica il primo, ed il
moto della carica il secondo. Gli effetti misurati da $\vec E$ e da $\vec B$  sono in entrambe i
casi le forze indotte. Essi dipendono da come i diversi materiali reagiscono.
Inoltre, dipendentemente dalla proprietà del materiale, ad un certo valore del
campo $\vec E$ si induce un determinato moto di carica misurato da $\vec J$ . Le relazioni
fra queste descrizioni spesso sono lineari. A volte però non lo sono con
relazioni anche di tipo isteretico.
\begin{center}
    \begin{tabular}{|c|c|}
    \hline
    \textbf{Materiali lineari} & \textbf{Materiali non lineari}\\
    \hline 
    $\vec D = \epsilon \vec E$ & $\vec D = f_1(\vec E)$\\
    \hline 
    $\vec B = \mu \vec H$ & $\vec B = f_2(\vec H)$\\
    \hline 
    $\vec J = \sigma \vec E$ & $\vec J = f_3(\vec E)$\\
    \hline
\end{tabular}
\end{center}
con $\epsilon$ costante dielettrica, $\mu$ permeabilità magnetica e $\sigma$ conducibilità termica.
\vspace*{0.1cm}\\
la costante dielettrica (permittività elettrica) $\epsilon$, e la permeabilità magnetica $\mu$
di un materiale sono espresse per mezzo dei loro valori relativi $\epsilon_r$ ed $\mu_r$ in riferimento al loro valore nel vuoto $\epsilon_0$ ed $\mu_0$:
\begin{align*}
    &\epsilon = \epsilon_r \epsilon_0 &\text{dove } \epsilon_0 = 8,856 x 10^{-12} Farad/metro \left[\frac{F}{m}\right]\\
    &\mu = \mu_r \mu_0 &\text{dove } \mu_0 = 1,256 x 10^{-6} Henry/metro \left[\frac{H}{m}\right]
\end{align*}
Riporto alcuni valori di $\epsilon_r$
\begin{center}
    \renewcommand{\arraystretch}{1}
    \begin{tabular}{c|c}
         & $\epsilon_r$\\
        \hline
        vuoto & 1\\
        aria & $\simeq 1$\\
        plastica & 2-5\\
        vetro & 4-8\\
        acqua & 80
    \end{tabular}
\end{center}
Molto diverse sono le variazioni per materiali differenti della
conducibilità elettrica, della permeabilità magnetica e della costante
dielettrica.
Per la conducibilità elettrica $\sigma$ vi è una variazione anche di $10^{23}$ (23 ordini
di grandezza) fra materiali isolanti e materiali conduttori.
Per la permeabilità magnetica $\mu$ la variazione raggiunge al massimo un
valore di circa $10^5$ (5 ordini di grandezza).
Per la costante dielettrica $\epsilon$ la variazione massima si riduce ad un valore
massimo di circa $10^3$ (3 ordini di grandezza).
\vspace*{0.2cm}\\
La relazione fra $\vec J$ ed $\vec E$ è
anche definita dalla
resistività elettrica $\rho$:
\[
    \vec E = \rho \vec J
\]
dove
$\rho = \frac{1}{\sigma}$
$\sigma$ è in Siemens/metro [$\frac{S}{M}$] e $\rho$ in Ohm/metro [$\frac{\Omega}{m}$].


\subsection{SI Units}
\subsubsection{Unità derivate SI}

    \begin{tabular}{|m{35mm}|c|c|m{35mm}|}
    \hline
    Grandezza&
    Simbolo (nome)&
    Unità SI non di base&
    Unità SI di base\\
    \hline 
    Carica elettrica&
    $C$ (Coulomb)&
    & 
    $s \times A$\\
    \hline
    Tensione elettrica e differenza di potenziale elettrico &
    $V$ (Volt)&
    $ \dfrac{W}{A}$&
    $m^2 \times kg \times s^{-3} \times A^{-1}$\\
    \hline
    Forza&
 $N$ (Newton)&
 &
 $m \times kg \times s^{-2}$\\
 \hline
Energia/Lavoro&
 $J$ (Joule)&
 $N \times m$&
 $m^2 \times kg \times s^{-2}$\\
 \hline
Potenza&
 $W$ (Watt)&
 $\dfrac{J}{s}$&
 $m^2 \times kg \times s^{-3}$\\
 \hline 
Flusso magnetico&
 $Wb$ (Weber)&
 $V \times s$&
 $m2 \times kg \times s-2 \times A^{-1}$\\
 \hline 
Induzione magnetica&
 $T$ (Tesla)&
 $\dfrac{Wb}{m^2}$&
 $kg \times s^{-2} \times A^{-1}$\\
 \hline 
Resistenza elettrica&
 $\Omega$ (Ohm)&
 $\dfrac{V}{A}$&
 $m^2 \times kg \times s^{-3} \times A^{-2}$\\
 \hline 
Conduttanza elettrica&
 $S$ (Siemens)&
 $\dfrac{A}{V}$&
 $m^{-2} \times kg^{-1} \times s^3 \times A^2$\\
 \hline 
Capacità&
 $F$ (Farad)&
 $\dfrac{C}{V}$&
 $m^{-2} \times kg^{-1} \times s^4 \times A^2$\\
 \hline 
Induttanza&
 $H$ (Henry)&
 $\dfrac{Wb}{A}$&
 $m^2 \times kg \times s^{-2} \times A^{-2}$\\
 \hline
Frequenza&
 $Hz$ (Hertz)&
 &
 $s^{-1}$\\
 \hline 
\end{tabular}


\subsubsection{Prefissi SI}
\begin{center}
    \renewcommand{\arraystretch}{1.5}
    \begin{tabular}{|c|c|c|}
        \hline
        Factor & Name & Symbol\\
        \hline 
        $10^{-24}$ & yocto  & y\\
        \hline 
        $10^{-21}$ & zepto & z\\
        \hline 
        $10^{-18}$ & atto & a\\
        \hline 
        $10^{-15}$ & femto & f\\
        \hline 
        $10^{-12}$ & pico & p\\
        \hline 
        $10^{-9}$ & nano & n\\
        \hline 
        $10^{-6}$ & micro & $\mu$\\
        \hline 
        $10^{-3}$ & milli & m\\
        \hline 
        $10^{-2}$ & centi & c\\
        \hline 
        $10^{-1}$ & deci & d\\
        \hline 
        $10^{1}$ & deca & da\\
        \hline 
        $10^{2}$ & hecto & mh\\
        \hline 
        $10^{6}$ & mega & M\\
        \hline 
        $10^{9}$ & giga & G\\
        \hline 
        $10^{12}$ & tera & T\\
        \hline 
        $10^{15}$ & peta & P\\
        \hline 
        $10^{18}$ & exa & E\\
        \hline 
        $10^{21}$ & zetta & Z\\
        \hline 
        $10^{24}$ & yotta & Y\\
        \hline 
    \end{tabular}
\end{center}






\section{Circuiti elettrici}
\subsection{Introduzione}
I circuiti elettrici sono degli elementi interconnessi tra loro e le connessioni possono essere considerate dei conduttori ideali.
\vspace*{0.2cm}\\
\textbf{Ipotesi:}
\begin{itemize}
    \item $ \dfrac{\partial \vec B}{\partial t} = 0 $ o $ \dfrac{\partial \vec D}{\partial t} = 0 $
    \item $L_c << \lambda$ (lunghezza d'onda)
\end{itemize}
Ricordiamo che $\lambda = \dfrac{c}{f} = \left[ \dfrac{velocita \ onda}{frequenza \ onda} \right]$.
\subsubsection*{Esempio}
La rete elettrica ha frequenza $f = 50 Hz$
\[
    \lambda = \dfrac{c}{f} = \dfrac{3 \cdot 10^{8}}{50} \simeq 6000 km
\]
infatti la linea di trasmissione della corrente elettrica è $L_c = 10^3 km$.
\vspace*{0.2cm}\\
Gli elementi del circuito vengono chiamati \textit{multipoli}; di seguito si riporta la lista:
\begin{itemize}
    \item Nodo: punto di intersezione tra 2 o più elementi;
    \item Maglia: linea chiusa all'interno del circuito;
    \item Ramo: componenti insieme ai suoi morsetti
\end{itemize}
Le formule fondamentali per i circuiti sono:
\begin{align*}
    &\underbrace{P(t) = v(t) \cdot i(t)}_{\text{potenza}} & &\underbrace{W(t) = \int P(t) \ dt}_{\text{energia}}
\end{align*}

\subsubsection{Convenzione del generatore e dell'utilizzatore}
Per ogni circuito esistono due convenzioni per il verso della corrente
\begin{center}
    \includegraphics[scale=0.3]{Image/Convenzione.png}
\end{center}
\begin{itemize}
    \item negli utilizzatori tensione e corrente sono discordi
    \item nei generatori tensione e corrente sono concordi (si fa riferimento ai generatori di corrente e di tensione)
\end{itemize}
Ad esempio, nel caso di una batteria ricaricabile, se si utilizza la convenzione del generatore:
\begin{align*}
    &\text{Scarica} & v&=1,5 V & i&=1A & P&=v\cdot i = 1,5W \ {\color{violet}>0}\\
    &\text{Carica} & v&=1,5 V & i&=-1A & P&=v\cdot i = -1,5W \ {\color{violet}<0}
\end{align*} 
mentre se si utilizza la convenzione dell'utilizzatore i segni di corrente e ,propedeuticamente, di potenza sono invertiti 
\begin{align*}
    &\text{Scarica} & v&=1,5 V & i&=-1A & P&=v\cdot i = -1,5W \ {\color{violet}<0}\\
    &\text{Carica} & v&=1,5 V & i&=1A & P&=v\cdot i = 1,5W \ {\color{violet}>0}
\end{align*} 


\subsection{Risolvere un circuito}
Risolvere un circuito vuol dire calcolare le tensioni e le correnti di tutti i componenti.
\subsubsection{Prima legge (LKT)}
Usando le \textbf{leggi di Kirchhoff} e le \textbf{leggi costitutive}
\[
    \oint \vec E \ d \vec l = \oiint \dfrac{\partial \vec B}{\partial t}\ \hat n dS \underbrace{=}_{Hp.1} 0
\]
ne consegue che il campo elettrico è conservativo e dunque si può definire una differenza di potenziale.
\[
    \int_l \vec E \ d \vec l = v(A) - v(B) = v_{AB}
\]
\begin{align*}
    \underbrace{\oint \vec E \ d \vec l}_{0} &= \int_{A}^{B} \vec E \ d\vec l + \int_{B}^{C} \vec E \ d\vec l + \int_{C}^{A} \vec E \ d\vec l \\
    &= v_{BA}+ v_{CB} + v_{AC} \\
    &= 0
\end{align*}
da qui la $1^a$ legge di Kirchhoff per le tensioni:
\[
    \sum_{k=1}^nv_k=0
\]
nella singola maglia.


\subsubsection*{Esempio}
\begin{center}
    \includegraphics[scale=0.37]{Image/Esempio1-Circuiiti.png}
\end{center}
\[
\begin{cases}
    M_1: & v_1 - v_2 = 0\\
    M_2: & v_2+v_3-v_4-v_5 = 0\\
    M_3: & v_1 + v_3 - v_4 - v_5=0
\end{cases}
\]
I versi delle differenze di potenziale sono date dal testo dell'esercizio, il verso di percorrenza
della maglia è scelto arbitrariamente. Il segno positivo o negativo delle tensioni è determinato in base al verso di percorrenza (positivo se concorde, negativo se discorde).

\subsubsection{Seconda legge (LKC)}
\[
    \oiint \oiint \vec J + \dfrac{\partial \vec D}{\partial t} \ \hat n dS = 0 \Longrightarrow \oiint \oiint \vec J \ \hat n dS = 0
\]
\begin{center}
    \textit{“La somma delle correnti entranti ed uscenti da un componente è nulla (in un
componente, entra ed esce la stessa quantità di corrente)."}
\end{center}
Prendiamo in esame il seguente circuito
\begin{center}
    \includegraphics[scale=0.25]{Image/Esempio LKC.png}
\end{center}
Sapendo che il segno di ogni corrente è \textbf{positivo} se \textbf{entra} in un nodo e \textbf{negativo} se ne esce, non è difficile calcolare le correnti nei nodi A e B 
\begin{align*}
    A&: i_1 + i_2 - i_3=0 & B&: -i_5-i_4+i_6=0
\end{align*}


\subsubsection{Teorema di Tellegen}
Combinando la LKT e la LKC su un circuito è possibile verificare che la somma delle potenze dei generatori è pari alla somma delle potenze degli utilizzatori.
\[
    \sum_{k=1}^{n^o \text{ generatori}} p_k = \sum_{j=1}^{n^o \text{ utilizzatori}} p_j
\]

\textbf{Versione estesa:}
Si consideri una rete elettrica con l tensioni di lato ed l correnti di lato che soddisfino le \underline{leggi di Kirchhoff}. Si ha che $\sum\limits_{k=1}^l r_k i_k = 0$. Se $v$ e $i$ rappresentano le tensioni e le corrispondenti correnti di lato in uno stesso istante, si ha che il teorema di Tellegen si riduce al principio \textbf{di conservazione delle potenze istantanee}. È possibile esprimere la potenza \underline{erogata} dai bipoli attivi come $\sum\limits_{h=1}^M p_h$ dove $M$ è il numero di componenti che rispettano la convenzione \underline{del generatore}, e la potenza









\subsection{Elementi circuitali passivi}
Gli elementi circuitali passivi sono elementi che (nel caso ideale) restituiscono la stessa energia che ricevono. 
\subsubsection{Resistore}
\begin{center}
    \includegraphics[scale=0.4]{Image/Resistore.png}
\end{center}
\begin{align*}
    &\underbrace{v(t) = R \cdot i(t)}_{1^a\text{ legge di Ohm}} \Longrightarrow i(t) = \dfrac{1}{R} \cdot v(t) & &\underbrace{R = \rho \cdot \dfrac{l}{S}}_{2^a \text{ legge di Ohm}} = \rho \cdot  \dfrac{[\text{lunghezza filo}]}{[\text{sezione filo}]}
\end{align*}
\[
    p(t) = v(t) \cdot i(t) = R \cdot i^2(t) >0
\]
Il resistore assorbe sempre potenza, non può erogarla.


\subsubsection{Condensatore}
\begin{center}
    \includegraphics[scale=0.4]{Image/Condensatore.png}
\end{center}
\[
    i = C \cdot \dfrac{dv}{dt} \underset{i=\frac{dq}{dt}}{\Longrightarrow} \frac{dq}{dt} = C \cdot \frac{dv}{dt}
\]
quindi $Q = C \cdot V$.\\
Un condensatore (ideale) è non dissipativo: eroga sempre la stessa quantità di potenza che ha assorbito.
\begin{align*}
    \underbrace{w(t)}_{\text{energia}} &= \int_{t_1}^{t_2} p(t) \ dt\\
    &= \int_{t_1}^{t_2} C v  \frac{dv}{dt} \ dt\\
    &= C \int_{t_1}^{t_2} v \ dv\\
    &= \frac{1}{2} C \left[v^2(t_2) - v^2(t1)\right]
\end{align*}
quindi se consideriamo $t_1=0$ e la tensione $v$ calcolata in un istante finale $t_f$
\[
    w(0,t_f) = \frac{1}{2}Cv^2
\]
$v$ è la variabile di stato del condensatore.
\begin{center}
    \includegraphics[scale=0.22]{Image/Grafico condensatore.png}
\end{center}
in quel punto $\dfrac{dv}{dt}=\infty \Rightarrow i(t)=\infty \Rightarrow p(t) = \infty$, ma è impossibile avere potenza infinita, quindi non possiamo avere una variazione istantanea di tensione (il grafico della tensione di un condensatore non può avere discontinuità).
\begin{align*}
    i(t) \ dt = C \cdot \frac{dv}{dt} \Longrightarrow \int_{- \infty}^{t} \frac{i(t)}{C} dt = \underbrace{\int_{- \infty}^{t} \frac{dv}{dt}dt}_{v(t)} \Longrightarrow  v(t) &= \int_{- \infty}^{t} \frac{1}{C} \cdot i(t) \ dt\\ 
    &= \underbrace{\int_{- \infty}^{0} \frac{1}{C} \cdot i(t) \ dt}_{v_0} + \int_{0}^{t} \frac{1}{C} \cdot i(t) \ dt
\end{align*}
A differenza del resistore, per conoscere la tensione non ho bisogno solo della corrente nell'istante corrente, ma anche della tensione iniziale; per questo il condensatore è un \textbf{elemento con memoria}.
\vspace*{0.2cm}\\
Un \textbf{condensatore reale} è semplicemente un condensatore messo in parallelo con un resistore.


\subsubsection{Induttore}
L'induttore è un componente elettrico che genera un campo magnetico al passaggio di corrente elettrica.
\begin{center}
    \includegraphics[scale=0.4]{Image/Induttore.png}
\end{center}
\[
    v= L \dfrac{di}{dt}
\]
Sappiamo che la corrente genera un flusso:
\begin{align*}
    &\underbrace{\Phi = L \cdot i}_{I^a\text{ equazione di Maxwell}} & \underbrace{v= \dfrac{d \Phi}{dt}}_{\T{legge di Faraday}} = L \dfrac{di}{dt}
\end{align*}
\[
    p(t) = v(t) \cdot i(t) = L \cdot i(t) \dfrac{di}{dt} \begin{cases}
        >0 &\text{assorbita}\\
        <0 &\text{erogata}
    \end{cases}
\]
L'induttore è un elemento \textbf{non dissipativo}.
\begin{align*}
    w(t) &= \int_{t_1}^{t_2} p(t) \ dt\\
    &=L \int_{t_1}^{t_2}i(t) \dfrac{di}{dt} \ dt \\
    &= \dfrac{1}{2} L \left[ i^2(t_2) -i^2(t_1)  \right]
\end{align*}
\[
    w(0,t_f) = \dfrac{1}{2} L \cdot  I^2_f
\]
Anche per l'induttore le variazioni istantanee di tensione non sono possibili.
\begin{center}
    \includegraphics[scale=0.4]{Image/Induttore ferr.png}
\end{center}
\[
   L = \dfrac{\mu  N^2  A}{l} 
\]
\begin{align*}
    i(t) &= \dfrac{1}{L} \int_{- \infty}^{t} v(t) \ dt \\
    &= \dfrac{1}{L} \underbrace{\int_{- \infty}^{0} v(t) \ dt}_{i_0} + \dfrac{1}{L} \int_{0}^{t} v(t) \ dt
\end{align*}
Si vede che anche l'induttore è un elemento \textbf{con memoria}.\\
Un induttore reale è un induttore in serie con un resistore.

\subsubsection{Dispositivi in serie e in parallelo}
\begin{center}
    \begin{tabular}{c c c c}
        \includegraphics[scale=0.37]{Image/Resistori in serie.png} 
        &
         \includegraphics[scale=0.37]{Image/Resistori in parallelo.png} 
        &
        \includegraphics[scale=0.37]{Image/Condensatori in serie.png}
        &
        \includegraphics[scale=0.37]{Image/Condensatori in parallelo.png}
        \\
        $R_{eq} = \mathlarger{\sum}_{k=1}^{n}R_k$
        &
        $R_{eq}= \left(\mathlarger{\sum}_{k=1}^n \dfrac{1}{R_k} \right)^{-1} $
        &
        $C_{eq} = \left( \mathlarger{\sum}_{k=1}^{n} \dfrac{1}{C_k} \right)^{-1}$
        &
        $C_{eq} = \mathlarger{\sum}_{k=1}^nC_k$
    \end{tabular}
\end{center}
\begin{center}
    \begin{tabular}{c c}
        \includegraphics[scale=0.37]{Image/Induttori in serie.png}
        &
        \includegraphics[scale=0.37]{Image/Induttori in parallelo.png}\\
        $L_{eq} = \mathlarger{\sum}_{k=1}^nL_k$
        &
        $ L_{eq} = \left( \mathlarger{\sum}_{k=1}^n \dfrac{1}{L_k} \right)^{-1} $
    \end{tabular}
\end{center}




\subsection{Elementi circuitali attivi}
Gli elementi attivi sono elementi che possono generare (e fornire) energia elettrica.


\subsubsection{Generatore indipendente di tensione}
\begin{center}
    \includegraphics[scale=0.4]{Image/Gen Tensione ideale.png}
\end{center}
Il \textit{generatore di tensione ideale} mantiene
la tensione tra i suoi terminali indipendentemente dalla
corrente che lo attraversa, quindi la tensione ai capi $V$ è uguale alla tensione interna $e$ (N.B. nel disegno la $v_s$ è la tensione interna $e$).
\[
    p(t) = V \cdot i = e \cdot i 
    \begin{cases} 
        >0 &\text{eroga}\\
        <0 &\text{assorbe}\\
    \end{cases}
\]
Essendo $e$ fisso, la potenza dipende solo da $i$ (variabile di stato).
\vspace*{0.2cm}\\
Per simulare un generatore
di tensione reale (ad es. una batteria) si
considera un resistenza
interna $R_i$ in serie con il
generatore ideale:
\begin{center}
    \includegraphics[scale=0.4]{Image/Gen tensione reale.png}
\end{center}
\[
    V = e - R \cdot i
\]
In realtà nella formula della tensione di una batteria ci sono anche termini dipendenti da altri piccoli condensatori nella stessa.
\vspace*{0.2cm}\\
Se si hanno più generatori di tensione \textbf{in serie}, la tensione totale è semplicemente la somma delle tensioni dei vari generatori.\\
I generatori di tensione \textbf{\underline{non sono collegabili in parallelo}}: se così fosse si avrebbero, nella stessa maglia, due differenti tensioni, ma questo non è possibile perché non viene soddisfatta la LKT.
\vspace*{0.2cm}\\
In un generatore di tensione reale, nel caso in cui la resistenza $R \rightarrow \infty \Longrightarrow i = \dfrac{V}{R} = 0$, mentre se $R \rightarrow 0 \Longrightarrow i=\dfrac{V}{R}= \infty$.



\subsubsection{Generatore indipendente di corrente}
\begin{center}
    \includegraphics[scale=0.4]{Image/Gen corrente ideale.png}
\end{center}
Il \textit{generatore di corrente ideale} mantiene
la corrente che lo attraversa al valore $ A $ indipendentemente dalla
differenza di potenziale fra i suoi terminali (N.B. nel disegno la $i_s$ è la corrente interna $A$).
\[
    p(t) = V \cdot i = V \cdot A 
    \begin{cases}
        >0 &\text{eroga}\\
        <0 &\text{assorbe}
    \end{cases}
\]
Un imprecisa approssimazione di un generatore di corrente reale è un pannello fotovoltaico, dotato di resistenza interna in parallelo con il generatore ideale
\begin{center}
    \includegraphics[scale=0.4]{Image/Gen corrente reale.png}
\end{center}
\[
    i = A -\frac{v}{R}
\]


\subsubsection{Generatori dipendenti}
Gli elementi attivi visti prima possono essere "controllati" in tensione o in corrente.
\vspace*{0.2cm}\\
Il \textbf{generatore dipendente di tensione} è controllato in tensione $V=v_c \cdot  \mu$, o in corrente $V = r \cdot i_c$.
\vspace*{0.2cm}\\
Il \textbf{generatore dipendente di corrente} è controllato in tensione $i = g \cdot v_c$, o in corrente $i = \alpha \cdot i_c$.



\subsection{Circuiti nel dettaglio}
\subsubsection{Circuiti aperti e chiusi}
Un ramo in circuito aperto si può
considerare nei due seguenti modi:
\begin{itemize}
    \item Un generatore di corrente con: $A = 0$
    \item Un resistore con: $R = + \infty$
\end{itemize}
L'equazione dell'elemento circuitale è $i=0, \ \forall v$.
\begin{center}
    \includegraphics[scale=0.35]{Image/Circuito aperto.png}
\end{center}
Un ramo in circuito chiuso si può
considerare nei due seguenti modi:
\begin{itemize}
    \item Un generatore di tensione con: $e = 0$
    \item Un resistore con: $R = 0$
\end{itemize}
L'equazione dell'elemento circuitale è $v=0, \ \forall i$.
\begin{center}
    \includegraphics[scale=0.35]{Image/Circuito chiuso.png}
\end{center}
\begin{center}
    \includegraphics[scale=0.3]{Image/Resistori aperto chiuso.png}
\end{center}


\subsubsection{Esempio 1}
\begin{center}
    \begin{tabular}{c c c c}
    \includegraphics[scale=0.25]{Image/Esempio1_1.png}
    &
    \includegraphics[scale=0.25]{Image/Esempio1_2.png}
    &
    \includegraphics[scale=0.27]{Image/Esempio1_3.png}
    & 
    \includegraphics[scale=0.27]{Image/Esempio1_4.png}\\
    &
    $R_{1eq} = R_2+R_3$
    &
    $R_{2eq} =\dfrac{R_1 R_{1eq}}{R_1+R_{1eq}} = \left( \dfrac{1}{R_1} + \dfrac{1}{R_{1eq}} \right)^{-1}$
    &
    $R_{3eq} = R_4 + R_{2eq} $
\end{tabular}
\end{center}


\subsubsection{Partitore di tensione}
Un partitore di tensione è un tipo di circuito costituito da due o più componenti passivi collegati in serie ai capi dei quali, se viene applicata una tensione, essa si ripartirà sulle stesse componenti in base al loro valore. La legge generale si ottiene moltiplicando il valore della tensione applicata alla serie per il rapporto tra la resistenza ai capi della quale si vuole conoscere la tensione e la somma delle resistenze componenti la serie.
\begin{center}
    \includegraphics[scale=0.35]{Image/Partitore di tensione.png}
\end{center}
\begin{align*}
    R_{eq} &= R_1+R_2 & i = \dfrac{V_s}{R_{eq}} = \dfrac{V_s}{R_1+R_2}
\end{align*}
Dalla LKT si ha
\begin{align*}
    &v_1+v_2 - V_s = 0 \Longrightarrow & v_1&=V_s - v_2 = V_s - R_2 i & v_2 &= V_s-v_1 = V_s - R_1 i\\
    &\underset{\text{sostituisco } i}{\Longrightarrow} & v_1&=\dfrac{R_1}{R_1+R_2} V_s & v_2&=\dfrac{R_2}{R_1+R_2} V_s
\end{align*}


\subsubsection{Partitore di corrente}
Il partitore di corrente è un circuito utilizzato per ottenere la corrente elettrica che scorre attraverso un'impedenza o attraverso un circuito quando esso viene connesso in parallelo con un'altra impedenza. La differenza sostanziale con il partitore di tensione è che in questo caso, nella formula, al numeratore della frazione va la resistenza che non si sta considerando.
\begin{center}
    \includegraphics[scale=0.35]{Image/Partitore di corrente.png}
\end{center}
\begin{align*}
    R_{eq} &= \dfrac{R_1R_2}{R_1+R_2} & v &= R_{eq}I_s = \dfrac{R_1R_2}{R_1+R_2}I_s
\end{align*}
Dalla LKC si ha
\begin{align*}
    &i_1 + i_2 -I_s = 0 \Longrightarrow & i_1 &= I_s - i_2 = I_s - \dfrac{v}{R_2} &  i_2 &= I_s - i_1 = I_s - \dfrac{v}{R_1}\\
    &\underset{\text{sostituisco } v}{\Longrightarrow} & i_1 &= \dfrac{R_2}{R_1+R_2}I_s &  i_2 &= \dfrac{R_1}{R_1+R_2}I_s
\end{align*}






\subsubsection{Collegamenti di resistori a stella e a triangolo}
Un sistema di tre resistenze
può essere collegato a
triangolo o a stella. Può essere
meglio per l'analisi circuitale
una connessione a stella invece
che un triangolo o viceversa.
Da sottolineare che \underline{una rete a stella può essere equivalente ad una rete a} \underline{triangolo}.
\begin{center}
    \includegraphics[scale=0.3]{Image/Stella triangolo.png}
    \includegraphics[scale=0.3]{Image/Nodi triangolo stella.png}
\end{center}
Ciò significa che le stesse tensioni $v_{12}$, $v_{23}$ e $v_{31}$ tra i nodi 1 e
2, i nodi 2 e 3 e i nodi 3 e 1 inducono le
stesse correnti entranti nella
stella e nel triangolo rispettivamente al nodo 1,
al nodo 2 ed al nodo 3.
Ora vediamo come passare da stella a triangolo e viceversa
\begin{center}
    \includegraphics[scale=0.23]{Image/Stella triangolo insieme.png}
\end{center}
Ogni resistenza della stella è il prodotto dei due
resistori del triangolo collegati allo stesso nodo,
diviso per la somma dei resistori a triangolo.
\begin{align*}
    R_{Y1} &= \dfrac{R_{\Delta 1} R_{\Delta 3}}{R_{\Delta 1} + R_{\Delta 2} + R_{\Delta 3}} &
    R_{Y2} &= \dfrac{R_{\Delta 1} R_{\Delta 2}}{R_{\Delta 1} + R_{\Delta 2} + R_{\Delta 3}} &
    R_{Y3} &= \dfrac{R_{\Delta 2} R_{\Delta 3}}{R_{\Delta 1} + R_{\Delta 2} + R_{\Delta 3}}
\end{align*}
Ogni resistenza del triangolo è la somma dei prodotti a due a due di tutti i
resistori della stella, divisa per la resistenza nel ramo opposto della stella.
\begin{align*}
    R_{\Delta 1} &= \dfrac{R_{Y_1}R_{Y_2} +R_{Y2}R_{Y3} + R_{Y3}R_{Y1}}{R_{Y_3}} &
    R_{\Delta 2} &= \dfrac{R_{Y_1}R_{Y_2} +R_{Y2}R_{Y3} + R_{Y3}R_{Y1}}{R_{Y_1}} &
    R_{\Delta 1} &= \dfrac{R_{Y_1}R_{Y_2} +R_{Y2}R_{Y3} + R_{Y3}R_{Y1}}{R_{Y_2}}
\end{align*}
Per $R_{Y1} = R_{Y2} = R_{Y3} = R_Y$ risulta $R_{\Delta 1} = R_{\Delta 2} = R_{\Delta 3} = R_{\Delta}$ e viceversa:
\begin{align*}
    &R_Y = R_{\Delta}/3
    &
    &R_{\Delta} = 3R_Y
\end{align*}





\subsubsection*{Esempio 2}
Determinare la resistenza equivalente del seguente circuito
\begin{center}
    \includegraphics[scale=0.3]{Image/Esempio2_1.png}
\end{center}
\begin{align*}
    R_1&=3 \Omega & R_2&=3 \Omega & R_3&=3 \Omega\\
    R_4&=2 \Omega & R_5&=2 \Omega
\end{align*}
\begin{center}
    \includegraphics[scale=0.3]{Image/Esempio2_2.png}
\end{center}
\begin{align*}
    R_{A\Delta} &= \dfrac{R_1R_2 + R_2R_3 +R_1R_3}{R_1} = 9 \Omega
    &
    R_{B\Delta} &= \dfrac{R_1R_2 + R_2R_3 +R_1R_3}{R_2} = 9 \Omega
    &
    R_{c\Delta} &= \dfrac{R_1R_2 + R_2R_3 +R_1R_3}{R_3} = 9 \Omega
\end{align*}
\begin{center}
    \includegraphics[scale=0.3]{Image/Esempio2_3.png}
\end{center}
\begin{align*}
    R_{B\Delta // R_5} &= \dfrac{R_{B\Delta R_5}}{R_{B\Delta} + R_5} = 1,6364 \Omega & R_{A\Delta // R_4} &= \dfrac{R_{A\Delta R_4}}{R_{A\Delta} + R_4} = 1,6364 \Omega
\end{align*}
\begin{center}
    \includegraphics[scale=0.3]{Image/Esempio2_4.png}
\end{center}
\[
    R_{eq} = R_{C\Delta} // (R_{B\Delta} // R5 + R_{A\Delta} // R4) = (9)//(3,272) = 2,41 \Omega
\]







\section{Metodi di analisi}
\subsection{Metodo di Tanenblau}
Prendiamo un circuito di riferimento
\begin{center}
    \includegraphics[scale=0.35]{Image/Metodo di Tanenblau_1.png}
\end{center}
Definiamo
\begin{itemize} 
    \item Nodo: intersezione tra 3 o più fili
    \item Ramo: Parte di circuito compresa tra due nodi
    \item Maglia: qualsiasi percorso chiuso del circuito
    \item Lato: connessione in successione di più rami
\end{itemize}
Possiamo riformulare la definizione di risoluzione di un circuito: calcolare tutte le tensioni di lato e tutte le correnti di lato.\\
Quindi se $L$ è il numero di lati, avremo $L$ tensioni e $L$ correnti in un circuito; in totale si hanno $2L$ incognite, che necessitano di $2L$ equazioni indipendenti.
\vspace*{0.2cm}\\
Possiamo scrivere $L$ equazioni costitutive dei componenti, le altre $L$ equazioni di ricavano da LKC e LKT.
\begin{center}
    \includegraphics[scale=0.3]{Image/Metodo di Tanenblau_2.png}
\end{center}
\begin{align*}
    &\begin{rcases}
        A(LKC): i_1+i_3-i_2=0\\
        B(LKC): i_4-i_3-i_5=0
    \end{rcases}+ = -i_2 =i_5+i_1+i_4\\
    &C(LKC): i_2+i_5-i_1-i_4=0
\end{align*}
La somma delle prime due equazioni è uguale alle terza con segni invertiti, quindi le tre equazioni non sono indipendenti, solo due di esse lo sono.
\vspace*{0.1cm}\\
Le equazioni che servono per risolvere un circuito sono:
\begin{itemize}
    \item $L$ equazioni costitutive
    \item $N-1$ LKC
    \item $L-N+1$ LKT (maglie non intersecate da rami)
\end{itemize}
Infatti per il circuito avremo
\begin{align*}
    &M_1: v_1-v_2 = 0 & M_3: v_2+v_3 - v_4=0\\
    &M_2: v_4-v_5=0
\end{align*}

\subsubsection*{Passaggi}
\begin{enumerate}
    \item individuare $N$ nodi e $L$ lati
    \item individuare $2L$ incognite ($v,i$)
    \item scrivere le $L$ equazioni di lato
    \item scrivere le $N-1$ LKC e $L-N+1$ LKT
    \item Risolvere il sistema di equazioni
\end{enumerate}

\subsubsection{Esempio}
\begin{center}
    \includegraphics[scale=0.3]{Image/Esempio_Tanenblau.png}
\end{center}

\begin{center}
\begin{tabular}{c|c}
    Equazioni di lato & Equazioni topologiche\\
    \hline
    Lato 1:
    $
    \begin{rcases}
        e_1=E_1\\
        v_{R_1} = R_1 \cdot i_1 
    \end{rcases}
    v_1 = E_1 - R_1 \cdot i_i$
    &
    LKC(A): $i_1+i_2-i_3=0$\\
    Lato 2: 
    $
    \begin{rcases}
        e_2 = E_2\\
        v_{R_2} = R_2 \cdot i_2
    \end{rcases}
    v_2 = E_2 - R_2 \cdot i_2
    $
    &
    LKT($M_1$) : $v_1 - v_3=0$\\
    Lato 3: $v_{R_3} = R_3 \cdot i_3$ 
    &
    LKT($M_2$) : $v_3-v_2=0$    
\end{tabular}
\end{center}
Ora effettuiamo la sostituzione:
\[
    \begin{rcases}
        v_1 - v_3 = E_1-R_1i_1 - R_3i_3=0\\
        v_3 - v_2 = R_3i_3 - E_2 + R_2i_2=0
    \end{rcases}
    \Longrightarrow
    \begin{cases}
        i_1+i_2-i_3=0\\
        R_1i_1+R_3i_3=E_1\\
        R_2i_2+R_3i_3=E_2
    \end{cases}
\]







\subsection{Metodo dei potenziali di nodo}
\begin{enumerate}
    \item scegliere un nodo di rifermento
    \item assegniamo le tensioni agli altri nodi
    \item LKC ai nodi non di riferimento
    \item scrivere le correnti in funzione dei potenziali di nodo
    \item risolvere infine il sistema di equazioni
\end{enumerate}



\subsection{Esempio}
\begin{center}
    \includegraphics[scale=0.12]{Image/Esempio_PotenzialiNodo.png}
\end{center}
Scegliamo come nodo di riferimento il nodo C
\begin{align*}
    \text{LKC(A): }&i_1-i_2 - I - i_3=0\\
    \text{LKC(B): }&i_3 + I - i_4=0\\
\end{align*}
\begin{align*}
    &i_1=\dfrac{E_1-e_A}{R_1} & &i_2 = \dfrac{e_A}{R_2} & &i_3 = \dfrac{e_A-e_B}{R_3}
\end{align*}







\subsection{Teoremi di rete}
\subsubsection*{Ipotesi di linearità}
 $f(x_1+x_2)$ è lineare
\begin{align*}
    \text{additività: } &f(x_1+x_2) = f(x_1) + f(x_2)\\
    \text{omogeneità: } &f(ax_1) = af(x_1)
\end{align*}

\subsection{Sovrapposizione degli effetti}
Ipotesi: circuito lineare.\\
Le variabili di rete (effetti) si possono ottenere come sovrapposizione delle risposte dovute alle singole cause.

\subsubsection{Esempio}
\begin{center}
    \includegraphics[scale=0.2]{Image/Esempio_MetAnalisi_1.png}
\end{center}
Calcoliamo $i_3$.\\
Bisogna considerare una causa alla volta, quindi passivare alcuni generatori: passivando un \underline{generatore di tensione} si ottiene un corto-circuito (quindi il circuito è chiuso), mentre se si passiva un generatore di corrente il circuito risultante è aperto. Per calcolare $v_3$ utilizzeremo la formula del partitore di tensione.
\subsubsection*{Causa \texorpdfstring{$E_1$}{E1}}
\begin{center}
    \includegraphics[scale=0.3]{Image/Esempio_MetAnal_2.png}
\end{center}

\begin{align*}
    &i_3' = i_1 \cdot \dfrac{R_2}{R_2 + R_3} = \dfrac{E_1}{R_1 + R_{eq}'} \cdot \dfrac{R_2}{R_2 + R_3} 
    & i_1 = \dfrac{E_1}{R_1 + R_{eq}'}\\
    &R_{eq}' = R_2 // R_3 = \dfrac{R_2 \cdot R_3}{R_2 + R_3} 
\end{align*}

Calcoli alternativi (più espliciti):
\begin{align*}
    &i_1 = \dfrac{E_1}{R_1 + R_{eq}'} & &v_3= \frac{R_{eq}'}{R_1+R_{eq}'} \cdot E_1 = \frac{R_2 \cdot R_3}{R_2+R_3} \cdot \frac{E_1}{R_1+R_{eq}'} =
    \frac{R_2 \cdot R_3}{R_2+R_3} \cdot i_1 \\
    &R_{eq}' = R_2 // R_3 = \dfrac{R_2 \cdot R_3}{R_2 + R_3} &
    &i_3' = \frac{v_3}{R_3} = \frac{R_2}{R_2+R_3} \cdot i_1
\end{align*}

\subsubsection*{Causa \texorpdfstring{$E_2$}{E2}}
\begin{center}
    \includegraphics[scale=0.25]{Image/Esempio_MetAnalisi_3.png}
\end{center}
\begin{align*}
    &i_2 = \dfrac{E_2}{R_2+R_{eq}'} & &i_3'' = i_2 \cdot \dfrac{R_1}{R_1+R_3}
\end{align*}

Calcoli alternativi (più espliciti):
\begin{align*}
    &i_2=\frac{E_2}{R_2+R_{eq}''} & 
    &v_3 = \frac{R_{eq}''}{R_2+R_{eq}''} \cdot E_2 = \frac{R_1 \cdot R_3}{R_1+R_3} \cdot \frac{E_2}{R_2+R_{eq}''} = \frac{R_1 \cdot R_3}{R_1+R_3} \cdot i_2 \\
    &R_{eq}'' = \frac{R_1 \cdot R_3}{R_1+R_3} &
    &i_3'' = \frac{v_3}{R_3} = \frac{R_1}{R_1+R_3} \cdot i_2 
\end{align*}

Ora possiamo calcolare $i_3$:
\[
    i_3 = i_3' + i_3''
\]

\subsubsection{Esempio 2}
\begin{center}
    \includegraphics[scale=0.3]{Image/Esempio_MetAnalisi_4.png}
\end{center}
\begin{align*}
    &I'= \dfrac{E}{R} & &I'' = - \dfrac{E}{R}
\end{align*}
\begin{align*}
    &I = I' + I'' = \dfrac{E}{R} - \dfrac{E}{R} = 0 & &P=RI^2 = 0
\end{align*}
\begin{align*}
    P &= P' + P'' \\
    &= R (I')^2 + R(I'')^2 \\
    &= R \left( \dfrac{E}{R}\right)^2 + R \left(- \dfrac{E}{R}\right)^2\\
    &=R \dfrac{E^2}{R^2} + R \dfrac{E^2}{R^2} \neq 0
\end{align*}








\subsection{Teorema di Thevenin}
In un circuito lineare tempo-indipendente è messa in evidenza una porta. Il circuito N visto dalla porta, è equivalente ad un circuito formato dalla serie di un generatore indipendente di tensione ed un resistore. La tensione del generatore è data dalla tensione a vuoto
della porta AB del circuito N. Il resistore è il resistore equivalente di N visto dalla porta AB quando tutti i generatori indipendenti interni al circuito N sono spenti.
\vspace*{0.2cm}\\
Ipotesi:
\begin{itemize}
    \item circuito lineare
    \item il carico non è accoppiato con la rete da semplificare
\end{itemize}
\begin{center}
    \includegraphics[scale=0.3]{Image/Thevenin_1.png}
    \includegraphics[scale=0.3]{Image/Thevenin_2.png}
\end{center}
$V_{eq}$ è la tensione vista ai morsetti A e B a circuito aperto.\\
$R_{eq}$ è la resistenza equivalente vista da A e B passivando i generatori indipendenti.


\subsubsection{Esempio}
\begin{center}
    \includegraphics[scale=0.27]{Image/Esempio_Thevenin_1.png}
\end{center}
Applichiamo il teorema di Thevenin ai morsetti A e B:
\begin{center}
    \includegraphics[scale=0.3]{Image/Esempio_Thevenin_2.png}
\end{center}
Utilizzando l'LKT sulla maglia più grande:
\begin{align*}
    E_1 - v_{R_1} + v_{R_2} - E_2=0 &\Rightarrow E_1 - R_1i - R_2i - E_2=0\\
    &\Rightarrow E_1 - E_2 = R_1i + R_2 i \\
    & \Rightarrow E_1 - E_2 = (R_1 + R_2)i\\ 
    & \Rightarrow i = \frac{E_1-E_2}{R_1+R_2}
\end{align*}
mentre l'LKT sulla maglia più piccola a sinistra:
\begin{align*}
    E_1 - R_1i -E_{eq}=0 \underbrace{\Rightarrow}_{i=\frac{E_1-E_2}{R_1+R_2}} E_{eq} &= E_1-R_1 \cdot \frac{E_1-E_2}{R_1+R_2} \\
    &= \frac{\cancel{E_1R_1}+E_1R_2-\cancel{R_1E_1}+R_1E_2}{R_1+R_2}\\
    & = \frac{E_1R_2+E_2R_1}{R_1+R_2}
\end{align*}
\begin{center}
    \includegraphics[scale=0.28]{Image/Esempio_Thevenin_3.png}
\end{center}
\[
    R_{eq} = R_1 // R_2 = \frac{R_1 \cdot R_2}{R_1 + R_2}
\]
\begin{center}
    \includegraphics[scale=0.27]{Image/Esempio_Thevenin_4.png}
\end{center}
\[
    i_3 = \frac{E_{eq}}{R_3+R_{eq}} = \frac{E_1R_2+E_2R_1}{R_1+R_2} \cdot \frac{1}{R_3+R_{eq}}
\]


\subsubsection{Esercizio}
Si consideri il seguente circuito:
\begin{center}
    \includegraphics[scale=0.33]{Image/Esercizio_Thevenin_1.png}
\end{center}
Vogliamo calcolare la corrente nella resistenza $R_5$; utilizzando il teorema di Thevenin otteniamo il circuito
\begin{center}
    \includegraphics[scale=0.33]{Image/Esercizio_Thevenin_2.png}
\end{center}
e il circuito della $R_{eq}$ è il seguente
\begin{center}
    \includegraphics[scale=0.33]{Image/Esercizio_Thevenin_3.png}
\end{center}
Da notare che le resistenze $R_1$ e $R_2$ (come anche $R_3$ e $R_4$) sono in serie viste da B e C, cosa non vera nel circuito ottenuto corto-circuitando la resistenza $R_5$.
\[
    R_{eq} = (R_1//R_2) + (R_3//R_4)
\]
\begin{center}
    \includegraphics[scale=0.3]{Image/Esercizio_Thevenin_4.png}
\end{center}
Per calcolare $E_{eq}$ consideriamo la maglia $M_1$ nel circuito ottenuto corto-circuitando la resistenza $R_5$; è evidente che $e_B$ è concorde a essa, mentre $e_C$ è discorde, quindi (chiamando la tensione del circuito aperto $E_{CA}$)
\begin{align*}
    M_1: e_B - E_{CA} - e_C= 0 \Longrightarrow E_{CA} &= e_B - e_C\\
    &= R_2I_2 - R_4I_3\\
    &=R_2 \dfrac{E}{R_1+R_2} - R_4 \dfrac{E}{R_3+R_4} = E_{eq}
\end{align*}
\[
    I_5 = \frac{E_{eq}}{R_{eq}+R_5}
\]





\subsection{Teorema di Norton}
In un circuito N lineare tempo-indipendente è messa in evidenza una porta. Visto dalla porta
il circuito è equivalente ad un circuito formato dal parallelo di un generatore indipendente di
corrente ed un resistore. La corrente del generatore è data dalla corrente di cortocircuito della porta AB del circuito N. Il resistore è il resistore equivalente di N visto dalla porta AB quando tutti i generatori indipendenti interni al circuito N sono spenti.
\begin{center}
    \includegraphics[scale=0.3]{Image/Norton_1.png}
    \includegraphics[scale=0.3]{Image/Norton_2.png}
\end{center}
Ipotesi:
\begin{itemize}
    \item circuito lineare
    \item il carico non è accoppiato con la rete da semplificare
\end{itemize}
$I_{eq}$ è la corrente tra A e B quando vi è un corto-circuito.\\
$R_{eq}$ è resistenza vista ai morsetti A e B passivando i generatori indipendenti.


\subsubsection{Esempio}
\begin{center}
    \includegraphics[scale=0.25]{Image/Esempio_Norton_0.png}
\end{center}
Vogliamo calcolare $i_3$. Quindi applichiamo il teorema di Norton, utilizzando la formula del partitore di corrente per calcolare $i_3$:
\begin{center}
    \begin{tabular}{c c c}
        \includegraphics[scale=0.25]{Image/Esempio_Norton_1.png}&\includegraphics[scale=0.25]{Image/Esempio_Norton_2.png} & \includegraphics[scale=0.25]{Image/Esempio_Norton_3.png}\\
        $I_{eq} = i_1+i_2 = \dfrac{E_1}{R_1} + \dfrac{E_2}{R_2}$ & $R_{eq} = R_1 // R_2 = \dfrac{R_1 \cdot R_2}{R_1+R_2}$ & $i_3 = I_{eq} \cdot \dfrac{R_{eq}}{R_{eq}+R_3}$
    \end{tabular}
\end{center}





\subsection{Teorema del massimo trasferimento di potenza}
Il circuito equivalente di Thevenin può essere utilizzato per determinare la potenza massima che un circuito lineare può fornire ad una resistore $R_L$ collegata alla porta del circuito.
Consideriamo il seguente circuito
\begin{center}
    \includegraphics[scale=0.3]{Image/T_Massima_Potenza_0.png}
\end{center}
Qual è il valore di $R_L$ che massimizza il trasferimento di potenza? Proviamo a calcolarlo.
\begin{align*}
    P_L &= I \cdot V_L\\
    &=R_L \cdot I^2\\
    &=R_L \left( \dfrac{E}{R+R_L} \right)^2 
    \begin{cases}
        R_L=0 \Rightarrow P_L=0\\
        R_L \rightarrow \infty \Rightarrow P_L=0 \text{ (circuito aperto)}
    \end{cases}
\end{align*}
\begin{center}
    \includegraphics[scale=0.25]{Image/T_Massima_Potenza_1.png}
\end{center}
Per trovare il massimo poniamo la derivata della potenza a 0; sapendo che $P_L = E^2 \dfrac{R_L}{(R+R_L)^2}$
\begin{align*}
    \dfrac{\partial P_L}{\partial R_L}=0 &\Rightarrow E^2 \dfrac{(R+R_L)^2 - 2R_L(R+R_L)}{(R+R_L)^4}=0\\
    & \Rightarrow R^2 + \cancel{R^2_L} + \cancel{2RR_L} - \cancel{2RR_L} -\cancel{2}R^2_L = 0\\
    &\Rightarrow R=R_L
\end{align*} 
Infatti il teorema afferma che:
\begin{center}
    \textit{La potenza massima viene trasferita ad una resistore di carico quando la sua resistenza è uguale alla resistenza equivalente di Thevenin}
\end{center}

\begin{align*}
    P_{L,Max} &= R \dfrac{E^2}{(R+R)^2}\\
    &= \cancel{R} \dfrac{E^2}{4R^{\cancel{2}}}\\
    &= \dfrac{E^2}{4R}
\end{align*}



\subsection{Teorema di Millman}
Data una rete con due o più lati in parallelo, la tensione ai capi della rete è pari al rapporto delle correnti di corto-circuito di ogni singolo lato e la sommatoria delle conduttanze di ogni lato.
\begin{center}
    \includegraphics[scale=0.26]{Image/Th_Millman_0.png}
\end{center}
\[
    v_{AB} = \frac{\sum\limits_k \frac{v_k}{R_k} + \sum\limits_j i_j}{\sum\limits_n\frac{1}{R_n}}
\]

\subsubsection{Esempio 1}
\begin{center}
    \includegraphics[scale=0.25]{Image/Es_Millman.png}
\end{center}
\[
    v_{AB} =\frac{\frac{E_1}{R_1}+\frac{E_2}{R_2} + I_3}{\frac{1}{R_1}+\frac{1}{R_2}+\frac{1}{R_3}}
\]
Dimostriamo che questa formula è vera utilizzando il metodo dei potenziali di nodo (applicato al nodo B):
\begin{center}
    \includegraphics[scale=0.3]{Image/Dim_Es_Millman.png}
\end{center}
\begin{align}
    &v_{AB} = E_1 - R_1i_1 \Rightarrow i_1 = \frac{E_1-v_{AB}}{R_1}\\
    &v_{AB} = -E_2+R_2i_2 \Rightarrow i_2 = \frac{E_2+v_{AB}}{R_2}    \\
    &i_3 = I_3    \\
    &i_4 = \frac{v_{AB}}{R_4}
\end{align}
\begin{align*}
    \text{LKC: } i_1-i_2+i_3-i_4=0 &\Rightarrow \frac{E_1-v_{AB}}{R_1} - \frac{E_2+v_{AB}}{R_2} + I_3 - \frac{v_{AB}}{R_4} = 0\\
    &\Rightarrow \frac{E_1}{R_1} - \frac{v_{AB}}{R_1} - \frac{E_2}{R_2} - \frac{v_{AB}}{R_2} + I_3 - \frac{v_{AB}}{R_4} = 0\\
    &\Rightarrow v_{AB} \left( \frac{1}{R_1} + \frac{1}{R_2} + \frac{1}{R_3} \right) = \frac{E_1}{R_1} - \frac{E_2}{R_2} + I_3\\
    &\Rightarrow v_{AB} = \frac{\frac{E_1}{R_1}+\frac{E_2}{R_2} + I_3}{\frac{1}{R_1}+\frac{1}{R_2}+\frac{1}{R_3}}
\end{align*}

\subsubsection{Esempio 2}
\begin{center}
    \includegraphics[scale=0.3]{Image/Es_Millman_2.png}
\end{center}
Calcoliamo $i_3$:
\[
    v_{AB} = \frac{\frac{E_1}{R_1}+\frac{E_2}{R_2}}{\frac{1}{R_1}+\frac{1}{R_2}+\frac{1}{R_3}} \Rightarrow i_3 = \frac{v_{AB}}{R_3}
\]




\section{Analisi del transitorio}

\subsection{Formule notevoli}
\subsubsection{Condensatore}
\textbf{Condensatore lineare}
\begin{align*}
    C&=\frac{Q}{\Delta V} & C &= \mathcal{E} \cdot \frac{S}{d}
\end{align*}
dove $C$ è la capacità in farad,  $\mathcal{E}$ la permittività elettrica assoluta dell'isolante in farad su metro, $S$ la superficie delle due armature in metri quadrati e $d$ la distanza fra le armature in metri. 
\vspace*{0.2cm}\\
\textbf{Energia immagazzinata}
\begin{align*}
    dW &= V \ dq = \frac{q}{C} \ dq & U &= \frac{1}{2} CV^2 \\
    W&={\frac {1}{2}}C\,V^{2}=U
\end{align*}
L'energia immagazzinata da un condensatore è uguale al lavoro $W$; $U$ è l'energia potenziale.
\vspace*{0.2cm}\\
\textbf{Equazione caratteristica}
\begin{align*}
    v(t) &= \frac{q(t)}{C} = \frac{1}{C} \int_{t_0}^t i(\tau) \ d\tau + v(t_0) & i(t) &= \frac{dq(t)}{dt} = C \frac{dv(t)}{dt}
\end{align*}


\subsubsection{Induttore}
\textbf{Energia}
\[
    W = \frac{1}{2}LI^2
\]
\textbf{Equazione differenziale}
\[
    v(t) = L \frac{di(t)}{dt}
\]
\textbf{Campo magnetico prodotto da un solenoide}
\[
    B = \mu  \frac{N}{l}  I
\]
Una cosa importantissima da tenere a mente sono i circuiti equivalenti di conduttore e induttore in regime stazionario:
\begin{center}
    \includegraphics[scale=0.2]{Image/Es_4_Trans_1.png}
\end{center}



\subsection{Introduzione}
Fin'ora abbiamo usato delle sorgenti costanti, quindi abbiamo fatto esercizi con circuiti DC (circuiti con corrente continua).\\
Consideriamo ora invece la variabile tempo; i nostri circuiti subiranno delle variazioni:
\begin{itemize}
    \item avremo generatori variabili nel tempo
    \item si possono presentare dei guasti nel circuito
    \item il circuito sarà dotato di un interruttore:
    \begin{itemize}
        \item se aperto $i=0 \Rightarrow$ circuito aperto
        \item se chiuso $v=0 \Rightarrow$ cortocircuito
    \end{itemize}
\end{itemize}
Per il \textbf{resistore} $v=R \cdot i$, vi è un legame istantaneo, viene chiamato legame \textbf{adinamico} (senza memoria).\\
Per \textbf{induttori e condensatori} il legame è integro-differenziale, viene chiamato legame \textbf{dinamico} (con memoria).
\vspace*{0.2cm}\\
\begin{align*}
    &\text{Induttore} & \text{Condensatore}\\
    &\mathcal{E} = \frac{1}{2} L \cdot I^2 & \mathcal{E} = \frac{1}{2}C \cdot V^2
\end{align*}
Per l'induttore la variabile di stato è la corrente, mentre per il condensatore la variabile di stato è la tensione.
\vspace*{0.2cm}\\
Ovviamente valgono ancora le formule LKC, LKT e le equazioni costitutive.
\begin{center}
    \includegraphics[scale=0.3]{Image/Es_Trans_0.png}
\end{center}
\[
    \text{LKT:}
    \begin{cases}
        E-v_R-v_L-v_C=0\\
        v_R=R  i\\
        v_L = L \dfrac{di}{dt}\\
        i = C \dfrac{v_L}{dt} \Rightarrow v_C = \dfrac{1}{C} \bigintss i \ dt
    \end{cases}
\]
\[
    E = R  i - L \frac{di}{dt} - \frac{1}{C} \int i \ dt = 0    
\]
Ecco che dobbiamo risolvere un'equazione differenziale.
\vspace*{0.2cm}\\
Per i circuiti lineari, quindi con equazioni differenziali ordinarie (ODE) a coefficienti costanti, utilizzeremo il metodo di Cauchy. La soluzione finale sarà la somma di due contributi:
\begin{itemize}
    \item la soluzione dell'omogenea associata $o(t)$
    \item la soluzione particolare $p(t)$
\end{itemize}


\subsection{Il problema di Cauchy}
In generale la soluzione sarà un espressione di questo tipo
\[
    a_n \frac{d^n x(t)}{dt} + a_{n-1} \frac{d^{n-1} x(t)}{dt} + ... + a_1 \frac{d x(t)}{dt} + a_0 x(t) = b(t)
\]
\begin{itemize}
    \item $b(t)$ è il termine noto
    \item $a_n,a_{n-1},...,a_1,a_0$ sono i coefficienti costanti
    \item $x(t)$ è l'incognita
\end{itemize}
La condizione iniziale del problema di Cauchy è $x(t=0) = X_0$; rappresenta lo stato energetico del sistema prima che si inneschi il transitorio. Per risolvere il problema di Cauchy non solo serve sapere quest'ultimo, ma anche le variabili di stato prima che si inneschi il transitorio.
\vspace*{0.2cm}\\
L'integrale particolare $p(t)$ ha la stessa evoluzione di $b(t)$; ad esempio se $b(t)$ è costante allora anche $p(t)$ è costante ($b(t)$ dipende solo dai generatori).\\
L'integrale dell'omogenea associata $o(t)$ è la soluzione dell'ODE quando il termine noto è nullo, cioè si passivano i generatori. Esso rappresenta l'\textbf{evoluzione libera}:
\[
    a_n \frac{d^n x(t)}{dt} + a_{n-1} \frac{d^{n-1} x(t)}{dt} + ... + a_1 \frac{d x(t)}{dt} + a_0 x(t) = 0
\]
il \textbf{polinomio associato} all'integrale dell'omogenea è
\[
    a_n \lambda ^n + a_{n-1}\lambda ^{n-1} + ... + a_1\lambda + a_0 = 0 \underset{\text{soluzioni}}{\Longrightarrow} \lambda_1,\lambda_2,...,\lambda_n
\]
\[
    o(t) = k_1e^{\lambda_1t} + k_2e^{\lambda_2t} + ... + k_ne^{\lambda_nt} \tag*{$\lambda_1,\lambda_2,...,\lambda_n<0$}
\]
\vspace*{0.2cm}\\
\textbf{Costante di tempo:}
\[
    \tau_n = -\frac{1}{\lambda_n} [s]
\]







\subsubsection{Principio di conservazione dell'energia}
Al fine di inibire variazioni istantanee dell'energia immagazzinata, nei
condensatori la variazione istantanea della tensione è nulla e negli induttori la
variazione istantanea della corrente è nulla. L'energia immagazzinata da un elemento non può variare bruscamente
\begin{center}
\begin{tabular}{m{7cm} m{7cm}}
    Induttore & Condensatore\\
    $\mathcal{E} = \dfrac{1}{2} L \cdot I^2$ & $\mathcal{E} = \dfrac{1}{2}C \cdot V^2 = \dfrac{1}{2}\dfrac{Q^2}{C}$\\
    Nell'induttore variazioni istantanee di $i$ non sono possibili & Nel condensatore variazioni istantanee di $V$ e $Q$ non sono possibili
\end{tabular}
\end{center}


\subsection{Circuito del primo ordine: Circuito RC}
\begin{center}
    \includegraphics[scale=0.4]{Image/RC_0.png}
    \includegraphics[scale=0.4]{Image/RC_1.png}
\end{center}
\begin{align*}
    \text{LKT:}
    \begin{cases}
        E - v_R - v_C = 0\\
        v_R = R \cdot i \\
        i = C \dfrac{dv_C}{dt}
    \end{cases}
    &\Longrightarrow E - RC \frac{dv_C}{dt} - v_C = 0\\
    &\Longrightarrow \frac{dv_C}{dt} + \frac{v_C}{RC} = \frac{E}{RC}
\end{align*}
Perveniamo a una equazione differenziale del primo ordine. Per risolverla dobbiamo trovare la soluzione dell'omogenea associata e la soluzione particolare.
\begin{enumerate}
    \item \textbf{Soluzione dell'omogenea associata}
    \[
        \frac{dv_C}{dt} + \frac{v_C}{RC} = 0
    \]
    il polinomio associato è 
    \[
        \lambda + \frac{1}{RC} = 0 \Rightarrow \lambda = -\frac{1}{RC}
    \]
    quindi la soluzione è
    \[
        o(t) = A e^{-t/RC}
    \]
    $\dfrac{t}{RC}$ è detto numero puro, e la costante di tempo è $\tau = RC$.
    Si può pervenire alla soluzione dell'omogenea associata anche in un altro modo: riprendiamo l'omogenea associata e poniamo sotto integrale ambo i membri:
    \begin{align*}
        \frac{dv_C}{dt} = -\frac{v_C}{RC} &\Rightarrow \int \frac{dv_C}{dt} = -\int\frac{v_C}{RC}\\
        &\Rightarrow \ln(v_C) + k_1 = - \frac{t}{RC} + k_2\\
        &\Rightarrow \ln(v_C) = - \frac{t}{RC} + \underbrace{(k_2 - k_1)}_{k_3}\\
        &\Rightarrow v_C = e^{-t/RC+k_3}\\
        &\Rightarrow v_C = e^{-t/RC}\underbrace{e^{k_3}}_{A}\\
        &\Rightarrow v_C = A e^{-t/RC}
    \end{align*}

    \item \textbf{Soluzione particolare}\\
    Siccome $p(t)$ si comporta come $b(t)$ e $b(t)=\dfrac{E}{RC}$ (una costante), allora $p(t)$ sarà una costante
    \[
        \cancel{\frac{dp(t)}{dt}} + \frac{p(t)}{RC} = \frac{E}{RC} \Rightarrow p(t) = E    
    \]

    \item \[ v_C = o(t) + p(t)  = A e^{-t/RC}+E \]
    
    \item Consideriamo la condizione iniziale
    \begin{align*}
        v_C(t=0^-) &= V_{C0} & v_C(t=0^+) &= A e^{-0/RC} + E = V_{C0}\\
        & & &\Rightarrow A + E = V_{C0}\\
        & & &\Rightarrow A=V_{C0} - E
    \end{align*}

    \item \[ v_C(t) = \underbrace{(V_{C0}-E)e^{-t/RC}}_{\text{Risposta transitoria}} + \underbrace{E}_{\substack{\text{Risposta}\\\text{a regime}}} = \underbrace{V_{C0} \cdot  e^{-t/RC}}_{\T{Evoluzione libera}} + \underbrace{E(1-e^{-t/RC})}_{\T{Risposta formata}} \]
\end{enumerate}
Ricordiamo che $\tau = -\dfrac{1}{\lambda}$, quindi in questo caso $\tau = RC$
\begin{center}
    \includegraphics[scale=0.3]{Image/RC_2.png}
\end{center}
N.B. nel grafico la $V0$ è $E$.
\begin{center}
    \includegraphics[scale=0.18]{Image/RC_3.png}
\end{center}
Dal grafico si può facilmente vedere che $\tau_1<\tau_2$, quindi la tensione è continua.
\[
    e^{-t/\tau}
    \begin{cases}
        \text{Analiticamente è 0 con } t \rightarrow \infty\\
        \text{Praticamente: } e^{-5\cancel{\tau}/\cancel{\tau}} = e^{-5} = \dfrac{1}{e^5}=0,0067\\
    \end{cases}
\]
\begin{center}
    \includegraphics[scale=0.3]{Image/RC_4.png}
\end{center}
\begin{align*}
    i(t) &= C \frac{dv_C}{dt}\\ 
    &= C \frac{d}{dt} \left[ \left(V_{C0}-E\right) e^{-t/RC}+E \right]\\
    &= \cancel{C} \left(\frac{1}{R \cancel{C}}\right) \left(V_{C0}-E\right)e^{-t/RC}\\
    &\Rightarrow i(t) = \frac{E-V_{C0}}{R} e^{-t/RC}
\end{align*}
Dal grafico è evidente che la corrente non varia con continuità.


\subsection{Circuito del primo ordine: Circuito RL}
\begin{center}
    \includegraphics[scale=0.18]{Image/RL_0.png}
    \includegraphics[scale=0.18]{Image/RL_1.png}
\end{center}
L'obiettivo è determinare la risposta naturale del circuito RL (che si ha alla chiusura del derivatore da 1 a 2).\\
Il circuito risultante è il seguente:
\begin{center}
    \includegraphics[scale=0.25]{Image/RL_2.png}
\end{center}
\begin{align*}
    \T{LKT}:&
    \begin{cases}
        E-v_R-v_L = 0\\
        v_R = R \cdot i\\
        v_L=L \dfrac{di}{dt}
    \end{cases}\\
    &\Rightarrow E - R \cdot i - L \frac{di}{dt}=0\\
    &\Rightarrow \frac{di}{dt} + \frac{R}{L}i = \frac{E}{L}
\end{align*}
Siamo arrivati a una equazione differenziale del primo ordine.
\begin{enumerate}
    \item \textbf{Soluzione dell'omogenea associata}
    \[
        \frac{di}{dt} + \frac{R}{L}i = 0
    \]
    il polinomio associato è 
    \[
        \lambda + \frac{R}{L} = 0 \Rightarrow \lambda = -\frac{R}{L}
    \]
    quindi la soluzione è
    \[
        o(t) = A e^{\lambda t} = A e^{-\frac{R}{L}t}
    \]
    la costante di tempo è $\tau = \dfrac{L}{R}$.
    \item \textbf{Soluzione particolare}\\
    Siccome $p(t)$ si comporta come $b(t)$ e $b(t)=\dfrac{E}{L}$ (una costante), allora $p(t)$ sarà una costante
    \[
        \cancel{\frac{dp(t)}{dt}} + \frac{R}{L}p = \frac{E}{L} \Rightarrow p(t) = \frac{E}{R} 
    \]

    \item \[ i(t) = o(t)+p(t) = A e^{-\frac{R}{L}t} + \frac{E}{R}  \] 
    
    \item Consideriamo le condizioni iniziali
    \begin{align*}
        &i(t=0^-) = I_0 & i(t=0^+) = &I_0\\
        &i(0) = A e^{-\frac{R}{L}0} + \frac{E}{R} = I_0\\
        &\Rightarrow A = I_0 - \frac{E}{R}
    \end{align*}

    \item \[ i(t) = \underbrace{\left(I_0-\frac{E}{R}\right)e^{-\frac{R}{L}t}}_{\text{Risposta transitoria}} + \underbrace{\frac{E}{R}}_{\substack{\text{Risposta}\\\text{a regime}}} = \underbrace{I_0 \cdot  e^{-\frac{R}{L}t}}_{\T{Evoluzione libera}} + \underbrace{\frac{E}{R}(1-e^{-\frac{R}{L}t})}_{\T{Risposta formata}} \]
\end{enumerate}
\begin{center}
    \includegraphics[scale=0.25]{Image/RL_3.png}
\end{center}
La corrente varia con continuità.
\begin{align*}
    v_L&=L\frac{di}{dt}\\
    &=L\frac{d}{dt} \left[\left(I_0-\frac{E}{R}\right)e^{-\frac{R}{L}t}+\frac{E}{R}\right]\\
    &=\cancel{L} \left((-\frac{R}{\cancel{L}})\right)\left(I_0-\frac{E}{R}\right)e^{-\frac{R}{L}t}\\
    &\Rightarrow v_L=\left(E-RI_0\right)e^{-\frac{R}{L}t}
\end{align*}
\begin{center}
    \includegraphics[scale=0.3]{Image/RL_4.png}
\end{center}
La tensione non varia con continuità.


\subsection{Confronto tra circuito RC e circuito RL}
\begin{center}
    \begin{tabular}{c c}
        \textbf{RC} & \textbf{RL}\\
        $ \dfrac{dv_c}{dt} + \dfrac{v_c}{RC} = \dfrac{E}{RC} $ & $ \dfrac{di}{dt} + \dfrac{R}{L}i = \dfrac{E}{L} $\\
        $ \dfrac{dv_c}{dt} + \dfrac{v_c}{\tau} = \dfrac{E}{RC} $ & $ \dfrac{di}{dt} + \dfrac{i}{\tau} = \dfrac{E}{L} $\\
    \end{tabular}
\end{center}
\[
    \frac{dx(t)}{dt} + \frac{x(t)}{\tau} = X_F
\]
\[
    x(t) = o(t) + p(t) = \underbrace{X_n(t)}_{\substack{\T{Evouluzione} \\ \T{libera}}} + X_F    
\]
\[
    \begin{cases}
        x(t) = \underbrace{k e^{-t/\tau}}_{Transitorio} + \underbrace{X_{\infty}}_{\T{Regime}}\\
        x(t=0^-) = x(t=0^+) = X_0
    \end{cases}
\]
\begin{align*}
    x(0) &= k+X_{\infty} = X_0 \\
    &\Rightarrow k = X_0 - X_{\infty}\\
    &\Rightarrow x(t) = \left(X_0 - X_{\infty}\right)e^{-t/\tau} + X_{\infty}
\end{align*}



\subsection{Circuiti del secondo ordine: Circuito RLC}
\begin{center}
    \includegraphics[scale=0.25]{Image/RLC_1.png}
    \includegraphics[scale=0.25]{Image/RLC_0.png}
\end{center}
\begin{align*}
    \T{LKT:}
    &\begin{cases}
        E-v_R-v_L-v_C=0\\
        v_R = R i \\
        v_L = L \dfrac{di}{dt}\\
        v_C = \dfrac{1}{C} \bigintsss i(\tau) d \tau
    \end{cases}\\
    &\Rightarrow E - R i - L \frac{di}{dt} - \frac{1}{C} \int i(\tau) d \tau = 0\\
    &\underset{\substack{\T{derviiamo}\\ \T{in }dt}}{\Rightarrow} R \frac{di}{dt} + L \frac{d^2i}{dt} + \frac{1}{C}i = 0\\
    &\Rightarrow \frac{d^2i}{dt} + \frac{R}{L} \frac{di}{dt} + \frac{1}{LC}i = 0
\end{align*}
Siamo arrivati a un'equazione differenziale del secondo ordine.\\
    \textbf{Soluzione dell'omogenea associata}
    \[
        i(t) = Ae^{\lambda_1t} + B e^{\lambda_2t}
    \]
    il polinomio associato è 
    \[
        \lambda^2 + \frac{R}{L} \lambda + \frac{1}{LC} = 0 \Rightarrow \lambda_{1,2} = - \frac{R}{2L} \pm \sqrt{\left(\frac{R}{2L}\right)^2 - \frac{1}{LC}}
    \]
    \begin{align*}
        \frac{R}{2L} &= \alpha \T{: coefficiente di smorzamento}\\
        \frac{1}{LC} &= \omega^2_0 \T{: pulsazione di risonanza}
    \end{align*}
    \begin{align*}
        \lambda_{1,2} &= - \alpha \pm \sqrt{\alpha^2 - \omega^2_0}\\
        &\Rightarrow \Delta = \alpha^2 - \omega^2_0 
        \begin{cases}
            >0 \rightarrow \T{ soluzioini reali distinte}\\
            =0 \rightarrow \T{ soluzioini reali coincidenti}\\
            <0 \rightarrow \T{ soluzioini complesse coniugate}
        \end{cases}
    \end{align*}
    \begin{itemize}
        \item $\Delta > 0$
        \[ \alpha^2 > \omega^2_0 \Rightarrow \alpha > \omega_0 \]
        \[i(t) = A e^{\lambda_1t}+Be^{\lambda_2t}\]
        Analizziamo le condizioni iniziali:
        \begin{align*}
            i(0^-) &= i(0^+) = 0\\
            &\Rightarrow i(0) = 0\\
            &\Rightarrow A+B = 0\\
            &\Rightarrow A = -B 
        \end{align*}
        \begin{align*}
            v_C(0) &= V_{C0}\\
            &\Rightarrow E - \underbrace{Ri(0)}_{0} - L\frac{di}{dt} - V_{C0} = 0\\
            &\Rightarrow E - L \frac{di}{dt}\bigg|_0 - V_{C0} = 0\\
            &\Rightarrow \frac{di}{dt}\bigg|_0 = \frac{E-V_{C0}}{L}
        \end{align*}
        \begin{align*}
            \frac{di}{dt}\bigg|_0 &= \left(A\lambda_1e^{\lambda_1t} +B\lambda_2e^{\lambda_2t} \right) \big|_0\\
            &= A \lambda_1 + B\lambda_2\\
            &\Rightarrow A \lambda_1 + \underbrace{B}_{-A}\lambda_2 = \frac{E-V_{C0}}{L}\\
            &\Rightarrow A(\lambda_1 - \lambda_2) = \frac{E-V_{C0}}{L}\\
            &\Rightarrow A = \frac{E-V_{C0}}{L(\lambda_1-\lambda_2)}
        \end{align*}
        \begin{align*}
            i(t) &= \frac{E-V_{C0}}{L(\lambda_1-\lambda_2)}(e^{\lambda_1 t} - e^{\lambda_2 t})\\
            &= \frac{E-V_{C0}}{2L(\alpha^2 - \omega^2_0)} \left( e^{\left(-\alpha+\sqrt{\alpha^2 - \omega^2_0}\right)t} - e^{\left(-\alpha-\sqrt{\alpha^2 - \omega^2_0}\right)t}\right)\\
            &=\frac{E-V_{C0}}{2L(\alpha^2 - \omega^2_0)} \left( e^{\left(\sqrt{\alpha^2 - \omega^2_0}\right)t} - e^{\left(-\sqrt{\alpha^2 - \omega^2_0}\right)t}\right) e^{-\alpha t}
        \end{align*}
        \textbf{Risposta sovrasmorzata:}
        \[i(t) = \frac{E-V_{C0}}{2L(\alpha^2 - \omega^2_0)} \left( e^{\left(\sqrt{\alpha^2 - \omega^2_0}\right)t} - e^{\left(-\sqrt{\alpha^2 - \omega^2_0}\right)t}\right) e^{-\alpha t}\]
        \begin{center}
            \includegraphics[scale=0.2]{Image/RLC_2.png}
        \end{center}

        \item $\Delta = 0$
        \begin{align*}
            &\alpha^2 = \omega^2 \Rightarrow \alpha = \omega\\
            \underset{\substack{\T{Risposte reali} \\ \T{coincidenti}}}{\Rightarrow} &\lambda_1 = \lambda_2 = \lambda = - \alpha
        \end{align*}
        \[i(t) = \left(A+Bt\right)e^{\lambda t} = \left(A+Bt\right)e^{-\alpha t}\]
        Utilizziamo le condizioni iniziali
        \begin{align*}
            i(t=0^-) &= i(t=0^+) = 0\\
            &\Rightarrow i(0) = A = 0
        \end{align*}
        \begin{align*}
            v_C(t=0^-) &= v_C(t=0^+) = V_{C0} & \frac{di}{dt} \bigg|_0 &= \left(-\alpha(Bt)e^{-\alpha t} + Be^{- \alpha t}\right)\big|_0\\
            &\Rightarrow E - Ri(0) - L\frac{di}{dt} - V_{C0} = 0 & &= B = \frac{E-V_{C0}}{L} \\
            &\Rightarrow \frac{di}{dt} \bigg|_0 = \frac{E-V_{C0}}{L}\\
        \end{align*}
        \textbf{Smorzamento critico:}
        \[i(t) = \frac{E-V_{C0}}{L}t \cdot e^{- \alpha t}\]
        \begin{center}
            \includegraphics[scale=0.3]{Image/RLC_3.png}
        \end{center}

        \item $\Delta < 0$
        \begin{align*}
            \alpha^2 &< \omega^2_0 \\
            &\Rightarrow \alpha < \omega_0
        \end{align*}
        \begin{align*}
            \lambda_1 &= -\alpha + \sqrt{\alpha^2-\omega_0^2} & \lambda_2 &= -\alpha - \sqrt{\alpha^2-\omega_0^2} \\
            &= - \alpha + \sqrt{-\underbrace{(\omega_0^2 -\alpha^2)}_{\substack{\T{Pulsazione}\\\T{nat. smorzata}}}} & &=- \alpha - j \omega_d\\
            &= - \alpha + j \omega_d
        \end{align*}
        \[i(t) = \frac{E-V_{C0}}{2Lj \omega_d} \left(e^{j \omega_d t} - e^{-j \omega_d t}\right)e^{-\alpha t}\]
        Utilizziamo le formule di eulero:
        \begin{align*}
            e^{j \theta} &= \cos \theta + j \sin \theta\\
            e^{-j \theta} &= \cos \theta - j \sin \theta
        \end{align*}
        \[i(t) = \frac{E-V_{C0}}{\cancel{2}L\cancel{j} \omega_d} \cancel{j} \cancel{2} \sin (\omega_d t) e^{-\alpha t}\]
        \textbf{Risposta sottosmorzata:}
        \[i(t) = \frac{E-V_{C0}}{L \omega_d}  \sin (\omega_d t) e^{-\alpha t}\]
        \begin{center}
            \includegraphics[scale=0.25]{Image/RLC_4.png}
        \end{center}
    \end{itemize}
Quindi, riassumendo:
\begin{center}
    \begin{tabular}{|c|c|c|}
        \hline
        \textbf{Risposta sovrasmorzata} & \textbf{Smorzamento critico} & \textbf{Risposta sottosmorzata}\\ \hline
        $\Delta>0$ & $\Delta=0$ & $\Delta<0$\\ \hline 
        %equazioni
        $i(t) = \dfrac{E-V_{C0}}{2L(\alpha^2 - \omega^2_0)} \left( e^{\left(\sqrt{\alpha^2 - \omega^2_0}\right)t} - e^{\left(-\sqrt{\alpha^2 - \omega^2_0}\right)t}\right) e^{-\alpha t}$
        &
        $i(t) = \dfrac{E-V_{C0}}{L}t \cdot e^{- \alpha t}$
        &
        $i(t) = \dfrac{E-V_{C0}}{L \omega_d}  \sin (\omega_d t) e^{-\alpha t}$
        \\ \hline
    \end{tabular}
\end{center}

\subsection{Esercizi in regime transitorio}
Risolvere un'equazione differenziale di ordine superiore al secondo è complesso; in questi casi si usa il \textbf{metodo per ispezione}.
\begin{center}
    \includegraphics[scale=0.3]{Image/Es_0_Trans.png}
\end{center}
Siamo interessati a 
\begin{itemize}
    \item $t=0^-$: istante immediatamente prima che si instauri il transitorio
    \item $t=0^+$: istante immediatamente dopo l'instaurasi del transitorio
    \item $t=\infty$: regime
\end{itemize}


\subsubsection{Esercizio 1: Circuito RC}
\begin{center}
    \includegraphics[scale=0.35]{Image/Es_1_Trans_1.png}
\end{center}
\begin{align*}
    R_1 &= 100 \Omega\\
    C_1 &= 100 F\\
    E_1 &= 10V\\
\end{align*}
T si chiude a $t=0$; il condensatore è inizialmente scarico.
\vspace*{0.1cm}\\
Dobbiamo trovare:
\begin{itemize}
    \item $P_{E_1,max}$ e $t_{P_{E_1}, max}$
    \item $P_{E_1}(t=\infty)$
    \item $t_{P_\delta^*}$ e $P_\delta^* = \dfrac{1}{2}P_{E,max}$
\end{itemize}
\[P_{E_1} = E_1 \cdot i_1\]
\begin{align*}
    t=0^- &: v_C(t=0^-) = 0V\\
    t=0^+ &: v_C(t=0^+) = v_C(t=0^-) = 0V\\
    t= \infty &: v_C(t = \infty) = E_1 = 10V
\end{align*}
Ricordiamo la formula trovata dal confronto tra circuito RC e RL:
\[x(t) = \left(X_0-X_{\infty}\right)e^{-t/\tau} + X_{\infty}\]
\begin{align*}
    v_C(t) &= (0-10)e^{-t/\tau} + 10V & i_C &= C\frac{dv_c}{dt} & \tau &= R_1C \\
    &= 10\left(1 - e^{-t/\tau}\right) & &=C \frac{d}{dt}\left[10 \left(1-e^{-t/\tau}\right)\right] & &=10ms\\
    & & &=C\cdot 10 \cdot \frac{1}{\tau} \cdot e^{-t/\tau}\\
    & & &=\frac{10}{R} e^{-t/\tau}\\
    & & &=0,1 e^{-t/\tau} [A]
\end{align*}
\begin{center}
    \includegraphics[scale=0.35]{Image/Es_1_Trans_2.png}
\end{center}
La corrente ha il suo massimo in $t=0$, dopo diminuisce in maniera esponenziale $\Rightarrow I_{1,max} = 0,1 A$
\begin{align*}
    P_{E_1,max} &= E_1 \cdot I_{1,max} & t_{P_{E_1}, max} = 0 \\
    &=10 V \cdot 0,1 A \\
    &=1 W
\end{align*}
\begin{align*}
    P_{E_1}(t=\infty) &= E_1 \cdot I_1(t=\infty)\\
    &=10V \cdot 0A\\
    &=0
\end{align*}
\begin{align*}
    P_{\delta_1} &= E_1 \cdot i_C & e^{-t_{P^*_\delta}/\tau} &=0,5 & t_{P^*_\delta} &=-RC \ln(0,5) \\
    &= E_1 \cdot  0,1 e^{-t_{P_\delta}/\tau} & &\Rightarrow -\frac{t_{P^*_\delta}}{RC} = \ln(0,5) & &=RC(0,69)\\
    &= \frac{P_{\delta_1, max}}{2}  & & & &=0,01\cdot(0,69) \\
    &=0,5W &  & & &=6,9ms
\end{align*}


\subsubsection{Esercizio 2: Circuito RL}
\begin{center}
    \includegraphics[scale=0.4]{Image/Es_2_Trans_0.png}
\end{center}
\begin{align*}
    &E_1 = 12V\\
    &R_1 = R_2 = 10 \Omega\\
    &L = 0,5 H
\end{align*}
L'induttanza inizialmente è scarica; T si chiude in $t=0$.
Bisogna trovare:
\begin{enumerate}
    \item $I_{E_1}(t=0^-)$
    \item $I_{E_1}(t=\infty)$
    \item Espressione analitica di $P_{E_1}$
\end{enumerate}

\begin{enumerate}[label=\protect\circled{\arabic*}]
    \item $t=0^-$
    \begin{center}
        \includegraphics[scale=0.4]{Image/Es_2_Trans_1.png}
    \end{center}
    \begin{align*}
        I_{E_1}(t=0^-) &=\frac{E_1}{R_1} \\
        &=1,2A
    \end{align*}
    \item $t=\infty$
    \begin{center}
        \includegraphics[scale=0.4]{Image/Es_2_Trans_2.png}
    \end{center}
    \begin{align*}
        I_{E_2}(t=\infty) &= \frac{E_1}{R_1//R_2} & i_L(t=\infty) &= \frac{E_1}{R_2}\\
        &=\frac{12V}{5\Omega} & &=1,2A\\
        &= 2,4 A
    \end{align*}
    \item $P_{E_1} = E_1 \cdot i_1 = E_1[i_{R_1}+i_L]$
    \begin{center}
        \includegraphics[scale=0.4]{Image/Es_2_Trans_3.png}
    \end{center}
    \begin{align*}
        i_L(t) &= \left(I_{L,0} - I_{L,\infty}\right)e^{-t/\tau} + I_{L, \infty} & \tau &= \frac{L}{R} \\
        &= (0-1,2)e^{-t/\tau} + 1,2 & &=\frac{0,5H}{10\Omega}\\
        &=1,2(1-e^{-t/\tau}) & &=50ms
    \end{align*}
    \begin{align*}
        P_{E_1} &= E_1 \left[\frac{E_1}{R_1}+1,2(1-e^{-t/\tau})\right]\\
        &=12[1,2+1,2 - 1,2e^{-t/\tau}]\\
        &= 12[2,4 - 1,2e^{-t/\tau}]\\
        &=28,8 - 14,4 e^{-t/\tau} W
    \end{align*}
\end{enumerate}


\subsubsection{Esercizio 3: Circuito RLC complesso}
\begin{center}
    \includegraphics[scale=0.26]{Image/Es_3_Trans_0.png}
\end{center}
\begin{align*}
    &E_2=12V\\
    &R_1=R_2=2\Omega\\
    &R_3=R_4=4\Omega\\
    &C_1=10\mu F\\
    &L_1=2mH
\end{align*}
T si chiude in $t=0$.
\vspace*{0.2cm}\\
Bisogna trovare:
\begin{enumerate}
    \item $t=0^-:Q_{C_1}$
    \item $t=0^+:i_{C_1}; \ \dfrac{di_L}{dt}$
    \item $t=\infty:\mathcal{E}_{L_1}; \ \mathcal{E}_{C_1}$
\end{enumerate}
\begin{enumerate}[label=\protect\circled{\arabic*}]
    \item $t=0^-$ siamo in regime stazionario, quindi $\dfrac{\partial}{\partial t}=0$
    \begin{center}
        \includegraphics[scale=0.2]{Image/Es_3_Trans_1.png}
    \end{center}
    \begin{align*}
        i_{L_1}(t=0^-) &=\frac{E_2}{R_2+R_4} & v_{C_1} (t=0^-) &=R_4\cdot i_{L_1} \\
        &= 2A & &=8V
    \end{align*}
    \begin{align*}
        Q_{C_1}(t=0^-) &=C_1 \cdot v_{C_1}(t=0^-)\\
        &=10\mu F \cdot 8V \\
        &=8\cdot 10^{-5}C
    \end{align*}
    \item $t=0^+$
    \begin{align*}
        i_{L_1}(t=0^+) &= i_{L_1}(t=0^-) & v_{C_1}(t=0^+) &= v_{C_1}(t=0^-)\\
        &=I_{L,0} & &=V_{C,0}\\
        &=2A & &=8V
    \end{align*}
    \begin{center}
        \includegraphics[scale=0.2]{Image/Es_3_Trans_2.png}
    \end{center}
    \begin{align*}
       M1&:v_{R_1}+V_{C,0} - E_2 + v_{R_2}=0 & & & &R_1\cdot i_{C_1} + V_{C,0}-E_2+R_2\cdot i_{E_2}=0\\
        M_2&:-v_{R_2}+E_2-v_{R_P}-v_L=0 & &\Rightarrow & & -R_2\cdot i_{E_2}+E_2-R_P\cdot I_{L,0}-v_L=0\\
        LKC&:I_{L,0}+i_{C_1}-i_{E_2} & & & &
        I_{L,0} + i_{C_1} - i_{E_2}=0 & &\Rightarrow & &i_{E_2} = I_{L,0}+i_{C_1}
    \end{align*}
    Sostituisco $i_{E_2}$ nella prima equazione
    \begin{align*}
        &R_1\cdot i_{C_1} + V_{C,0}-E_2+R_2\left(I_{L,0}+i_C\right)=0\\
        \Rightarrow& i_{C_1}(R_1+R_2) = E_2-V_{C,0}-R_2I_{L,0}\\
        \Rightarrow& i_{C_1} = \frac{E_2-V_{C,0}-R_2I_{L,0}}{R_1+R_2}\\
        \Rightarrow& i_{C_1} = \frac{12V-8V-2 \cdot 2}{2\Omega+2\Omega}\\
        \Rightarrow& i_{C_1}=0A
    \end{align*}
    Quindi $i_{E_2}=I_{L,0}$; sostituisco questo nell'equazione della seconda maglia
    \begin{align*}
        -R_2\cdot I_{L,0}+E_2-R_P\cdot I_{L,0}-v_L&=0\\
        \Rightarrow v_L&=E_2-(R_2+R_P)I_{L,0}\\
        &=4V
    \end{align*}
    Ricordando l'equazione differenziale del'induttore $v(t) = L \dfrac{di(t)}{dt}$, da essa si può calcolare la derivata della corrente $\dfrac{di(t)}{dt} = \dfrac{v(t)}{L}$
    \begin{align*}
        \frac{di_L}{dt}(t=0^+) &= \frac{v_L(t=0^+)}{L} \\
        &=\frac{4V}{2\cdot 10^{-3}H}\\
        &=2 \ kA/s
    \end{align*}
    \item $t = \infty$
    \begin{center}
        \includegraphics[scale=0.25]{Image/Es_3_Trans_3.png}
    \end{center}
    \begin{align*}
        i_L(t=\infty) &= \frac{E_2}{R_2+R_P} & v_{C_1}(t=\infty) &= R_P\cdot i_L\\
        &=3A & &=6V
    \end{align*}
    \begin{align*}
        \mathcal{E}_L &= \frac{1}{2}LI^2 & \mathcal{E}_C&=\frac{1}{2}CV^2 \\
        &=\frac{1}{2}\cdot 2 \times 10^{-3} \cdot(3)^2 & &=\frac{1}{2}\cdot 10 \times 10^{-6}\cdot(6)^2\\
        &=9 \ mJ & &= 180 \ \mu J 
    \end{align*}
\end{enumerate}



\subsubsection{Esercizio 4: Circuito RLC complesso}
\begin{center}
    \includegraphics[scale=0.27]{Image/Es_4_Trans_0.png}
\end{center}
T si chiude all'istante $t=0$.\\
Dobbiamo calcolare 
\begin{enumerate}
    \item $t=0^-:\mathcal{E}_{C_2}$ 
    \item $t=0^+: \dfrac{di_L}{dt}; \ P_{E_1}$
    \item $t=\infty: \Varepsilon_{C_1}; \ \Varepsilon_{C_2}; \ \Varepsilon_{L}$
\end{enumerate}
Il generatore di tensione $E_1$ è costante, quindi abbiamo un regime stazionario.
\begin{enumerate}[label=\protect\circled{\arabic*}]
    \item $t=0^-$
    Siccome T è aperto e siamo in un regime stazionario $\dfrac{\partial}{\partial t}=0$
    \begin{center}
        \includegraphics[scale=0.2]{Image/Es_4_Trans_1.png}
    \end{center}
    \begin{align*}
        v_L &=L\frac{di}{dt} \underset{\frac{\partial}{\partial t}=0}{\Rightarrow}v_L=0 & i_C &= C \frac{dv_C}{dt} \underset{\frac{\partial}{\partial t}=0}{\Rightarrow} i_C=0
    \end{align*}
    \begin{center}
        \includegraphics[scale=0.2]{Image/Es_4_Trans_2.png}
    \end{center}
    Come prima cosa calcoliamo le variabili di stato all'istante $t=0^-$:
    \begin{align*}
        v_{C_1}(t=0^-) &= 0V & i_L(t=0^-) &= \frac{E_1}{R_1+R_2} & v_{C_2}(t=0^-) &= v_{R_2} = R_2 \cdot i_L\\
        & & & & &=E_1 \frac{R_2}{R_1+R_2}
    \end{align*}
    \[
        \Varepsilon_{C_2} = \frac{1}{2}V_C^2=\frac{1}{2}C\left(E_1\frac{R_2}{R_1+R_2}\right)
    \]
    \item $t=0^+$
    \begin{align*}
        v_{C_1}(t=0^+)&=v_{C_1}(t=0^-) & v_{C_2}(t=0^+)&=v_{C_2}(t=0^-) & i_L(t=0^+)&=i_L(t=0^-)\\
        &=0V & &=v_{C_2,0} & &=i_{L,0}\\
        & & &=E_1\frac{R_2}{R_1+R_2} & &=\frac{E_1}{R_1+R_2}
    \end{align*}
    Manteniamo la convenzione dell'utilizzatore
    \begin{center}
        \includegraphics[scale=0.3]{Image/Es_4_Trans_3.png}
        \includegraphics[scale=0.3]{Image/Es_4_Trans_4.png}
    \end{center}
    \begin{align*}
        \frac{di_L}{dt}\Bigg|_{t=0^+} &= \frac{v_L}{L}\Bigg|_{t=0^+} & P_{E_1} &= E_1\cdot I_1 \\
        &=\frac{v_{C_1,0}}{L} & &=E_1\left(\frac{E_1-v_{C_2,0}}{R_1}\right) \\
        &=0
    \end{align*}
    \item $t=\infty$, siamo sempre in regime stazionario
    \begin{center}
        \includegraphics[scale=0.3]{Image/Es_4_Trans_6.png}
    \end{center}
    \begin{align*}
        v_{C_1}(t=\infty) &= 0V & I_1 &= \frac{E_1}{R_1+R_P} \\ 
        i_L(t=\infty) &=\frac{E_1}{R_1+R_P}\cdot \frac{R_3}{R_2+R_3} & i_L &=I_1 \frac{R_3}{R_2+R_P} \\
        v_{C_2}(t=\infty) &= R_2\cdot i_L
    \end{align*}
    \begin{align*}
        \Varepsilon_{C_1} &= \frac{1}{2}C_1V_{C_1}^2 & \Varepsilon_{C_2} &= \frac{1}{2}C_2V_{C_2}^2 & \Varepsilon_L &= \frac{1}{2}L\cdot I_L^2\\
        &=\frac{1}{2}C_1(0)^2 & &=\frac{1}{2}C_2(R_2\cdot i_L)^2 [J] & &=\frac{1}{2}L\left(\frac{E_1}{R_1+R_P}\cdot \frac{R_3}{R_2+R_3}\right) [J]\\
        &=0 \ J
    \end{align*}
    La R per la quale ho smorzamento critico è
    \begin{align*}
        \alpha &= \omega_0 \\
        &\Rightarrow \frac{R}{2L} = \frac{1}{\sqrt{LC}}\\
        &\Rightarrow R = \frac{2L}{\sqrt{LC}}
    \end{align*}
\end{enumerate}





\section{Regime sinusoidale}
\subsection{Introduzione}
La \textbf{corrente alternata} (CA o AC dall'inglese: Alternating Current) è un tipo di corrente elettrica nella quale il flusso di elettroni inverte la propria direzione continuativamente nel tempo. A differenza della corrente continua, in cui il flusso degli elettroni è unidirezionale, nella corrente alternata il flusso di elettroni inverte la propria direzione, con un'alternanza che avviene con frequenza fissa (tipicamente 50 Hz o 60 Hz). L'inversione di polarità però non avviene in modo repentino ma con una variazione progressiva secondo un andamento chiamato \textbf{sinusoidale}, nel quale il valore di corrente (e quindi di tensione) parte da zero e gradualmente aumenta in un determinato verso, raggiunge il suo valore massimo per poi diminuire fino a zero e riprendere nel verso opposto col medesimo andamento.

\subsubsection{Segnali periodici}
Una grandezza temporale $a(t)$ si definisce periodica quando si ripete a ogni periodo $T$
\[a(t)=a(t+nT) \tag*{$\forall t$}\]
dove:
\begin{itemize}
    \item $T$ è il \textit{periodo}, in $[s]$;
    \item $f=\dfrac{1}{T}$ è la \textit{frequenza}, in $\T{Hz}$ (numero di periodi nell'unità di tempo)
    \item $A_m=\dfrac{1}{T}\bigintss_0^Ta(t)\ dt$ è il \textit{valore medio}
    \item $A_{eff}=\sqrt{\dfrac{1}{T}\bigintss_0^Ta^2(t)\ dt}$
\end{itemize}

\subsection{Sinusoide}
Si definisce come grandezza sinusoidale nel tempo quanto segue:
\[a(t)=\hat{A} \cos(\omega t+\theta)\]
dove:
\begin{itemize}
    \item $\hat A$ è l'\textit{ampiezza}, il massimo di $a(t)$, reale positivo;
    \item $\omega$ è la \textit{frequenza angolare}, in $[s^{-1}]$, reale positivo;
    \item $\theta$ è la \textit{fase}, in $[\rad]$;
    \item $T = \dfrac{2\pi}{\omega}$ è il \textit{periodo}, in $[s]$;
    \item $f=\dfrac{1}{T} = \dfrac{\omega}{2\pi}$ è la \textit{frequenza naturale}, in $[\T{Hz}]$;
    \item $A_{eff} = \sqrt{\dfrac{1}{T}\bigintss_t^{t+T}\hat A^2\cos^2(\omega t + \theta)dt}=\dfrac{\hat A}{\sqrt{2}}=0,707 \hat A$ è il \textit{valore efficace}, o radice della media dei quadrati, in inglese \textit{root mean square}, abbreviato in RMS
\end{itemize}
\begin{center}
    \includegraphics[scale=0.3]{Image/Regime_sinusoidale_0.png}
\end{center}
Usiamo il cappello ($\hat A$) per distinguere il valore di picco dal valore efficace.

\subsubsection{Potenza media dissipata da una resistenza}
\begin{align*}
    p(t) &= R\cdot i^2(t) & i(t) &= \hat I \cos (\omega t + \alpha)
\end{align*}
la potenza media dissipata è
\begin{align*}
    <p(t)> &= \dfrac{1}{T} \int_0^T R \cdot i^2(t) \ dt\\
    &=R\underbrace{\dfrac{1}{T} \int_0^T i^2(t) \ dt}_{I_{eff}^2}\\
    &=RI_{eff}^2
\end{align*}
Una corrente DC di 1A comporta una potenza dissipata sul resistore pari a quella di una corrente sinusoidale di valore efficace pari a 1A.


\subsubsection{Segnali isofrequenziali}
Una sinusoide è completamente definita da 3 parametri:
\begin{itemize}
    \item valore massimo $\hat A$ o valore efficace $A_{eff} = \dfrac{\hat A}{\sqrt{2}}$;
    \item pulsazione $\omega$ o frequenza $f$ o periodo $T$: $f=\dfrac{1}{T}; \ \omega=2\pi f$;
    \item fase $\alpha$
\end{itemize}
\[
    \begin{rcases}
        a(t) = \hat A \cos(\omega t + \alpha)\\
        b(t) = \hat B \cos(\omega t + \beta)
    \end{rcases}
    \substack{\T{Segnali}\\ \T{isofrequenziali}}
\]
\begin{center}
    \includegraphics[scale=0.3]{Image/Regime_sinusoidale_1.png}
\end{center}
Viene definito \textit{sfasamento} o \textit{differenza di fase}:
\[
    \phi = \alpha - \beta
\]
\begin{itemize}
    \item $\phi = 0$: $a(t)$ e $b(t)$ sono \textbf{in fase}
    \begin{center}
        \includegraphics[scale=0.2]{Image/Regime_sinusoidale_2.png}
    \end{center}
    \item $\phi >0$:
    \begin{itemize}
        \item $a(t)$ è in \textbf{anticipo} rispetto a $b(t)$
        \item $b(t)$ è \textbf{ritardo} rispetto a $a(t)$
    \end{itemize}
    \item $\phi <0$:
    \begin{itemize}
        \item $a(t)$ è in \textbf{ritardo} rispetto a $b(t)$
        \item $b(t)$ è \textbf{anticipo} rispetto a $a(t)$
    \end{itemize}
    \item $\phi = \pm \pi$: $a(t)$ e $b(t)$ sono in \textbf{opposizione di fase}
    \begin{center}
        \includegraphics[scale=0.2]{Image/Regime_sinusoidale_3.png}
    \end{center}
    \item $\phi = \pm \dfrac{\pi}{2}$: $a(t)$ e $b(t)$ sono in \textbf{quadratura}
    \begin{center}
        \includegraphics[scale=0.2]{Image/Regime_sinusoidale_4.png}
    \end{center}
    nella figura $b(t)$ è in quadratura in ritardo.
\end{itemize}




\subsection{Fasori}
I \textit{fasori} rappresentano una grandezza sinusoidale tramite un numero complesso (un vettore).
\vspace*{0.1cm}\\
\textbf{Identità di Eulero:}
\[
    e^{jx} = \cos(x) + j\sin(x)
\]
l'espressione caratteristica di un fasore così diventa
\[
    a(t)=\hat A \cos(\omega t + \alpha) = \T{Re}\left[\hat A e^{j(\omega t + \alpha)}\right]
\]
l'argomento della parte reale è un vettore rotante nel piano complesso
\begin{center}
    \animategraphics[autoplay,loop,width=3cm]{12}{Image/Unfasor.png/eda58be88a8b4f9caa208b92f2948807JZ1I0VnWz4iD1ZLQ-}{0}{47}
\end{center}
Nella figura:
\begin{itemize}
    \item l'asse orizzontale della circonferenza è l'asse dei numeri \textit{immaginari}, l'asse verticale è l'asse dei numeri \textit{reali};
    \item il grafico disegnato sopra è il nostro $a(t)$
\end{itemize}

Prendiamo dei fasori isofrequenziali:
\begin{align*}
    a_1&=\hat A_1\cos(\omega t + \alpha_1) & &\longleftrightarrow & \hat A_1 e^{j(\omega t + \alpha_1)} &= \hat A_1 \underline{e^{j \omega t}}e^{j \alpha_1}=\sqrt{2}A_{1,e} \cdot e^{j \omega t} \cdot e^{j \alpha_1}\\
    a_2&=\hat A_2\cos(\omega t + \alpha_2) & &\longleftrightarrow & \hat A_2 e^{j(\omega t + \alpha_2)} &= \hat A_2 \underline{e^{j \omega t}}e^{j \alpha_2}=\sqrt{2}A_{2,e} \cdot e^{j \omega t} \cdot e^{j \alpha_2}\\
    &...\\
    &...\\
    &...\\
    a_n&=\hat A_n\cos(\omega t + \alpha_n) & &\longleftrightarrow & \hat A_n e^{j(\omega t + \alpha_n)} &= \hat A_n \underline{e^{j \omega t}}e^{j \alpha_n}=\sqrt{2}A_{n,e} \cdot e^{j \omega t} \cdot e^{j \alpha_n}
\end{align*}
Il valore sottolineato è chiamato \textbf{fattore rotante}.
\vspace*{0.1cm}\\
Tutti i vettori ruotano alla stessa pulsazione in senso antiorario nel piano complesso; un fasore può essere scritto nella forma
\[
    \underline{A} = A_e \cdot e^{j \alpha}
\]
\begin{itemize}
    \item $\underline{A}$ è il vettore nel piano complesso
    \item $A_e$ è il valore efficace
    \item $\alpha$ è la fase iniziale
\end{itemize}
Il \textbf{metodo fasoriale}, cioè risolvere esercizi con corrente sinusoidale scrivendo le equazioni dei generatori in forma fasoriale, vale per segnali isofrequenziali.
\begin{center}
    \includegraphics[scale=0.2]{Image/Regime_sinusoidale_5.png}
\end{center}



\subsection{Trasformata di Steinmetz}
\[
    S[a(t)] = \frac{\sqrt{2}}{T} \int_0^T a(t) e^{-j \omega t} \ dt
\]
siccome $a(t) = \hat A \cos(\omega t + \alpha)$
\begin{align*}
    S[a(t)] &= \frac{\sqrt{2}}{T} \int_0^T \hat A \cos(\omega t + \alpha)e^{-j \omega t} \ dt\\
    &= \frac{\sqrt{2}}{T} \hat A \int_0^T \frac{e^{j(\omega t + \alpha)}+e^{-j(\omega t + \alpha)}}{2} e^{-j \omega t} \ dt\\
    &=\frac{\sqrt{2}}{T} \frac{\hat A}{2} \int_0^T \left[e^{j(\omega t + \alpha)}+e^{-j(\omega t + \alpha)}\right]e^{-j \omega t} \ dt\\
    &=\frac{\sqrt{2}}{T} \frac{\hat A}{2} \left[ \int_0^T e^{j \alpha} \ dt + \cancel{\int_0^T e^{-2j \omega t - j \alpha} \ dt} \right] \substack{\Longleftarrow \T{un fasore integrato su un} \\ \T{ periodo con } \omega \neq 0 \T{ vale 0}}\\
    &=\frac{\hat A}{\sqrt{2}\cancel{T}} e^{j \alpha} \cancel{T}\\
    &= \frac{\hat A}{\sqrt{2}}e^{j \alpha}\\
    &=\underbrace{A_e \cdot e^{j \alpha}}_{\T{Fasore}}
\end{align*}
La \textit{trasformata di Steinmetz} associa a una grandezza nel dominio dei tempi una grandezza nel dominio fasoriale.
\vspace*{0.2cm}\\
La trasformata di Steinmetz è un operatore lineare:
\[
    S [m a(t) + n b(t) ] = m S[a(t)]+nS[b(t)]
\]
\[
    S\left[\frac{da(t)}{dt}\right] = \frac{\sqrt{2}}{T} \int_0^T \frac{da(t)}{dt}e^{-j \omega t} \ dt
\]
integrando per parti:
\begin{align*}
    S\left[\frac{da(t)}{dt}\right] &= \frac{\sqrt{2}}{T} \left[\cancel{\left[a(t)e^{-j \omega t}\right]^T_0}- \int_0^T a(t) \cdot (-j \omega e^{-j \omega t}) \ dt\right] \\
    &=\frac{\sqrt{2}}{T} \left[j \omega \int_0^T  a(t)e^{-j \omega t} \ dt\right] \\
    &= j\omega \underbrace{\frac{\sqrt{2}}{T} \int_0^T a(t) e^{-j \omega t} \ dt}_{S[a(t)]}\\
\end{align*}
\[
    S\left[\frac{da(t)}{dt}\right] = j \omega S[a(t)]
\]
$j \omega$ è lo sfasamento.


\subsection{Operazioni sui numeri complessi}
Un fasore può essere rappresentato un due modi:
\begin{itemize}
    \item rappresentazione cartesiana: $\underline{A} = A_R + j A_I$
    \item rappresentazione polare: $\underline{A} = A_e \cdot e^{j \alpha}$
\end{itemize}
\begin{center}
   \begin{tabular}{c|c}
    Polare $\rightarrow$ Cartesiana & Cartesiana $\rightarrow$ Polare\\
    $ Ae^{j \alpha} \longrightarrow 
    \begin{cases}
        A_R = A \cos(\alpha)\\
        A_I = A \sin(\alpha)
    \end{cases} $
    &
    $A_R+jA_I \longrightarrow 
    \begin{cases}
        A = \sqrt{A_R^2+A_I^2}\\
        \alpha = \arctan \left(\dfrac{A_I}{A_R}\right)
    \end{cases}$
    \end{tabular} 
\end{center}
\begin{align*}
    \underline{A} &= A_R+j A_I = Ae^{j \alpha} & \textbf{B} &= B_R + j B_I = Be^{j\beta}
\end{align*}

\subsubsection*{Somma}
\begin{align*}
    \underline{C} &= \underline{A} + \underline{B}\\
    &=A_R+jA_I + B_R + jB_I \\
    &= \underbrace{(A_R+B_R)}_{C_R} + j\underbrace{(A_I+B_I)}_{C_I}\\
\end{align*}

\subsubsection*{Prodotto}
\begin{align*}
    \underline{C} &= \underline{A} \cdot \underline{B} & 
    \underline{C} &= \underline{A} \cdot \underline{B}
    \\
    &=A e^{j\alpha}\cdot Be^{j\beta} &
    &=(A_R+jA_I) \cdot (B_R+jB_I)
    \\
    &=\underbrace{A \cdot B}_{C} e^{j\underbrace{(\alpha + \beta)}_{\gamma}} &
    &=(A_RB_R-A_IB_I) + j(A_RB_I+A_IB_R)
\end{align*}

\subsubsection*{Rapporto}
\begin{align*}
    \underline{C} &= \frac{\underline{A}}{\underline{B}} \\
    &= \frac{A e^{j \alpha}}{Be^{j \beta}}\\
    &=\underbrace{\frac{A}{B}}_{C} e^{j\underbrace{(\alpha - \beta)}_{\gamma}}
\end{align*}

\subsubsection*{Complesso coniugato}
Se $\underline{A}=A_R+jA_I$ allora il suo complesso coniugato è
\[
    \underline{A}^* = A_R - jA_I = Ae^{-j \alpha}
\]
\begin{center}
    \includegraphics[scale=0.3]{Image/Complex_conjugate_picture.svg.png}
\end{center}


\subsection{Metodo simbolico}
Utilizziamo Steinmetz sulle equazioni differenziali per ottenere delle equazioni algebriche:
\begin{align*}
    \T{LKT:}&\sum_{i=1}^m v_i(t)=0 & &\Longrightarrow & &\sum_{i=1}^m \underline{V_i}=0\\
    \T{LKC:}&\sum_{i=1}^n i(t)=0 & &\Longrightarrow &  &\sum_{i=1}^m \underline{I_i}=0
\end{align*}
Possiamo scrivere ora le equazioni costitutive in \textbf{forma simbolica}:
\begin{itemize}
    \item resistore: 
    \begin{align*}
        v(t) &=R \cdot i(t) & S[v(t)] &= S[R \cdot i(t)] \\
        S[v(t)] &= \underline{V} & &=R\cdot S[i(t)]\\
        & & &=R \cdot \underline{I}\\
    \end{align*}
    \[\underline{V} = R \cdot \underline{I}\]
    \item induttore:
    \begin{align*}
        v(t) &= L \frac{di}{dt} & S[v(t)] &= S\left[L \frac{di}{dt}\right] \\
        S[v(t)] &= \underline{V} & &=L \cdot S\left[\frac{di}{dt}\right]\\
        & & &=Lj \omega \cdot  S \left[i(t)\right]
    \end{align*}
    \[\underline{V}=j \omega L \underline{I}\]
    \item condensatore:
    \begin{align*}
        i(t) &= C \frac{dv}{dt} &
        \underbrace{S[i(t)]}_{\underline{I}} &= S\left[C \frac{dv}{dt}\right] & \underline{I} = j\omega C \underline{V} \Longrightarrow \underline{V} &= \frac{1}{j \omega C}\underline{I} \\
        & & &=C \cdot S\left[\frac{dv}{dt}\right] & &= - \frac{j}{\omega C}\underline{I}\\
        & & &= C \cdot S\left[j \omega \cdot v(t)\right]\\
        & & &= j \omega C \underline{V}
    \end{align*}
    \[\underline{V} = - \frac{j}{\omega C}\underline{I}\]
\end{itemize}
Le equazioni simboliche di tensione e corrente sono:
\[
    e(t) = \sqrt{2}E \cos(\omega t + \alpha) \longrightarrow S[e(t)] = \underline{E}
\]
\[
    i(t) = \sqrt{2} I \cos(\omega t + \beta) \longrightarrow S[i(t)] = \underline{I}
\]


\subsubsection{Impedenza e ammettenza - Legge di Ohm simbolica}
Definiamo l'\textit{impedenza} $Z$
\[
    \underline{Z} = \frac{\underline{V}}{\underline{I}}
\]
Ad esempio se $\underline{V}=Ve^{j \alpha}$ e $\underline{I} = Ie^{j \beta}$:
\[
    \underline{Z} = \frac{Ve^{j \alpha}}{Ie^{j \beta}} = \underbrace{\frac{V}{I}}_{Z}e^{j\underbrace{(\alpha - \beta)}_{\phi}}
\]
$\phi$ è detto \textit{sfasamento}, cioè la differenza di fase tra tensione e corrente introdotta dal componente.
\begin{itemize}
    \item $\phi >0 \rightarrow \alpha_V > \alpha_I$, la tensione è in anticipo rispetto alla corrente
    \item $\phi <0 \rightarrow \alpha_V < \alpha_I$, la tensione è in ritardo rispetto alla corrente
\end{itemize}


Definiamo l'\textit{ammettenza} $Y$
\[
    \underline{Y} = \frac{1}{\underline{Z}}
\]
Definiamo la \textit{legge di Ohm simbolica}
\[
    \underline{V} = Z\underline{I}
\]

\subsubsection*{Impedenze in serie}
\begin{center}
    \includegraphics[scale=0.3]{Image/Impedenza_serie.png}
\end{center}
\begin{align*}
    \underline{V} &= \underline{V_1}+\underline{V_2}\\
    &= \underline{Z}_1 \underline{I} + \underline{Z}_2 \underline{I}\\
    &= \left(\underline{Z}_1+ \underline{Z}_2\right)\underline{I}
\end{align*}
\begin{align*}
    \underline{Z}_1 &= Z_{R_1} + j Z_{I_1} & Z_{eq} &= \left(Z_{R_1}+Z_{R_2}\right) + j \left(Z_{I_1}+Z_{I_2}\right)\\
    \underline{Z}_2 &= Z_{R_2} + j Z_{I_2}
\end{align*}

\subsubsection*{Impedenze in parallelo}
\begin{center}
    \includegraphics[scale=0.3]{Image/Impedenza-parallelo.png}
\end{center}
\begin{align*}
    \underline{I} &= \underline{I}_1 + \underline{I}_2\\
    &= \frac{\underline{V}}{\underline{Z}_1} + \frac{\underline{V}}{\underline{Z}_2} \\
    &= \underbrace{\left(\frac{1}{\underline{Z}_1} + \frac{1}{\underline{Z}_2}\right)}_{\frac{1}{\underline{Z}_{eq}} = Y_{eq} } \underline{V}
\end{align*}

\subsubsection{Formule simboliche di resistore, induttore e condensatore}
\begin{center}
\begin{tabular}{c c c}
    \includegraphics[scale=0.25]{Image/Fasori_1.png} & \includegraphics[scale=0.25]{Image/Fasori_2.png} & \includegraphics[scale=0.25]{Image/Fasori_3.png}\\
    $\underline{V}=R \underline{I}$ &
    $\underline{V}= j \omega L \underline{I}$ &
    $\underline{V}=- \dfrac{j}{\omega C}\underline{I}$
    \\
    $\underline{Z}=R$ &
    $\underline{Z}=j \omega L$ &
    $\underline{Z}=\dfrac{1}{j \omega C}=-\dfrac{j}{\omega C}$
    \\
    $\underline{I}=\dfrac{\underline{V}}{\underline{Z}} = \dfrac{\underline{V}}{R}$ &
    $\underline{I}=\dfrac{\underline{V}}{\underline{Z}} = \dfrac{\underline{V}}{j \omega L}$ &
    $\underline{I}=\dfrac{\underline{V}}{\underline{Z}} = j \omega C \underline{V}$
    \\
    $I = \dfrac{V}{R}$ &
    $I = \dfrac{V}{\omega L}$ &
    $I = \omega C V$
    \\
    $\alpha_V=0, \ \phi = 0 \rightarrow \alpha_I=0$ &
    $\alpha_V=0, \ \alpha_I=- \dfrac{\pi}{2}$ &
    $\alpha_V=0, \ \alpha_I= \dfrac{\pi}{2}$
    \\
    \includegraphics[scale=0.25]{Image/Fasori_4.png}&
    \includegraphics[scale=0.25]{Image/Fasori_5.png}&
    \includegraphics[scale=0.25]{Image/Fasori_6.png}
\end{tabular}
\end{center}



\subsubsection{Equazioni costitutive in forma simbolica}
\subsubsection*{Resistore}
\[
    v(t)=R \cdot i(t) \overset{S[]}{\longrightarrow} \underline{V} = R \underline{I} \longrightarrow \underline{Z} = \frac{\underline{V}}{\underline{I}} = R
\]
\[
    \begin{cases}
        Z_R=R\\
        Z_I=0
    \end{cases}
    \ \ \ \T{è solamente reale}
\]
\[
    \begin{cases}
        |Z|=R\\
        \phi = 0
    \end{cases}
\]
Supponiamo $\alpha_V=0$
\begin{align*}
    \phi &= \alpha_V - \alpha_I \\
    &\Longrightarrow \alpha_I = - \phi = 0
\end{align*}
\begin{center}
    \includegraphics[scale=0.22]{Image/Regime_sinusoidale_6.png}
\end{center}

\subsubsection*{Induttore}
\[
    v(t)=L \frac{di}{dt} \overset{S[]}{\longrightarrow} \underline{V} = j \omega L \underline{I} \longrightarrow \underline{Z} = \frac{\underline{V}}{\underline{I}} = j \omega L = \omega L e^{j \frac{\pi}{2}}
\]
\[
    \begin{cases}
        Z_R=0\\
        Z_I=\omega L
    \end{cases}
    \ \ \ \T{è solamente immaginiaria}
\]
\[
    \begin{cases}
        |Z|=\omega L\\
        \phi = \dfrac{\pi}{2}
    \end{cases}
\]
Supponiamo $\alpha_V=0$
\begin{align*}
    \phi &= \alpha_V - \alpha_I \\
    &\Longrightarrow \alpha_I = - \phi = -\frac{\pi}{2}
\end{align*}
\begin{center}
    \includegraphics[scale=0.22]{Image/Regime_sinusoidale_7.png}
    \includegraphics[scale=0.22]{Image/Regime_sinusoidale_8.png}
\end{center}

\subsubsection*{Condensatore}
\begin{align*}
    i(t)=C \frac{dv}{dt} \overset{S[]}{\longrightarrow} \underline{I} = j \omega C \underline{V} \longrightarrow \underline{Z} = \frac{\underline{V}}{\underline{I}} &= \frac{1}{j \omega C}\\
    &= -\frac{j}{\omega C} \\
    &=\frac{1}{\omega C} \underbrace{e^{-j \pi}}_{\T{per il meno}} \overbrace{e^{j \frac{\pi}{2}}}^{\T{per la }j} \\
    &= \frac{1}{\omega C}e^{-j \frac{\pi}{2}}
\end{align*}
\textbf{N.B.} lo stesso risultato si ottiene se si utilizzano le formule del passaggio da coordinate cartesiane a coordinate polari.
\[
    \begin{cases}
        Z_R=0\\
        Z_I= - \dfrac{1}{\omega C}
    \end{cases}
    \ \ \ \T{solo parte immaginaria negativa}
\]
\[
    \begin{cases}
        |Z|=\omega L\\
        \phi = \dfrac{\pi}{2}
    \end{cases}
\]
Supponiamo $\alpha_V=0$
\begin{align*}
    \phi &= \alpha_V - \alpha_I \\
    &\Longrightarrow \alpha_I = - \phi = \frac{\pi}{2}
\end{align*}
\begin{center}
    \includegraphics[scale=0.22]{Image/Regime_sinusoidale_9.png}
\end{center}
\begin{align*}
    \underline{Z} &= Z_R + jZ_I\\
    &= R +j \underbrace{X}_{\T{Reattanza}}
\end{align*}
La \textit{reattanza può essere di due tipi}
\begin{itemize}
    \item \textit{reattanza induttiva}: $X_L=\omega L$\\
    \item \textit{reattanza capacitiva}: $X_C= - \dfrac{1}{\omega C}$
\end{itemize}
$X = f(\omega)$ è la \textbf{risposta in frequenza}.


\subsubsection{Il metodo}
\begin{center}
\begin{tabular}{c c c}
    \fbox { \parbox {6cm} {Circuito in regime AC (a corrente alternata). Scriviamo le equazioni differenziali date da LKT,LKC ed equazioni costitutive}}
    &
    $\scaleto{\overset{S[ \ ]}{\longrightarrow}}{17pt}$
    &
    \fbox { \parbox {6cm} {Equazioni algebriche date da LKT, LKC ed equazioni costitutive in forma simbolica}}\\
    $\scaleto{\big\downarrow\substack{\T{metodo} \\ \T{diretto}}}{17pt}$  & &
    $\scaleto{\big\downarrow\substack{\T{metodo} \\ \T{simbolico}}}{17pt}$ 
    \\
    \fbox { \parbox {6cm} {Troviamo tensioni e correnti nel dominio dei tempi}} 
    &
    $\scaleto{\overset{S^{-1}[ \ ]}{\longleftarrow}}{17pt}$
    &
    \fbox { \parbox {6cm} {Troviamo tensioni e correnti in forma simbolica}} 
    \\
    Dominio dei tempi & & Dominio fasoriale
\end{tabular}
\end{center}



\subsection{Potenza elettrica nel regime sinusoidale}

\subsubsection{Potenza istantanea}
La potenza istantanea è data da 
\[
    p(t) = v(t)i(t)    
\]
Assumendo $\alpha_V$ = 0, da cui $\alpha_I=- \phi$ si ha:
\begin{align*}
    v(t) &= V \cos(\omega t) & 
    i(t) &= I \cos(\omega t - \phi)
    \\
    & &
    &=I (\cos(\omega t)\cos(\phi) + \sin(\omega t)\sin(\phi))
    \\
    & &
    &= I\cos(\omega t)\cos(\phi) + I\sin(\omega t)
    \sin(\phi)
    \\
    & &
    &= i_a(t) + i_r(t)
\end{align*}
dove 
\begin{itemize}
    \item $i_a(t)= I\cos(\omega t)\cos(\phi)$ è la \textbf{corrente attiva}, la componente in fase con la tensione
    \item $i_r(t)=I\sin(\omega t) \sin(\phi)$ è la \textbf{corrente reattiva}, componente in quadratura rispetto alla tensione
\end{itemize}
\begin{align*}
    p(t) &= v(t)i(t)
    \\
    &=v(t)i_a(t) + v(t)i_r(t)
    \\
    &=p_a(t) + p_r(t)
\end{align*}
dove
\begin{itemize}
    \item $p_a(t)= v(t)i_a(t)$ è la \textbf{potenza attiva istantanea}
    \item $p_r(t)= v(t)i_r(t)$ è la \textbf{potenza reattiva istantanea}
\end{itemize}
\vspace*{0.2cm}
La \textbf{potenza istantanea attiva} $p_a(t)$ è dovuta
alla componente attiva della corrente che è in
fase con la tensione. La corrente attiva è
quindi dovuta alla componente resistiva
dell'elemento circuitale. La potenza
istantanea attiva è sempre positiva. Quindi è
una potenza che fluisce nell'elemento
circuitale ed è dissipato dalla sua componente
resistiva. Essa quindi viene utilizzata
dall'elemento circuitale ed, in particolare ,dalla
sua resistenza che costituisce un carico del
circuito.
\begin{align*}
    p_a(t) &= v(t)i_a(t)
    \\
    &=V \cos(\omega t)I \cos(\omega t)\cos(\phi)
    \\
    &= VI\cos(\phi)\cos^2(\omega t)
\end{align*}
La \textbf{potenza istantanea reattiva} $p_r(t)$ è dovuta
alla componente reattiva della corrente che è in
quadratura con la tensione. La corrente reattiva
è quindi dovuta alla reattanza dell'elemento circuitale. La potenza istantanea reattiva media è
nulla. Quindi è una potenza che entra nell'elemento circuitale per un quarto di periodo $(T/4 =\pi/(2w))$ ed esce per il quarto di periodo successivo. È quindi una componente della potenza
che non viene dissipata dagli elementi circuitali
ma viene immagazzinata e poi rilasciata dagli
elementi con memoria del circuito.
\begin{align*}
    p_r(t) &= v(t)i_r(t)
    \\
    &= V \cos(\omega t)I \sin(\omega t)\sin(\phi)
    \\
    &= \frac{1}{2}VI\sin(\phi)\sin(2\omega t)
\end{align*}

\subsubsection{Potenza attiva}
La \textbf{potenza attiva P} è definita come la media della potenza $p(t)$ sul periodo $T$
\[P = \frac{1}{T} \int_{T_0}^{t_0+T}p(t') \ dt' = V_eI_e\cos(\phi)\]
dove $V_e$ è la tensione efficace e $I_e$ è la corrente efficace; $\cos(\phi)$ è il \textbf{fattore di potenza}.

\subsubsection{Potenza reattiva}
La \textbf{potenza reattiva} $Q$ è definita come
\begin{align*}
    Q &= \left[p_r(t)\right]_{max} \cdot \T{sgn}(\phi)
    \\
    &=\frac{1}{2}VI\sin(\phi)
    \\
    &=V_eI_e \sin(\phi)
\end{align*}
dove $\phi$ è l'angolo di sfasamento fra tensione e corrente. $Q$ quindi è il valore massimo della potenza scambiata dagli elementi circuitali con memoria nel circuito.\\
$Q$ può essere positivo o negativo a seconda del segno di j. Per un carico induttivo $Q$ è positivo, per un carico capacitivo $Q$ è negativo.\\
Nel Sistema SI delle Unità $Q$ è misurato in volt-ampere reattivi [VAR].

\subsubsection{Potenza complessa}
La potenza complessa $\underline{N}$ è definita come:
\[\underline{N} = \frac{1}{2} \underline{V} \underline{I}^* = \underline{V_e} \underline{I_e}^*\]
dove $\dot I^*$ è il complesso coniugato del fasore $\dot I$
\begin{align*}
    \underline{I} =& I e^{j \alpha}
    \\
    &\Longrightarrow \underline{I}^* = I e^{-j \alpha}
\end{align*}
\begin{align*}
    \underline{N} &= \underline{V_e} \underline{I_e}^*\\
    &=V_ee^{j\alpha_v}I_e e^{-j \alpha_i} \\
    &=V_eI_e e^{j(\alpha_v-\alpha_i)}\\
    &=V_eI_e e^{j\phi}\\
    &= V_eI_e \cos(\phi) + jV_eI_e\sin(\phi)\\
    &= P+jQ
\end{align*}
inoltre poiché $\underline{V_e}=\underline{Z} \underline{I_e}$
\begin{align*}
    P+jQ=\underline{N} &= \underline{V_e} \underline{I_e}^*\\
    &=\underline{Z} \underline{I_e} \underline{I_e}^*\\
    &= \underline{Z} I_e^2\\
    &=(R+jX)I_e^2\\
    &=\underbrace{RI_e^2}_{P}+j\underbrace{XI_e^2}_{Q}
\end{align*}
\begin{align*}
    P &= RI_e^2 & Q &= XI_e^2
\end{align*}

\subsubsection{Formule notevoli}
\begin{align*}
    \underline{N} &= \frac{1}{2} \underline{V} \underline{I}^*
    &
    P &= \frac{1}{2}VI \cos(\phi)
    & 
    Q &= \frac{1}{2} VI \sin(\phi)
    \\
    &= P+jQ
    &
    &= V_eI_e \cos(\phi)
    &
    &=V_eI_e \sin(\phi)
    \\
    &=RI_e^2 + j XI_e^2
    &
    &= \frac{1}{2}RI^2
    &
    &= \frac{1}{2}XI^2
    \\
    & &
    &= RI_e^2
    &
    &= XI_e^2
    \\
    & &
    &=\frac{V^2}{R} &
    &=Im \left(\frac{V^2}{Z^*} \right)
\end{align*}

\subsubsection{Potenza apparente}
La potenza apparente $N$ è definita come:
\[N = \frac{1}{2}VI=V_eI_e\]
Nel sistema di unità SI la potenza apparente è misurata in volt-ampere [VA].\\
Dalla definizione di P e di N risulta:
\begin{align*}
    P &= N \cos(\phi) & N &= \frac{P}{\cos(\phi)}
\end{align*}
Poiché $P$ è la componente della potenza utilizzata ed N riporta i volt-ampere
che devono essere forniti alla rete dal generatore, le espressioni prodotte
mettono in risalto l'importanza del fattore di potenza $\cos(\phi)$ che deve essere il
più vicino possibile all'unità e l'angolo sfasamento $j$ il più ridotto possibile.


\subsubsection{Additività delle potenze}
Per un circuito, indipendentemente dal fatto che le impedenze siano collegate in serie o in parallelo, o siano presenti entrambi i collegamenti, la potenza totale fornita dal/dai generatore/i è uguale alla somma delle potenze utilizzate dai singoli $r$ elementi circuitali passivi
\[\underline{N} = \sum_{k=1}^r \underline{N_k}\]

\subsubsection{Fattore di potenza}
Il fattore di potenza $\cos(\phi)$, con $\phi = \alpha_v-\alpha_i$, angolo di sfasamento fra
tensione e corrente, determina il rapporto fra potenza attiva, componente della potenza utilizzata, e potenza apparente, definita dal prodotto dei valori efficaci di corrente e tensione che devono essere forniti dai generatori.\\
È necessario che il fattore di potenza sia il più possibile vicino all'unità e perciò che l'angolo di sfasamento sia il più piccolo possibile (molte compagnie elettriche fanno pagare per il $\cos(\phi) < 0,9$ corrispondente a $\phi > 25,8^o$)
\begin{align*}
    P &= V_eI_e \cos(\phi)=N \cos(\phi) & &\Longrightarrow & \cos(\phi) &= \frac{P}{V_eI_e} = \frac{P}{N}
\end{align*}
\[I_e = \frac{P}{V_e \cos(\phi)}\]
fissati $P$ e $V_e$, se il $\cos(\phi)$ diminuisce $I_e$ e $N$ aumentano.
\subsubsection*{Correzione dell'angolo di sfasamento}
La maggior parte dei carichi sono induttivi e funzionano con un angolo di sfasamento $\phi$ positivo (corrente in ritardo sulla tensione).
\begin{center}
    \includegraphics[scale=0.26]{Image/Fattore_di_potenza_1.png}
\end{center}
\[
    \begin{rcases}
        P = V_eI_e \cos(\phi)\\
        Q = V_eI_e \sin(\phi)
    \end{rcases}
    \Longrightarrow Q = P \tan(\phi)
\]
Per ridurre $\phi$ e portarlo a $\phi '$ ($\cos(\phi ')=0,9$) si pone un condensatore in parallelo con il circuito dell'utilizzatore. Il condensatore richiede una potenza reattiva negativa che compensa la potenza reattiva positiva dovuta alla reattanza di tipo induttivo dell'utilizzatore.







\subsection{Esercizi}
\subsubsection{Esercizio 1}
\begin{center}
    \includegraphics[scale=0.26]{Image/Es_Regime_Sinusoiudale_1.png}
\end{center}
\begin{align*}
    e(t) &= \sqrt{2} \cdot 230 \cdot \sin(\omega t + 30^o)\\
    f &= 50 Hz\\
    R_! &= 0,1 \Omega & R_2&=5,2 \Omega\\
    L_1 &= 0,64 mH & L_2 &= 12 mH\\
    C_1 &= 180 \mu F
\end{align*}

Bisogna trovare:
\begin{itemize}
    \item le correnti in ogni ramo
    \item $v_C(t)$
\end{itemize}

\begin{enumerate}[label=\protect\circled{\arabic*}]
    \item Passiamo da dominio nei tempi a dominio fasoriale
    \begin{center}
        \includegraphics[scale=0.26]{Image/Es_Regime_Sinusoiudale_1_1.png}
    \end{center}
    Ricordando che $S[a(t)]=A_e \cdot e^{j \alpha}$ e che $e(t) = \sqrt{2} E \cos(\omega t + \alpha)$
    \[\underline{E}=230 \cdot e^{j 30}\]
    e $\omega = 2 \pi f = 314,16 \ rad/s$.
    \vspace*{0.1cm}\\
    Calcoliamo le impedenze:
    \begin{align*}
        \underline{Z}_{R_1} &= R_1 & 
        \underline{Z}_{L_1} &= j X_{L_1}&
        \underline{Z}_C &= j X_C\\
        &=0,1 \Omega & &=j \omega L_1 & &=-\frac{j}{\omega C}
        \\
        & & 
        &= j \cdot 314,16 \cdot 0,64 \times 10^{-3} &
        &= - j 17,66 \ \Omega
        \\
        & & 
        &= j 0,2 \ \Omega
    \end{align*}
    \begin{align*}
        \underline{Z}_{R_2} &= R_2 &
        \underline{Z}_{L_2} &= jX_{L_2}
        \\
        &= 5,2 \ \Omega & &= j \omega L_2
        \\
        & &
        &=j \cdot 314,16 \cdot 1,2 \times 10^{-3}
        \\
        & &
        &= j \ 3,77 \ \Omega
    \end{align*}
    \begin{center}
        \includegraphics[scale=0.26]{Image/Es_Regime_Sinusoiudale_1_2.png}
    \end{center}
    \begin{align*}
        \underline{Z}_1 &= \underline{Z}_{R_1} + \underline{Z}_{L_1} &
        \underline{Z}_2 &= \underline{Z}_{R_2} + \underline{Z}_{L_2} &
        \underline{Z}_3 &= \underline{Z}_C
        \\
        &=0,1 +j \ 0,2 \ [\Omega] &
        &= 5,2 + j \ 3,77 \ [\Omega] &
        &= - j\ 17,66 \ [\Omega]
    \end{align*}
    \item Risolviamo il circuito (nel dominio fasoriale)
    \vspace*{0.2cm}\\
    Applichiamo Millman:
    \begin{align*}
        \underline{V}_C &= \frac{\dfrac{\underline{E}}{\underline{Z}_1}}{\dfrac{1}{\underline{Z}_1} + \dfrac{1}{\underline{Z}_2} + \dfrac{1}{\underline{Z}_3}} \\
        &=197,68 + j \ 108,58 \ [V]\\
        \substack{\T{passaggio da coo.} \\ \T{cartesiane a polari}} \Longrightarrow&=\underbrace{225,53}_{E} \ \underbrace{\boxed{28,78 ^o}}_{\alpha} \ [V]
    \end{align*}
    \begin{align*}
        \underline{I}_C &= \frac{\underline{V}_C}{\underline{Z}_3} &
        \underline{I}_L &= \frac{\underline{V}_C}{\underline{Z}_2}
        \\
        &= 12,76 \ \boxed{118,78^o} \ [A] &
        &= 35,11 \ \boxed{-7,16^o} \ [A]
    \end{align*}
    \begin{align*}
        \T{LKC:} \ \underline{I}_e &= \underline{I}_C + \underline{I}_L\\
        &=29,49 \ \boxed{13,34^o}
    \end{align*}
    \item Passaggio da dominio fasoriale a dominio nei tempi
    \vspace*{0.2cm}\\
    Applico \textit{l'antitrasformata} $S^{-1}[ \ ]$ ai fasori trovati
    \begin{align*}
        \underline{V}_C &= \underbrace{225,53}_{E} \ \underbrace{\boxed{28,78 ^o}}_{\alpha} &
        &\underset{S^{-1} [\ ]}{\Longrightarrow} &
        v_C(t) &=\sqrt{2} \cdot 225,53 \cdot \sin(\omega t + 28,78 ^o)
        \\
        \underline{I}_C &= \underbrace{12,76}_{I_C} \ \underbrace{\boxed{118,78 ^o}}_{\alpha} &
        &\underset{S^{-1} [\ ]}{\Longrightarrow} &
        i_C(t) &= \sqrt{2} \cdot 12,76 \cdot \sin(\omega t + 118,78 ^o)\\
        \underline{I}_L &= 35,11 \ \boxed{-7,16^o} &
        &\underset{S^{-1} [\ ]}{\Longrightarrow} &
        i_L(t) &= \sqrt{2} \cdot 35,11 \cdot \sin(\omega t + -7,16 ^o)
        \\
        \underline{I}_e &= 29,49 \ \boxed{13,34^o} &
        &\underset{S^{-1} [\ ]}{\Longrightarrow} &
        i_e(t) &= \sqrt{2} \cdot 28,49 \cdot \sin(\omega t + 13,34 ^o)
    \end{align*}
\end{enumerate}


\subsubsection{Esercizio 2}
\begin{center}
    \includegraphics[scale=0.27]{Image/Es_Regime_Sinusoiudale_2.png}
\end{center}
\begin{align*}
    e(t) &= \sqrt{2} \cdot 10 \cdot \cos \left(\omega t + \frac{\pi}{2}\right)\\
    R_1 &= 2 \ \Omega & R_2 &= 1 \ \Omega \\
    L_1 &= 1 \ mH & C_1 &= 500 \ \mu F\\
    \omega &= 10^3 \ rad/s 
\end{align*}
Bisogna trovare:
\begin{itemize}
    \item $i_1(t)$
    \item $i_2(t)$
    \item $i_3(t)$
\end{itemize}
\begin{enumerate}[label=\protect\circled{\arabic*}]
    \item Passiamo da dominio nei tempi a dominio fasoriale
    \begin{center}
        \includegraphics[scale=0.26]{Image/Es_Regime_Sinusoiudale_2_1.png}
    \end{center}
    Ricordando che $S[a(t)]=A_e \cdot e^{j \alpha}$ e che $e(t) = \sqrt{2} E \cos(\omega t + \alpha)$
    \[\underline{E}=10 \cdot e^{j \frac{\pi}{2}}\]
    Calcoliamo le impedenze:
    \begin{align*}
        \underline{Z}_{R_1} &= R_1 & 
        \underline{Z}_{L} &= j X_{L_1}&
        \underline{Z}_C &= j X_{C_1} &
        \underline{Z}_{R_2} &= R_2
        \\
        &=2 \Omega & 
        &=j \omega L_1 & 
        &=-\frac{j}{\omega C} &
        &= 1 \ \Omega
        \\
        & & 
        &= j \cdot 10^3 \cdot 10^{-3} &
        &= - j \frac{1}{10^3 \cdot 500 \times 10^{-6}} 
        \\
        & & 
        &= j &
        &= -j \cdot 2
    \end{align*}
    \begin{center}
        \includegraphics[width=0.23\textwidth]{Image/Es_Regime_Sinusoiudale_2_2.png}
    \end{center}
    \leavevmode
    \begin{align*}
        \underline{Z}_{1} &= \underline{Z}_R + \underline{Z}_L &
        \underline{Z}_{2} &= \underline{Z}_{R_2} // \underline{Z}_C
        \\
        &= 2 + j &
        &= \frac{\underline{Z}_{R_2} \cdot \underline{Z}_C}{\underline{Z}_{R_2} + \underline{Z}_C}
        \\
        & &
        &=\frac{1 \cdot (-2j)}{1 -2j}
        \\
        & &
        &= -\frac{2j}{1-2j} \cdot \frac{1+2j}{1+2j}
        \\
        & &
        &=-2j \cdot \frac{1+2j}{1+4}
        \\
        & &
        &=-\frac{2j-4}{5}
        \\
        & &
        &= \frac{4}{5} - j \frac{2}{5}
        \\
        & & 
        &= 0,8 -0,4j
    \end{align*}
    \begin{align*}
        \underline{I}_1 &= \frac{\underline{E}}{\underline{Z}_1+\underline{Z}_2} &
        \underline{I}_2 &= \underline{I}_1 \cdot \frac{\underline{Z}_C}{\underline{Z}_{R_1}+\underline{Z}_{R_2}} &
        \underline{I}_3&= \underline{I}_1 \cdot \frac{\underline{Z}_{R_2}}{\underline{Z}_{R_2}+\underline{Z}_C}
        \\
        &=\frac{\underline{E}}{2,8+0,6j} &
        &= 1.95 + 2,44j &
        &= -1,22 + 0,92j
        \\
        &=\frac{10j}{2,8+0,6j} &
        &=3,12 \ \boxed{51,34^o} &
        &= 1,56 \ \boxed{141^o}
        \\
        &=0,73 + 3,41j
        \\
        &=3,49 \ \boxed{77,9 ^o}
    \end{align*}
    
    \item Passaggio da dominio fasoriale a dominio nei tempi
    \vspace*{0.2cm}\\
    Applico \textit{l'antitrasformata} $S^{-1}[ \ ]$ ai fasori trovati
    \begin{align*}
        \underline{I}_1 &= \underbrace{3,49}_{I_1} \ \underbrace{\boxed{77,9 ^o}}_{\alpha} &
        &\underset{S^{-1} [\ ]}{\Longrightarrow} &
        i_1(t) &= \sqrt{2} \cdot 3,49 \cdot \sin(\omega t + 77,9 ^o)
        \\
        \underline{I}_2 &= 3,12 \ \boxed{51,34^o} &
        &\underset{S^{-1} [\ ]}{\Longrightarrow} &
        i_2(t) &= \sqrt{2} \cdot 3,12 \cdot \sin(\omega t + 51,34 ^o)
        \\
        \underline{I}_3 &= 1,56 \ \boxed{141^o} &
        &\underset{S^{-1} [\ ]}{\Longrightarrow} &
        i_3(t) &= \sqrt{2} \cdot 1,56 \cdot \sin(\omega t + 141 ^o)
    \end{align*}
\end{enumerate}


\begin{comment}
\subsubsection{Esercizio sulle potenze}
\begin{center}
    \includegraphics[scale=0.3]{Image/Es_Regime_Sinusoidale_Potenze.png}
\end{center}
\begin{align*}
    V &= 400 \ V & L &= 0,3 \ H
    \\
    R_1 &= 50 \ \Omega & R_2 &= 20 \ \Omega 
    \\
    C &= 30 \ \mu F & f &= 50 \ Hz \\
    & & \omega &= 2 \pi f = 314,16 \ rad/s 
\end{align*}
Dobbiamo calcolare la potenza generata da $\underline{E}$
\begin{enumerate}
    \item Calcolo $\underline{I}$
    \[\underline{I} = \frac{\underline{V}}{\underline{Z}_{eq}}\]
    \begin{align*}
        \underline{Z}_{eq} &= \frac{R_1\cdot (jX_C)}{R_1 + (jX_C)} & 
        \underline{I} &= \frac{400}{41-19j}
        \\
        &= \frac{-R_1\cdot \left(j \dfrac{1}{\omega C}\right)}{R_1+j \left(- \dfrac{1}{\omega C}\right)} &
        &=8,03 + 3,72j \ A
        \\
        &= - \frac{50\left(j\dfrac{1}{314,16 \cdot 30 \times 10^{-6}}\right)}{50 - j\dfrac{1}{314,16 \cdot 30 \times 10^{-6}}}
        \\
        & = 41 - 19j \ \Omega
    \end{align*}
    \item Calcolo $\underline{E}$
    \begin{align*}
        \underline{Z}_2 &= R_2 +jX_L &
        \underline{E} &= \underline{V} + \underline{Z}_2 \cdot \underline{I}
        \\
        &= R_2 +j \omega L &
        &= 400 + (20 + 94,2j)\cdot (8,03+3,72j)
        \\
        &= 20 + 94,2j \ \Omega &
        &=210,18 + 830,83j \ V
    \end{align*}
    \item Calcolare $\underline{N}$ (potenza)
    \begin{align*}
        \underline{N} &= \underline{E} \cdot \underline{I}
    \end{align*}
\end{enumerate}
\end{comment}

\subsubsection{Esercizio metodo delle potenze 1}
\begin{center}
    \includegraphics[scale=0.27]{Image/Esercizio_potenze_1.png}
\end{center}
Dati:
\begin{align*}
    R &= 10 \ \Omega & C &= 100 \ \mu F\\
    L &= 0,1 \ H & V &= 230 \ V
\end{align*}
Bisogna trovare:
\begin{itemize}
    \item corrente elettrica
    \item P, Q
\end{itemize}
\subsubsection*{Metodo 1}
Si calcolano le impedenze in $C$ ed $L$:
\begin{align*}
    \underline{Z}_C &= -\frac{j}{\omega C} & \underline{Z}_L &= j \omega L
    \\
    &= -j31,83 \ \Omega &
    &= j31,416 \Omega
    \\
\end{align*}
Quindi l'impedenza totale risulta:
\[\underline{Z}_{eq} = 10 \phase{-0,24^{\circ}}\]
\[\underline{I} = \frac{\underline{V}}{\underline{Z}_{eq}} = 23 \phase{-0,237^{\circ}}\]
Per il calcolo di $P$ e $Q$ si usa la formula:
\begin{align*}
    \dot N &= P+jQ
    \\
    &=\underline{V}\cdot \underline{I}^*
    \\
    &=5290 \phase{0,237^{\circ}}
    \\
    &= 5290 + j21,9
\end{align*}
quindi
\begin{align*}
    P &= 5290 \ W &  Q = 21,8 \ \T{Var}
\end{align*}
\subsubsection*{Metodo 2}
Calcoliamo la potenza assorbita da $R$ nel modo classico:
\begin{align*}
    P &= \frac{V^2}{R}
    \\
    &= \frac{230^2}{10}
    \\
    &= 5,29 \ kW
\end{align*}
Mentre per la potenza reattiva di $C$ ed $L$ si usa la formula $Q = Im \Bigg(\dfrac{V^2}{Z^*}\Bigg)$:
\begin{align*}
    Q_C &= Im \Bigg( \frac{V^2}{Z_C^*}\Bigg) &
    Q_L &= Im \Bigg( \frac{V^2}{Z_L^*}\Bigg)
    \\
    &= -1661,95 \ \T{Var} &
    &= 1683,86 \ \T{Var}
\end{align*}
\[Q_{tot} = 21,9 \ \T{Var}\]



\subsubsection{Esercizio metodo delle potenze 2}
\begin{center}
    \includegraphics[scale=0.27]{Image/Esercizio_potenze_2.png}
\end{center}
Dati:
\begin{align*}
    R &= 1 \Omega & L &= 10\ mH \\
    V &= 320 \ V & Q &= 10 \ kVAr
\end{align*}
Bisogna trovare
\begin{itemize}
    \item $P$ e $Q$ del ramo $RL$
    \item $Q$ del condensatore
    \item $C$
    \item corrente assorbita
\end{itemize}
Calcoliamo le impedenze di $R$ e $L$:
\begin{align*}
    \underline{Z}_L &= j \omega L & 
    \underline{Z}_R &= R &
    \underline{Z}_{RL} &= 1 + j 3,14 \ \Omega
    \\
    &= j3.14 \  \Omega &
    &= 1 \ \Omega
\end{align*}
\begin{align*}
    \dot N_{RL} &= \underline{V} \cdot \underline{I}^*
    \\
    &= \frac{V^2}{\underline{Z}_{RL}^*}
    \\
    &= \underbrace{4,87}_{P} + j\underbrace{15,3}_{Q} \ kVA
\end{align*}
quindi
\begin{align*}
    P &= 4,87 \ kW & Q &= 15,3 \ kVAr
\end{align*}
\[Q_C = Q- Q_{RL} = -5,3 \ kVAr\]
Ora dobbiamo calcolare la capacità del condensatore; sappiamo che 
\begin{align*}
    \underline{Z}_C &= - \frac{j}{\omega C} & Q_C &= Im \left(\frac{V^2}{\underline{Z}_C^*}\right)
\end{align*}
ma l'operatore "immagine" si può omettere nella formula della potenza reattiva perché la frazione al suo interno dá come risultato un numero puramente immaginario, quindi
\begin{align*}
    Q_C &= \frac{V^2}{\underline{Z}_C^*}
    \\
    &= \frac{V^2}{\dfrac{j}{\omega C}}
    \\
    &= \frac{V^2 \omega C}{j}
\end{align*} 
isolando il termine $C$ si arriva all'equazione
\begin{align*}
    C &= \frac{Q_C \cdot j}{V^2 \omega}
    \\
    &= \frac{- \cancel{j} \cdot 5300 \cdot \cancel{j}}{(230)^2 \cdot 314,16}
    \\
    &= 319 \ \mu F
\end{align*}
La formula della potenza complessa totale è:
\[\dot N_{tot} = 4,87 + j10 \ kVA\]
quindi l'angolo del fattore di potenza è uguale a
\[\phi = \arctan \left(\frac{Q_{tot}}{P_{tot}}\right) = 64^{\circ}\]
ora possiamo determinare la corrente assorbita nel circuito:
\begin{align*}
    I &= \frac{P}{V \cos(\phi)}\\
    &= 48,3 \ A
\end{align*}


\section{Risposta in frequenza}
Definiamo la \textbf{funzione di trasferimento} $\dot H(\omega)$
\[\dot H(\omega) = \frac{\dot Y(\omega)}{\dot X(
    \omega
)}\]
In relazione al tipo di ingresso e risposta (se corrente o tensione) si hanno quattro tipi di funzioni di trasferimento:
\begin{itemize}
    \item Guadagno di tensione
     \[\dot \alpha(\omega)= \frac{\dot V_{out}}{\dot V_{in}}=\alpha(\omega) e^{j \phi}\]
     \item Guadagno di corrente 
     \[\dot \beta(\omega)= \frac{\dot I_{out}}{\dot I_{in}}=\beta(\omega) e^{j \phi}\]
     \item Impedenza di trasferimento
     \[\dot z(\omega)= \frac{\dot V_{out}}{\dot I_{in}}=Z(\omega) e^{j \phi}\]
     \item Ammettenza di trasferimento
     \[\dot y(\omega)= \frac{\dot I_{out}}{\dot V_{in}}=Y(\omega) e^{j \phi}\]
\end{itemize}
Dove $\alpha(w), \beta(w), Z(w) \T{ e } Y(w)$ sono le ampiezze delle funzioni di trasferimento e $\phi$ è lo sfasamento fra l'output e l'input.


\subsection{Filtro passa-basso}
\begin{center}
    \includegraphics[scale=0.3]{Image/Passa-basso_1.png}
\end{center}
Nel circuito RC con funzione di \textit{filtro passa-basso}, a un segnali di ingresso $v_S(t)=V \cos(\omega t)$ il segnale di uscita è dato dalla tensione sul condensatore:
\begin{align*}
    \dot V_S = \dot V_{in} &= \left[R - j \frac{1}{\omega C}\right]\dot I & \dot V_{out} &= -j \frac{1}{\omega C} \dot I
\end{align*}
il guadagno di tensione è:
\begin{align*}
    \dot \alpha(\omega) &= \alpha(\omega) e^{j \phi}
    \\
    &=\frac{\dot V_{out}}{\dot V_{in}}
    \\
    &= \frac{-\frac{j}{\omega C}}{R -\frac{j}{\omega C}}
    \\
    &= \frac{1-j \frac{\omega}{\omega_0}}{1 + \left(\frac{\omega}{\omega_0}\right)^2}
\end{align*}
dove $\omega_0=\dfrac{1}{RC}$ è la \textbf{frequenza di taglio}. L'ampiezza e la fase del guadagno sono:
\begin{align*}
    \alpha(\omega) & \frac{1}{\sqrt{1+\left(\frac{\omega}{\omega_0}\right)^2}} & \phi &= \tan^{-1}\left(- \frac{\omega}{\omega_0}\right)
\end{align*}
\begin{center}
    \includegraphics[scale=0.27]{Image/Passa-basso_2.png}
    \includegraphics[scale=0.27]{Image/Passa-basso_3.png}
\end{center}
\begin{center}
    \begin{tabular}{|c|c|c|c|c|c|c|c|}
        \hline
        $\dfrac{\omega}{\omega_0}$
        & 0 & 1 & 2 & 3 & 10 & 20 & 100\\
        \hline
        $\alpha$ & 1 & 0.71 & 0.45 & 0.32 & 0.1 & 0.05 & 0.01\\
        \hline
        $\phi$ & 0 & $-45^o$ & $-63^o$ & $-72^o$ & $-84^o$ & $-87^o$ & $-90^o$\\
        \hline
    \end{tabular}
\end{center}
Poiché questo filtro è di soli elementi passivi, il
valore massimo del guadagno è pari all'unità.

\subsubsection{Filtro passa-basso RL}
Il filtro RL può assumere la funzione di filtro
passa-basso, se $\dot V_{out}$ è la caduta di tensione sul resistore:
\[\dot \alpha(\omega)=\frac{R}{R+j \omega L}\]



\subsection{Filtro passa-alto}
Nel circuito RL con funzione di filtro passa-alto, a un segnale di ingresso $v_S(t)=V \cos(\omega t)$ il sanale di uscita è dato dalla tensione dell'induttore:
\begin{align*}
    \dot V_S = \dot V_{in} &= \left[R + j \omega L\right]\dot I & \dot V_{out} &= j \omega L \dot I
\end{align*}
Il guadagno è:
\begin{align*}
    \dot \alpha(\omega) &= \alpha(\omega) e^{j \phi}
    \\
    &=\frac{\dot V_{out}}{\dot V_{in}}
    \\
    &= \frac{j \omega L}{R+j \omega L}
    \\
    &= \frac{j \frac{\omega}{\omega_0}}{1+j \frac{\omega}{\omega_0}}
\end{align*}
dove $\omega_0 = \dfrac{R}{L}$ è la \textbf{frequenza di taglio}. L'ampiezza e la fase del guadagno del \textit{filtro RL passa-alto}, come visto, sono:
\begin{align*}
    \alpha(\omega) &= \frac{\frac{\omega}{\omega_0}}{\sqrt{1+\left(\frac{\omega}{\omega_0}\right)^2}} & \phi &= \tan^{-1} \frac{\omega_0}{\omega}
\end{align*}
Poiché questo filtro è di soli elementi passivi, il
valore massimo del guadagno è pari all'unità.


\subsubsection{Filtro passa-alto RC}
Il filtro RC può assumere la funzione di filtro
passa-alto, se $\dot V_{out}$ è la caduta di tensione sul resistore:
\[\dot \alpha(\omega) = \frac{\dot V_{out}}{\dot V_{in}} = \frac{R}{R - \dfrac{j}{\omega C}}\]



\subsection{Filtro passa-banda}
\begin{center}
    \includegraphics[scale=0.3]{Image/Passa-banda_1.png}
\end{center}
Nel circuito RLC con funzione di filtro passa-banda, ad un segnale di ingresso $v_S(t) = V \cos (\omega t)$ il segnale di uscita è dato dalla tensione sul resistore:
\begin{align*}
    \dot V_S &= \dot V_{in} = \left[R + j\left(\omega L - \dfrac{1}{\omega C}\right)\right] \dot I & \dot V_{out} = R \dot I
\end{align*}
Il guadagno è:
\[\dot \alpha(\omega)=\alpha(\omega) e^{j \phi}=\frac{\dot V_{out}}{\dot V_{in}}=\frac{R}{R + j\left(\omega L - \dfrac{1}{\omega C}\right)}\]
La frequenza di centro-banda è $\omega_0= \dfrac{1}{\sqrt{LC}}$. Si definiscono anche $\omega_1$ e $\omega_2$ frequenze di taglio limiti di banda. I segnali con frequenze $\omega_1<\omega<\omega_2$ sono lasciati passare, i segnali con $\omega < \omega_1$ e con $\omega > \omega_2$ sono fermati.
\begin{center}
    \includegraphics[scale=0.27]{Image/Passa-banda_2.png}
\end{center}



\subsection{Filtro arresta banda}
\begin{center}
    \includegraphics[scale=0.3]{Image/Arresta-banda_1.png}
\end{center}
Nel circuito RLC con funzione di filtro arresta-banda, ad un segnale di ingresso $v_S(t) = V \cos(\omega t)$ il segnale di uscita è data dalla tensione su induttore e condensatore:
\begin{align*}
    \dot V_S &= \dot V_{in} = \left[R + j\left(\omega L - \dfrac{1}{\omega C}\right)\right] \dot I & \dot V_{out} = j \left(\omega L - \dfrac{1}{\omega C}\right) \dot I
\end{align*}
Il guadagno è:
\[\dot \alpha(\omega)=\alpha(\omega) e^{j \phi}=\frac{\dot V_{out}}{\dot V_{in}}=\frac{j\left(\omega L - \dfrac{1}{\omega C}\right)}{R + j\left(\omega L - \dfrac{1}{\omega C}\right)}\]
La frequenza di centro-banda è $\omega_0= \dfrac{1}{\sqrt{LC}}$. Si definiscono anche $\omega_1$ e $\omega_2$ frequenze di taglio limiti di banda. I segnali con frequenze $\omega_1<\omega<\omega_2$ non sono lasciati passare, i segnali con $\omega < \omega_1$ e con $\omega > \omega_2$ sono lasciati passare.
\begin{center}
    \includegraphics[scale=0.27]{Image/Arresta-banda_2.png}
\end{center}


\subsection{Filtri ideali}
\begin{center}
    \includegraphics[scale=0.3]{Image/Filtri_ideali.png}
\end{center}
I \textbf{filtri passivi} sono costituiti solo da elementi passivi (R, L e C). Il loro guadagno non può essere maggiore di 1 in quanto gli elementi passivi non
possono fornire energia alla rete. Essi per lo più richiedono elementi ingombranti e costosi; si comportano male alle basse frequenze.\\
I \textbf{filtri attivi} contengono elementi attivi come transistori e amplificatori operazionali.






\section{Sistemi trifase}
\subsection{Introduzione}
\begin{center}
    \includegraphics[scale=0.27]{Image/Sistemi_trifase_1.png}
\end{center}
La generazione e la distribuzione di energia è in genere realizzata mediante \textit{sistemi trifase}. Questi sistemi sono costituiti da tre cavi conduttori collegati ad un alternatore (macchina elettrica che trasforma energia meccanica in energia elettrica), che può essere considerato composto da tre generatori indipendenti di tensione in regime alternato, con la stessa ampiezza della tensione, l'una sfasata rispetto all'altra di $120^o$.
\begin{center}
    \includegraphics[scale=0.26]{Image/Sistemi_trifase_2.png}
\end{center}
Frequenza e tensione efficace sono di 50 Hz, 314 rad/s e 230 - 400 V in Europa, buona parte dell'Asia ed in Australia, mentre sono di 60 Hz, 377 rad/s, 120 - 210 V in Nord America.
\vspace*{0.2cm}\\
Tre sono le motivazioni principali per l'utilizzo dei sistemi trifase:
\begin{enumerate}
    \item La maggior parte di energia elettrica è generata da macchine elettriche rotanti a tre fasi (alternatori);
    \item Nei sistemi a tre fasi bilanciati la \textbf{potenza istantanea totale} è costante nel tempo (le tre fasi si bilanciano nel tempo). Ciò si traduce in una trasmissione uniforme con meno vibrazioni lungo la linea
    \item Per la stessa quantità di energia trasportata, un sistema a tre fasi è più
    economico di uno singola fase.
\end{enumerate}



\subsection{Correnti e tensioni nei sistemi trifase}
Le correnti di linea $i_1(t)$, $i_2(t)$ e $i_3(t)$ sono le correnti che fluiscono in ciascuna delle tre linee. Dalla LKC segue che:
\begin{align*}
    i_1(t) + i_2(t)+i_3(t)&=0 & \dot I_1 + \dot I_2 + \dot I_3 &=0
\end{align*}
Le tensioni concatenate $v_{12}(t), v_{23}(t) \T{ e } v_{31}(t)$ sono le tensioni fra linea e linea. Dalla LKT segue:
\begin{align*}
    v_{12}(t) + v_{23}(t) + v_{31}(t) &= 0 & \dot V_{12} + \dot V_{23} + \dot V_{31} &= 0
\end{align*}
Anche per le tensioni di fase $e_1 (t), e_2 (t) \T{ e } e_3(t)$, le tensioni fra il centro stella dei tre generatori e ciascuna linea, si ha:
\begin{align*}
    e_1 (t) + e_2 (t) + e_3(t) &= 0 & \dot E_1 + \dot E_2 + \dot E_3 &= 0
\end{align*}
Da questo se ne deduce che i fasori delle tensioni e delle correnti soddisfano le leggi di Kirchhoff. Nel piano complesso perciò sono rappresentati da
triangoli: il triangolo delle tensioni concatenate, il triangolo delle tensioni di linea ed il triangolo
delle correnti di linea. 


\subsubsection{Sistema delle tensioni simmetrico}
\begin{center}
    \includegraphics[scale=0.3]{Image/Sistema_delle_tensioni_simmetrico.png}
    \includegraphics[scale=0.3]{Image/Sistemi_trifase_2.png}
\end{center}
Le tensioni di fase $\dot E_1, \dot E_2 \T{ ed } \dot E_3$ sono collegate a stella tra un centro stella $n$ e i terminali 1,2 e 3. Le tensioni concatenate $\dot V_{12}, \dot V_{23} \T{ e } \dot V_{31}$ sono la differenza di potenziale trai terminali 1 e 2, 2 e 3 e 3 e 1. Le relazioni tra esse, date dalla LKT, sono:
\begin{align*}
    \dot V_{12} &= \dot E_1 - \dot E_2\\
    \dot V_{23} &= \dot E_2 - \dot E_3\\
    \dot V_{31} &= \dot E3 - \dot E_1
\end{align*} 
Il nodo $n$ può essere un qualsiasi punto del piano
complesso per il quale viene individuata una terna
di tensioni di fase che soddisfa le tre relazioni
precedenti per le stesse tre tensioni di linea. Pertanto ci sono $\infty ^2$ serie di tensioni di fase corrispondenti allo stessa terna di tensioni di linea.
\begin{center}
    \includegraphics[scale=0.3]{Image/Triangolo_delle_tensioni_concate_e_tensioni_di_fase.png}
\end{center}


\subsubsection{Sistema equilibrato delle correnti}
\begin{center}
    \includegraphics[scale=0.3]{Image/Sistema_equilibrato_delle_correnti.png}
\end{center}


\subsubsection{Sistema delle tensioni di linea e di fase simmetrico}
Un sistema di tensioni di linea simmetrico con
sequenza positiva ed il corrispondente sistema
simmetrico di tensioni di fase, anch'esso con
sequenza positiva sono:
\begin{align*}
    \dot V_{12} &= V
    &
    \dot V_{23} &= \dot \alpha^2 \dot V_{12}
    &
    \dot V_{31} &= \dot \alpha \dot V_{12}
    \\
    & &
    &=V e^{-j 2 \pi /3}
    &
    &= V e^{j 2 \pi /3}
    \\
    & &
    &= - \frac{V}{2} \left(1+j \sqrt{3}\right)
    &
    &= \frac{V}{2} \left(1+j \sqrt{3}\right)
\end{align*}
\begin{align*}
    \dot E_1 &= E e^{-j \pi /6}
    &
    \dot E_2 &= E e^{-j \pi /6} e^{-j 2\pi /3}
    &
    \dot E_3 &= E e^{-j 4\pi /6} e^{-j 4\pi /3}
    \\
    & & 
    &=E e^{-j 5\pi /6}
    &
    &= E e^{-j 9\pi /6}
\end{align*}
\begin{center}
    \includegraphics[scale=0.27]{Image/Sistema_tensioni_linea_fase_simmetrico.png}
\end{center}



\subsection{Impedenze di carico}
\subsubsection{Impedenze collegate a stella}
\begin{center}
    \includegraphics[scale=0.27]{Image/Impedenze_collegate_a_stella_1.png}
\end{center}
Ci sono $\infty^2$ serie di impedenze collegate a stella che inducono $\infty^2$ serie di tensioni di fase che possono essere collegate alla stessa serie di tensioni di linea.
\begin{align*}
    \dot V_{12} &= \dot Z_1 \dot I_1 - \dot Z_2 \dot I_2 & & 
    &
    \dot E_1 &= \dot Z_1 \dot I_1
    \\
    \dot V_{23} &= \dot Z_2 \dot I_2 - \dot Z_3 \dot I_3 & 
    &\Longrightarrow 
    &
    \dot E_2 &= \dot Z_2 \dot I_2
    \\
    \dot I_1 &= \dot I_2 + \dot I_3 = 0
    & &
    &
    \dot E_3 &= \dot Z_3 \dot I_3
\end{align*}
\begin{center}
    \includegraphics[scale=0.24]{Image/Impedenze_collegate_a_stella_2.png}
\end{center}


\subsubsection{Impedenze collegate a triangolo}
\begin{center}
    \includegraphics[scale=0.27]{Image/Impedenze_collegate_a_triangolo_1.png}
\end{center}
\begin{align*}
    \dot V_{12} &= \dot Z_{12} \dot I_{12} & & 
    &
    \dot I_1 &= \dot I_{12} - \dot I_{31}
    \\
    \dot V_{23} &= \dot Z_{23} \dot I_{23}
    & 
    &\Longrightarrow 
    &
    \dot I_2 &= \dot I_{23} - \dot I_{12}
    \\
    \dot  V_{31} &= \dot Z_{31} \dot I_{31}
    & &
    &
    \dot I_3 &= \dot I_{31} - \dot I_{23}
\end{align*}


\subsubsection{Impedenze collegate a una rete trifase}
\begin{center}
    \begin{tabular}{c c}
        \textbf{Impedenze a stella connesse a una rete trifase} & \textbf{Impedenze a triangolo connesse a una rete trifase}
        \\
        \includegraphics[scale=0.26]{Image/Impedenze_connesse_1.png}
        &
        \includegraphics[scale=0.26]{Image/Impedenze_connesse_2.png}
    \end{tabular}
\end{center}



\subsection{Sistemi trifase bilanciati}

Qualora si abbia un sistema di impedenze $\dot Z_1, \dot Z_2, \T{ e } \dot Z_3$ a stella o a
triangolo uguali (cioè con uguale resistenza e uguale reattanza), il sistema è un \textbf{sistema di impedenze di carico bilanciato}.\\
Qualora un sistema trifase simmetrico (con tensioni simmetriche) sia collegato ad un sistema di impedenze di carico bilanciato, il sistema
delle correnti è equilibrato. Il sistema trifase con tensioni simmetriche e correnti equilibrate si definisce \textbf{sistema trifase bilanciato}.


\subsection{Sistemi trifase con neutro}
Un generatore trifase di un sistema
simmetrico (un alternatore) equivale a tre
generatori monofase collegati a stella che
generano un sistema simmetrico di tensioni di
fase con centro O. In un \textit{sistema trifase con
neutro} il centro O della connessione a stella è
collegato a una quarta linea. La relazione tra
le tensioni di fase $E$, simmetriche, e le tensioni
di linea $V$, anch'esse simmetriche, è:
\[E = \frac{V}{\sqrt{3}}\]
\begin{center}
    \includegraphics[scale=0.27]{Image/Sistemi_trifase_neutro_1.png}
\end{center}
Nel sistema trifase con neutro il triangolo
delle correnti è sostituito da un quadrilatero
delle correnti dato dall'equazione (LKC):
\[\dot I_1 + \dot I_2 + \dot I_3 + \dot I_n =0\]
La tensione di fase efficace $E_e = 230 V$ si ottiene per mezzo di un sistema simmetrico di tensioni di linea, con un valore efficace $V_e = 400 V$.
\vspace*{0.2cm}\\
In un sistema trifase senza neutro ci
sono $\infty^2$ serie di tensioni di fase, indotte da sistemi di impedenze di carico non
equilibrati. Le tensioni di fase sono ottenute dalla serie simmetrica di tensioni
concatenate generata dall'alternatore.
\begin{center}
    \includegraphics[scale=0.27]{Image/Sistemi_trifase_neutro_2.png}
\end{center}
Per avere sempre un sistema simmetrico di tensioni di fase, viene utilizzato il sistema trifase con neutro. Il conduttore detto "neutro" connette il centro dei tre generatori a stella (terna di generatori monofase equivalenti dell'alternatore) con il centro
del sistema di carico a stella. In tal modo ciascuna fase delle tensioni di fase, generata da uno dei tre generatori equivalenti all'alternatore, alimenta una fase del sistema delle impedenze collegate a stella.
\vspace*{0.2cm}\\
In un sistema trifase simmetrico con una sequenza positiva le tensioni di fase sono:
\begin{align*}
    e_1(t) &= \sqrt{2} E \cos(\omega t) 
    &
    e_2(t) &= \sqrt{2} E \cos(\omega t - 120^{\circ})
    &
    e_2(t) &= \sqrt{2} E \cos(\omega t + 120^{\circ})
\end{align*}
dove $E$ è il valore efficace della tensione di fase con $E = \dfrac{V}{\sqrt{3}}$ ($V$ valore efficace della tensione della linea).
\begin{center}
    \includegraphics[scale=0.25]{Image/Diagramma_delle_fasi_nel_sistema_trifase.jpg}
\end{center}


\subsubsection{Potenza istantanea trifase}
Con un carico a stella bilanciato, con le correnti della stella in ritardo di un angolo $\phi$ rispetto alle tensioni di fase si ha:
\begin{align*}
    i_1(t) &= \sqrt{2} I \sin\left(\omega t - \phi\right) 
    &
    i_2(t) &= \sqrt{2} I \sin\left(\omega t - \phi - 120^{\circ}\right)
    &
    i_2(t) &= \sqrt{2} I \sin\left(\omega t - \phi + 120^{\circ}\right)
\end{align*}
dove $I$ è il valore efficace delle correnti. La potenza istantanea totale assorbita dal carico allora è:
\begin{align*}
    p(t) &= e_1(t)i_1(t) + e_2(t)i_2(t) + e_3(t)i_3(t)
    \\
    &=2EI \left[\sin(\omega t) \sin(\omega t - \phi)+\sin \left(\omega t - \frac{2 \pi}{3}\right)\sin \left(\omega t - \phi - \frac{2 \pi}{3}\right)+\sin\left(\omega t - \frac{4 \pi}{3}\right)\sin\left(\omega t - \phi - \frac{4 \pi}{3}\right)\right]
    \\
    &= EI \left[3 \cos(\phi) - \cos(2 \omega t - \phi) - \cos \left(2 \omega t - \phi - \frac{4 \pi}{3}\right)-\cos\left(2 \omega t - \phi - \frac{8 \pi}{3}\right)\right]
\end{align*}
Le tre sinusoidi a pulsazione $2 \omega$ all'ultimo membro costituiscono una terna simmetrica diretta ed hanno quindi somma nulla in ogni istante, sicché la potenza
istantanea si riduce al solo addendo costante:
\[p(t) = 3 EI \cos(\phi)\]
Dunque la potenza istantanea in un sistema trifase simmetrico ed equilibrato è
costante nel tempo, a differenza di quanto si verifica in un sistema monofase. Questa è una delle motivazioni principali per cui, per la generazione e la distribuzione della potenza elettrica in corrente alternata, vengono utilizzati sistemi trifase.


\subsection{Differenze tra sistemi trifase bilanciati e non bilanciati}
\subsubsection{Potenza attiva e potenza reattiva}
La potenza attiva $P$ e la potenza reattiva $Q$ in un sistema trifase bilanciato,
detti $E_e$, $V_e$ ed $I_e$ rispettivamente i valori efficaci delle tensioni di fase, delle
tensioni di linea e delle correnti di linea, sono:
\begin{align*}
    P &= 3E_eI_e \cos(\phi) = \sqrt{3}V_eI_e \cos(\phi)
    &
    Q &= 3E_eI_e \sin(\phi) = \sqrt{3}V_eI_e \cos(\phi)
\end{align*}
Potenza apparente e potenza complessa sono:
\begin{align*}
    N &= \sqrt{P^2+Q^2} & \dot N &= P + jQ
\end{align*}
Il \textbf{fattore di potenza} è:
\[
    \cos(\phi) = \cos \left(\tan^{-1} \frac{Q}{P}\right)
\]
In un sistema non bilanciato l'angolo di sfasamento tra la tensione e la corrente di linea solitamente è diverso per ogni fase. In questo caso il rifasamento (correzione dell'angolo di sfasamento) deve essere fatto per ogni singola fase.\\
Per il sistema non bilanciato il fattore di potenza è:
\[
    \cos(\phi) = \cos \left(\tan^{-1} \frac{Q}{P}\right) = \frac{P}{\sqrt{P^2+Q^2}} = \frac{P}{N}
\]
Per i sistemi bilanciati e non, la rotazione del sistema trifase di correnti di un angolo $\phi$ (arco-coseno del fattore di potenza) fa si che la potenza attiva assorbita dal circuito sia massimizzata.
\begin{center}
    \includegraphics[scale=0.27]{Image/Potenza_sistemi_trifase_non_bilanciati_1.png}
\end{center}






\section{Circuiti e forze magnetiche}
\subsection{Introduzione}
Il campo magnetico $\mathbf{H}$ è definito dalla prima
legge di Maxwell (L. Ampere):
\[\oint_l \mathbf{H} \ dl = i_c\]
dove $i_c$ è il flusso di $\mathbf{J}_t$ (corrente di conduzione e spostamento) attraverso una superficie con contorno $l$ il cui verso è determinato dalla regola della vite destrogira.\\
In un filo conduttore rettilineo percorso
dalla corrente i viene indotto un campo H
le cui linee di flusso sono in piani normali al
conduttore, circolari col conduttore al
centro. 
\begin{center}
    \includegraphics[scale=0.27]{Image/Legge di Faraday.png}
    \includegraphics[scale=0.27]{Image/Campo magnetico.png}
\end{center}


\subsection{Dai campi ai circuiti magnetici}
\begin{center}
    \includegraphics[scale=0.27]{Image/Circuiti_magnetici_1.png}
\end{center}
Considerato il sistema toroidale in figura, il campo magnetico è indotto da un conduttore percorso dalla corrente $i$ ed avvolto attorno al toro con $N$ spire. Il flusso della densità di corrente concatenato con l'asse interno al toro, di lunghezza $2 \pi r$, è $i_c = Ni$. Per le linee di flusso di $\mathbf{H}$ interne al toro, assunte parallele all'asse del toro, si ha:
\begin{align*}
    2 \pi r \mathbf{H} &= Ni & &\rightarrow & \mathbf{H} &= \frac{N i}{2 \pi r}
\end{align*}
Per un materiale lineare il campo di induzione magnetica B è proporzionale
ad $H$ ($B = \mu H$). Nel caso in esame si ha:
\begin{align*}
    \frac{\mathbf{B}}{\mu} &= \frac{N i}{2 \pi r} & &\rightarrow & \mathbf{B} &= \frac{\mu N i}{2 \pi r}
\end{align*}

\subsubsection{Legge di Hopkinson}
Il campo di induzione magnetica {\boldmath {$B$}} presenta notevoli \textit{analogie} con il campo di corrente di conduzione stazionaria $J$. Anzitutto, essendo entrambi i campi solenoidali, le loro linee vettoriali sono chiuse e danno luogo a tubi di flusso chiusi. Inoltre le linee vettoriali di entrambi i campi possono essere ben canalizzate utilizzando, nel caso di {\boldmath {$J$}}, materiali ad elevata conducibilità (conduttori) circondati da altri a bassissima conducibilità (isolanti), mentre, nel caso di {\boldmath {$B$}}, utilizzando materiali ad elevata permeabilità (\textit{ferromagnetici}) circondati da altri a bassa permeabilità (\textit{amagnetici}).\\
In considerazione di tali analogie, come un campo di corrente solenoidale ben canalizzato costituisce un circuito elettrico (o rete elettrica), così un campo di induzione ben canalizzato è detto \textbf{circuito magnetico} (o rete magnetica). 
\vspace*{0.1cm}\\
Fra due sezioni $S$ trasversali 1 e 2 di un tubo di \textbf{flusso magnetico} $\Phi$, qualora fra 1 e 2 non vi siano correnti concatenate, si ha (con $\mathcal{T}$ tensione magnetica):
\begin{align*}
    \mathcal{T} &= \mathcal{T}_1 - \mathcal{T}_2
    \\
    &=\int_1^2 \mathbf{H} \ dl 
    \\
    &=\int_1^2 \frac{1}{\mu}\mathbf{B} \ dl 
    \\
    &=\int_1^2 \frac{\mathbf{B}S}{\mu S} \ dl 
    \\
    &=\Phi \int_1^2 \frac{1}{\mu S} \ dl
    \\
    &=\mathcal{R}\Phi
\end{align*}
dove, come nel caso elettrico, $\mathbf{H} = - \nabla \mathcal{T}$. Se si considera il circuito magnetico in figura, il circuito elettrico concatenato genera una forza magneto-motrice (f.m.m.), quindi $\mathcal{T}=Ni$. Si ottiene per il circuito magnetico la \textit{legge di Hopkinson}:
\[Ni = \mathcal{R}\Phi\]
\begin{center}
    \includegraphics[scale=0.27]{Image/Legge_di_Hopkinson.png}
\end{center}
dove la \textit{riluttanza} $\mathcal{R}$ per una lunghezza $l$ e una sezione trasversale $S$ del tubo è $\mathcal{R}= \dfrac{l}{S \mu}$.
\vspace*{0.2cm}\\
Possiamo quindi evidenziare le seguenti similitudini con i circuiti elettrici:
\begin{center}
    \begin{tabular}{|c|c|}
        \hline 
        \textbf{Legge di Ohm} & \textbf{Legge di Hopkinson}\\
        \hline 
        $V_s=Ri$ & $Ni = \mathcal{R}\Phi$\\
        \hline 
        $\underset{\T{f.e.m.}}{V_s}$  & $\underset{\T{f.m.m.}}{Ni}$\\
        \hline
        $\underset{\T{corrente elettrica}}{i}$ & $\underset{\T{flusso magnetico}}{\Phi}$ \\
        \hline
        $\underset{\T{resistenza}}{R}$ & $\underset{\T{riluttanza}}{\mathcal{R}}$ \\
        \hline
    \end{tabular}
\end{center}


\subsubsection{Differenze tra circuiti elettrici e circuiti magnetici}
La principale differenza fra i circuiti elettrici ed i circuiti magnetici è dovuta alle caratteristiche dei materiali.\\
Nel caso elettrico vi sono materiali isolanti e materiali conduttori con differenze di resistività elettrica dell'ordine di $10^{20}$. Nei materiali isolanti infatti il flusso di $\mathbf{J}_t$ può essere considerato nullo.\\
Nel caso magnetico invece i materiali con elevata resistività magnetica non sono completamente isolanti magnetici, per questo sono sempre presenti dei \textbf{flussi magnetici dispersi} fuori dal circuito magnetico.



\subsection{Circuiti magnetici nel dettaglio}
\begin{center}
    \includegraphics[scale=0.27]{Image/Circuiti_magnetici_dettaglio_1.png}
\end{center}
I materiali ferromagnetici usati nei circuiti magnetici sono ghisa, acciaio fuso e lamiera di acciaio.\\
Questi materiali non sono lineari, ma sotto un valore critico $B <B_S$(con $B_S$ \textbf{campo di saturazione}) il materiale si può considerare lineare con elevati valori della permeabilità magnetica $\mu$ (bassa resistività magnetica). Di conseguenza anche la \textbf{permeabilità relativa} di questi materiali è elevata:
\[
    \mu_r >>1
\]
Per la determinazione della riluttanza magnetica nel caso in figura in alto essa viene ottenuta come somma delle riluttanze dei tronchi del circuito magnetico:
\begin{align*}
    \mathcal{R}_1 &= \frac{l_1}{S \mu} & \mathcal{R}_2 &= \frac{l_2}{S \mu} & \mathcal{R}_3 &= \frac{l_3}{S \mu}
    \\
    \mathcal{R}_4 &= \frac{l_4}{S \mu} &
    \mathcal{R}_5 &= \frac{l_5}{S \mu} &
    \mathcal{R}_{tf} &= \frac{\delta}{S \mu_0}
\end{align*}
Nel ferro $\mathcal{R}_{tr}>\mathcal{R}_1,\mathcal{R}_2,\mathcal{R}_3,\mathcal{R}_4,\mathcal{R}_5$, quindi di solito le riluttanze dei rami ferromagnetici sono trascurate rispetto a quelle dei traferri (un traferro è un'interruzione dell'anello di materiale ferromagnetico).
\vspace*{0.2cm}\\
In un circuito magnetico si distinguono rami e nodi. $\Phi_k$ è il flusso magnetico del
ramo $k$, $\mathcal{T}_k$ è la tensione magnetica del ramo (differenza di potenziale magnetico fra i capi del ramo), $\mathcal{R}_k$ è la riluttanza del ramo e $N_ki_k$ è la f.m.m., generatore indipendente di tensione magnetica del ramo ove $N_k$ è il numero di spire che avvolgono il ramo e $i_k$ è la corrente elettrica che scorre nell'avvolgimento.

\subsubsection{Leggi di Kirchhoff per i circuiti megnetici e equazione di ramo magnetico}
Dalla prima legge di Maxwell si ottiene la legge di Kirchhoff delle tensioni magnetiche LKT e per la solenoidalità di $\mathbf{B}$ si perviene alla legge di Kirchhoff dei flussi magnetici LKF. Inoltre per la legge di Hopkinson si ottiene l'equazione di ramo magnetico. In un circuito magnetico di $r$ rami ed $n$ nodi si ha quindi:
\begin{align*}
    \T{LKF:}& 
    &
    &\sum_{k=1}^n \Phi_k = 0
    &
    &\T{per } n-1 \T{ nodi}
    \\
    \T{LKT:}& 
    &
    &\sum_{k=1}^m \mathcal{T}_k = 0
    &
    &\T{per } r-n+1 \T{ maglie}
    \\
    \T{ER:}& 
    &
    &\mathcal{T}_k = \mathcal{R}_k \Phi_k - N_ki_k
    &
    &\T{per gli } r \T{ rami}
\end{align*}
Utilizzando il metodo di sostituzione delle tensioni si ha:
\begin{align*}
    \T{LKF:}& 
    &
    &\sum_{k=1}^n \Phi_k = 0
    &
    &\T{per } n-1 \T{ nodi}
    \\
    \T{LKT -ER:}& 
    &
    &\sum_{k=1}^m  \mathcal{R}_k \Phi_k - N_ki_k = 0
    &
    &\T{per } r-n+1 \T{ maglie}
\end{align*}


\subsection{Analisi di un circuito magnetico}
\begin{center}
    \includegraphics[scale=0.27]{Image/Analisi_circuito_magnetico_1.png}
\end{center}
Equazioni del sistema:
\begin{align*}
    N_1i_2 &= R_1 \Phi_1 + R_3\Phi_3 
    &
    N_2i_2 &= R_3\Phi_3
    &
    \Phi_1-\Phi_2-\Phi_3=0
\end{align*}
\begin{center}
    \includegraphics[scale=0.27]{Image/Induttori_accoppiati_1.png}
\end{center}
Il flusso magnetico $\Phi_{C1}$ concatenato con il circuito 1, è generato sia da $i_1$ che da $i_2$:
\[\Phi_{C1} = \Phi_1(i_1) + \Phi_1(i_2)\]
Per un induttore lineare tempo-invariante $\Phi_{C1}$ diviene:
\[\Phi_{C1} = L_1i_1 + M_{12}i_2\]
con $L_1$ auto-induttanza e $M_{12}$ mutua-induttanza. 
\vspace*{0.2cm}\\
Analogamente per flusso magnetico $\Phi_{C1}$ concatenato con il circuito 2,si ha:
\begin{align*}
    \Phi_{C2} &= \Phi_2(i_1) + \Phi_2(i_2)
    &
    \Phi_{C2} &= L_2i_2 + M_{21}i_1
\end{align*}
con $M_{12} = M_{21}$.
\vspace*{0.2cm}\\
Le componenti dei flussi del circuito sono:
\begin{align*}
    N_1\Phi_1(i_1) &= L_1i_1 & N_1\Phi_1(i_2) &= M_{12}i_2
    \\
    N_2\Phi_2(i_2) &= L_2i_2 & N_2\Phi_2(i_1) &= M_{21}i_1
\end{align*}
Per determinare $L-1$ e $M_{12}$ è necessario calcolare $\Phi_1(i_1)$ e $\Phi_1(i_2)$.\\
Il contributo del flusso $\Phi_1(i_1)$ concatenato a $i_1$ e generato da $i_1$ è il valore di $\Phi_1$ quando la f.m.m. $N_1i_1$ è accesa e la f.m.m. $N_2i_2$ è spenta:
\[L_1 = \frac{N_1\Phi_1(i_1)}{i_1}\]
\begin{center}
    \includegraphics[scale=0.27]{Image/Analisi_circuito_magnetico_2.png}
\end{center}
nel nostro caso:
\begin{align*}
    \Phi_1(i_1) &= \underbrace{\frac{N_1i_1}{R_1}}_{Ni=\mathcal{R}\Phi} = \underbrace{\frac{L_1i_1}{N_1}}_{ N_1\Phi_1(i_1) = L_1i_1}
    &
    &\Longrightarrow
    &
    L_1 &= \frac{N_1^2}{R_1}
\end{align*}
Il contributo del flusso $\Phi_1(i_2)$ concatenato a $i_1$ e generato da $i_2$ è il valore di $\Phi_1$ quando la f.m.m. $N_1i_1$ è spenta e la f.m.m. $N_2i_2$ è accesa:
\[M_{12}=\frac{N_1\Phi_1(i_2)}{i_2}\]
\begin{center}
    \includegraphics[scale=0.27]{Image/Analisi_circuito_magnetico_3.png}
\end{center}
nel nostro caso:
\begin{align*}
    \Phi_1(i_1) &= -\frac{N_2i_2}{R_1} = \underbrace{\frac{M_{12}i_2}{N_1}}_{N_1\Phi_1(i_2)=M_{12}i_2}
    &
    &\Longrightarrow
    &
    M_{12} &= -\frac{N_1N_2}{R_1}
\end{align*}



\subsection{Energia magnetica}
L'\textit{energia magnetica} $\Varepsilon_M$ è utilizzata per generare il flusso $\Phi$ per mezzo della corrente $i$ che scorre in un avvolgimento di $N$ spire attorno a $\Phi$ ($\Phi_C=N\Phi$).\\
L'\textit{energia magnetica infinitesima} $d \Varepsilon_M$ assorbita dal campo magnetico
all'aumento di $\Phi$ nell'intervallo di tempo $dt$ è:
\begin{align*}
    d \Varepsilon_M &= P_M \ dt \\
    &=i \cdot v \ dt\\
    &= i \frac{d \Phi_C}{dt} \ dt \\
    &= i \cdot d \Phi_C
\end{align*}
dove $P_M$ è la potenza magnetica.\\
All'istante $t$ l'energia accumulata dal campo magnetico quindi è:
\begin{align*}
    \Varepsilon_M(- \infty, t) &= \int_{-\infty}^t i \cdot v \ dt \\
    &= \int_{-\infty}^t i \cdot d \Phi_C \\
    \underset{\T{caso lineare }\Phi=L \cdot i}{\Longrightarrow} &= \int_{-\infty}^t i \ d(Li) \\
    \underset{L \T{ costante}}{\Longrightarrow} &= \int_{-\infty}^t L \cdot i \ di \\
    &= \frac{1}{2}L \cdot i^2
\end{align*}


\subsubsection{Energia magnetica di correnti concatenate}
\begin{center}
    \includegraphics[scale=0.27]{Image/Energia_magnetica_1.png}
\end{center}
L'\textit{energia magnetica infinitesima} $d\Varepsilon_M$ assorbita dal campo magnetico
all'aumento di $\Phi$ prodotto dalle correnti $i_1$ e $i_2$ nell'intervallo di tempo $dt$ è:
\begin{align*}
    d\Varepsilon_M &= P_{M_1} \ dt + P_{M_2} \ dt \\
    &= i_1 \cdot v_1 \ dt + i_2 \cdot v_2 \ dt \\
    &=i_1 \ d\Phi_{C1} + i_2 \ d\Phi_{C2}
\end{align*}
All'istante $t$ l'energia accumulata dal campo
magnetico quindi è:
\begin{align*}
    \Varepsilon_M(- \infty, t) &= \int_{-\infty}^t [i_1 \cdot v_1 + i_2 \cdot v_2] \ dt \\
    &= \int_{-\infty}^t \left[i_1 \frac{d\Phi_{C1}}{dt} + i_2 \frac{d\Phi_{C2}}{dt}\right] \\
    &= \int_{-\infty}^t \left[L_1i_1 \frac{di_1}{dt}+M_{12}i_1 \frac{di_2}{dt}+M_{12}i_2 \frac{di_1}{dt}+L_2i_2 \frac{di_2}{dt}\right] \ dt\\
    &= \int_{-\infty}^t d \left[\frac{1}{2}L_1(i_1)^2 + M_{12}i_1i_2 + \frac{1}{2}L_2(i_2)^2\right]\\
    &= \frac{1}{2}L_1i_1(t)^2 + M_{12}i_1(t)i_2(t) + \frac{1}{2}L_2i_2(t)^2
\end{align*}


\subsection{Isteresi magnetica}
I materiali \textbf{ferromagnetici} sono caratterizzati dalla capacità di magnetizzarsi molto intensamente sotto l'azione di un campo magnetico esterno e di restare a lungo magnetizzati quando il campo si annulla, diventando così magneti. Invece i materiali \textbf{ferrimagnetici} sono caratterizzati da una magnetizzazione spontanea.
\vspace*{0.1cm}\\
I materiali ferromagnetici e ferrimagnetici per temperature inferiori alla temperatura di Curie hanno, per la suscettività magnetica, un comportamento \textbf{isteretico}. Perciò la relazione $\mathbf{B} = f(\mathbf{H})$ è una \textbf{funzione isterica}.\\
\begin{center}
    \includegraphics[scale=0.27]{Image/Isteresi_magnetica_2.png}
\end{center}
Nella figura è raffigurato un \textbf{ciclo si isteresi:} dopo una \textit{curva di prima magnetizzazione}, al materiale vengono applicati il campo magnetico $\mathbf{H}_{max}$ e l'induzione magnetica $\mathbf{B}_{max}$; riducendo il campo magnetico fino ad annullarlo, l'induzione non si annulla, ma raggiunge il valore $\mathbf{B}_r$ (\textit{retentivity}, \textit{induzione residua}). Per ridurre ulteriormente l'induzione è necessario invertire il verso del campo magnetico, applicando valori di $\mathbf{H}<0$. L'induzione si annulla quando si raggiunge il valore $-\mathbf{H}_c$ (\textit{coercitivity, }, \textit{campo magnetico coercitivo}). Quando, aumentando ulteriormente il modulo di $\mathbf{H}$, si raggiungo il valore $-\mathbf{H}_{max}$, l' induzione assume il valore $-\mathbf{B}_{max}$. Se si fanno variare ciclicamente i due campi tra valori estremi opposti $\mathbf{H}_{max}$ e $-\mathbf{H}_{max}$, la curva $\mathbf{H},\mathbf{B}$ descrive un \textit{ciclo di isteresi simmetrico}.
\vspace*{0.1cm}\\
Si dimostra che la densità di energia dissipata per isteresi è data dall'area del ciclo di isteresi.


\subsubsection{Materiali ferromagnetici e ferrimagnetici morbidi}
\begin{center}
    \includegraphics[scale=0.2]{Image/Isteresi_magnetica_3.png}
\end{center}
I materiali ferromagnetici (o ferrimagnetici)
morbidi hanno alti valori di permeabilità magnetica e un piccolo campo coercitivo ($\mathbf{H}_c < 103 A/m$). Pertanto la superficie del ciclo di isteresi è piccola (quindi hanno delle perdite per isteresi ridotte).\\
Questi materiali sono utilizzati per ridurre le correnti necessarie per produrre e controllare il flusso magnetico con perdite per isteresi contenute (nuclei elettromagnetici, nuclei di trasformatori, rotori e statori di macchine rotanti).\\
Tipici materiali morbidi sono il ferro e le sue leghe di nichel, cobalto e silicio, acciaio a basso contenuto di carbonio e leghe di ferro dolce.


\subsubsection{Materiali ferromagnetici e ferrimagnetici duri}
\begin{center}
    \includegraphics[scale=0.2]{Image/Isteresi_magnetica_4.png}
\end{center}
I materiali \textit{ferromagnetici} (o ferrimagnetici) \textit{duri} hanno un'alta
densità di flusso magnetico residua ($\mathbf{B}r$) e un campo coercitivo elevato ($\mathbf{H}_c > 104 A/m$). Pertanto la superficie del ciclo di isteresi è grande (dunque hanno dele perdite per isteresi elevate).\\
Questi materiali sono usati per realizzare magneti permanenti.\\
I materiali utilizzati per loro sono leghe e
terre rare con un grande campo coercitivo.




\section{Trasformatore}
\subsection{Introduzione}
\begin{center}
    \includegraphics[scale=0.27]{Image/Trasformatore_1.png}
\end{center}
Una \textbf{macchina elettrica} è costituita da due circuiti elettrici accoppiati da un flusso magnetico concatenato ad entrambi.\\
Il flusso magnetico generato dalla corrente $i_1$, se varia nel tempo rispetto al circuito 2, induce in esso una tensione (f.e.m.) $v_2$ e quindi una corrente $i_2$. La corrente $i_2$, a sua volta, genera un flusso magnetico che interagisce con il flusso prodotto da $i_1$, con l'induzione di una forza fra i due campi: si realizza così la conversione della potenza elettrica in potenza meccanica.


\subsection{Conversione elettromeccanica dell'energia}
Si considerino $n$ circuiti che possono ruotare
uno rispetto all'altro. Il circuito $j$ è percorso
dalla corrente $i_j$ ha resistenza $R_j$, angolo di rotazione $\vartheta_j$ e flusso magnetico $\Phi_{C_j}$ concatenato al circuito
\[\Phi_{C_j} = \Phi_{C_J}(i_k,\vartheta_k)\]
Il flusso concatenato col circuito j è generato
dalle correnti $i_1, i_2,..., i_n$. La f.e.m. indotta nel circuito $j$ è dovuta alla variazione nel tempo delle correnti $i_k$ ed alla variazione nel tempo della posizione di ogni circuito $k$
rispetto al j-esimo.\\
La variazione$d \Phi_{C_j}$ al variare delle correnti e delle posizioni reciproche dei circuiti è:
\begin{align*}
    d \Phi_{C_j} &= \sum_{k=1}^n \frac{\partial \Phi_{C_j}}{\partial i_k}d i_k + \sum_{k=1}^n \frac{\partial \Phi_{C_j}}{\partial \vartheta_k}d \vartheta_k
    \\
    &= \sum_{k=1}^n L_{jk} \ d i_k + \sum_{k=1}^n i_j \frac{\partial L_{jk}}{\partial \vartheta_k} d \vartheta_k i_k
\end{align*}
per $k=j$ è $L_{jk}=L_j$, per $k \neq j$ è $L_{jk} = M_{jk}$.\\
La tensione indotta nel circuito $j$ perciò diviene:
\[v_j = R_ji_j + \sum_{k=1}^n L_{jk} \ \frac{d i_k}{dt} +\underbrace{ \sum_{k=1}^n  i_k\frac{\partial L_{jk}}{\partial \vartheta_k} \omega_{jk}}_{k \neq j}\]
dove $\omega_{jk}$ è la velocità di rotazione del circuito $k$ rispetto al circuito $j$.
\vspace*{0.2cm}\\
Considerando un sistema di due circuiti di materiale lineare con relative correnti, uno in moto con velocità angolare $\omega$ rispetto all'altro, si ha:
\begin{align*}
    d \Phi_{C_1} &- L_1 \ di_1 + M_{12}\ di_2 + i_2 \frac{\partial M_{12}}{\partial \vartheta} d \vartheta
    &
    d \Phi_{C_2} &- L_2 \ di_2 + M_{21} \ di_1 + i_1 \frac{\partial M_{21}}{\partial \vartheta} d \vartheta
\end{align*}
Le tensioni ai capi dei due circuiti sono:
\begin{align*}
    v_1 &= R_1i_1 + L_1 \frac{di_1}{dt} + M_{12} \frac{di_2}{dt} + i_2 \frac{\partial M_{12}}{d\vartheta} \omega
    &
    v_2 &= R_2i_2 + L_2 \frac{di_2}{dt} + M_{12} \frac{di_1}{dt} + i_1 \frac{\partial M_{12}}{d\vartheta} \omega
\end{align*}
Nell'espressione della tensione indotta dalla
variazione del flusso magnetico concatenato
coi circuiti 1 e 2 , si distinguono due meccanismi diversi che danno luogo alla f.e.m.:
\begin{itemize}
    \item le \textbf{f.e.m. trasformatoriche} indotte nel circuito e nel circuito 2
    \begin{align*}
        &L_1 \frac{di_1}{dt} + M_{12} \frac{di_2}{dt}
        &
        L_2 \frac{di_2}{dt} + M_{12} \frac{di_1}{dt}
    \end{align*}
    \item le \textbf{f.e.m. mozionali} indotte nel circuito 1 e nel circuito 2
    \begin{align*}
        &i_2 \frac{\partial M_{12}}{\partial \vartheta} \omega
        &
        &i_1 \frac{\partial M_{12}}{\partial \vartheta} \omega
    \end{align*}
\end{itemize}
In un sistema di $n$ circuiti, la f.e.m. indotta nel circuito $j$ dalla variazione del flusso con esse concatenato è data da:
\[\underbrace{\sum_{k=1}^n L_{jk} \frac{di_k}{dt}}_{\T{f.e.m. trasformatorica}}+ \underbrace{\sum_{k=1}^n i_k \frac{\partial L_{jk}}{\partial \vartheta_k}\omega_{jk}}_{\T{f.e.m. mozionale}}\]




\subsection{Il trasformatore nel dettaglio}
\subsubsection{Differenze con macchine elettriche}
Prima di tutto evidenziamo le differenze tra trasformatore e macchine elettriche rotanti:
\begin{itemize}
    \item Nel trasformatore la velocità meccanica reciproca dei circuiti $\omega_m$ è nulla e le correnti $i_1$ e $i_2$ sono correnti sinusoidali
    \item Nelle macchine elettriche rotanti la velocità meccanica reciproca dei circuiti $\omega_m$ è non nulla e le correnti variano in base al tipo di macchina:
    \begin{itemize}
        \item per \textbf{macchine asincrone} lo statore $i_1$ è trifase, il rotore $i_2$ è trifase
        \item per \textbf{macchine sincrone} lo statore $i_1$ è trifase, il rotore $i_2$ è a corrente continua
        \item per \textbf{macchine in corrente continua} lo statore $i_1$ e il rotore $i_2$ sono in corrente continua
    \end{itemize}
\end{itemize}


\subsubsection{Struttura}
\begin{center}
    \includegraphics[scale=0.12]{Image/Trasformatore_2.png}
\end{center}
Il trasformatore consiste in due circuiti
(primario e secondario) accoppiati
magneticamente. I due circuiti sono fra
loro immobili e la f.e.m. è indotta dalla
variazione da correnti sinusoidali. In questo caso quindi la f.e.m. è trasformatorica.\\
Le tensioni ai capi dei due circuiti sono:
\begin{align*}
    v_1 &= R_1i_1+ L_1 \frac{di_1}{dt} + M_{12} \frac{di_2}{dt} + \cancel{i_2 \frac{\partial M_{12}}{\partial \vartheta}\omega}
    \\
    v_2 &= R_2i_2+ L_2 \frac{di_2}{dt} + M_{12} \frac{di_1}{dt} + \cancel{i_1 \frac{\partial M_{12}}{\partial \vartheta}\omega}
\end{align*}
Se $i_1$ ed $i_2$ sono in un sistema isofrequenziale nello spazio delle fasi si ha:
\begin{align*}
     \mathbf{\dot V}_1 &= (R_1+ j \omega L_1)  \mathbf{\dot I}_1 + j \omega M_{12}   \mathbf{\dot I}_2
     \\
     \mathbf{\dot V}_2 &= (R_2+ j \omega L_2)  \mathbf{\dot I}_2 + j \omega M_{12}   \mathbf{\dot I}_1
\end{align*}


\subsubsection{Trasformatore ideale}
Nell'ipotesi di trasformatore ideale i due circuiti si considerano \textbf{perfettamente accoppiati magneticamente}. Il flusso generato da un avvolgimento è completamente concatenato con l'altro, e si ritengono nulli i flussi dispersi. 
Inoltre non vi sono perdite nel rame dei due
avvolgimenti ($R_1 = R_2 = 0$) e nel ferro del circuito magnetico.\\
Si definisce il \textbf{rapporto spire:}
\[K=\frac{N_1}{N_2}\]
dove $N_1$ ed $N_2$ sono il numero di spire degli avvolgimenti 1, primario, e 2, secondario. In un trasformatore ideale quindi per $V1$ e $V_2$ risulta:
\begin{align*}
    V_1 &= N_1 \frac{d \Phi}{dt} & 
    V_2 &= N_2 \frac{d\Phi}{dt} &
    &\Longrightarrow &
    \frac{V_1}{V_2} &= \frac{N_1}{N_2} = K
    \\
    p_1 &= p_2 \Longrightarrow V_1 I_1 = V_2I_2 &
    & &
    &\Longrightarrow &
    \frac{I_1}{I_2} &= \frac{N_2}{N_1}=\frac{1}{K}
\end{align*}


\subsection{Perdite energetiche}
Le perdite nelle macchine elettriche sono di due tipologie:
\begin{itemize}
    \item \textbf{perdite nel rame:} queste perdite si verificano nei circuiti elettrici e sono perdite per effetto Ohm dovute alla resistenza elettrica dei circuiti, che è non-nulla;
    \item \textbf{perdite nel ferro:} sono dovute a isteresi magnetica e correnti parassite.
\end{itemize}


\subsubsection{Perdite nel ferro}
\subsubsection*{Perdite per isteresi}
\begin{center}
    \includegraphics[scale=0.24]{Image/Isteresi_ferro.png}
\end{center}
L'energia elettromagnetica $w_{\mathbf{B}}$ dissipata per unità di volume di ferro a causa dell'isteresi dipende dall'area del ciclo di isteresi $\mathbf{H}-\mathbf{B}$ ed è  proporzionale al numero di cicli per unità di tempo (alla frequenza $f$ ):
\[w_{\mathbf{B}} = \mathbf{H} \times d \mathbf{B}\]
Una formulazione semi-empirica afferma che la potenza dissipata per isteresi $P_{DH}$ è:
\[
    P_{DH} = k_H f B_{M}^\alpha
\]
$B_{M}$ è il valore massimo della induzione magnetica nel ciclo dell'isteresi, $k_\mathbf{H}$ è un coefficiente di proporzionalità, l'esponente $\alpha$ è circa 2.

\subsubsection*{Perdite per correnti parassite}
\begin{center}
    \includegraphics[scale=0.24]{Image/Nucleo_ferro.png}
\end{center}
Nei nuclei in ferro, a causa della variazione nel
tempo dell'induzione magnetica $\mathbf{B}$, viene indotta una f.e.m. perpendicolare a $\mathbf{B}$. Di  conseguenza, per l'elevata conduttività elettrica del ferro, vengono generate \textit{correnti parassite} nel ferro. Perciò viene dissipata energia elettromagnetica per effetto Joule.\\
Assumendo il regime sinusoidale e, conseguentemente un campo $\mathbf{B}(t) = \mathbf{B}_M \cos (\omega t)$ , detta $S$ la sezione di un tubo di flusso sul cui contorno si forma un circuito nel ferro per la corrente parassita e detta $R$ la resistenza del circuito in ferro, la potenza dissipata per correnti parassite $P_{DP}$ è data da:
\[
    P_{DP} =\frac{S^2\mathbf{B}_M^2 \omega^2}{2R} = k_H f^2 \mathbf{B}_M^2
\]
dove $k_H$ è un coefficiente di proporzionalità.\\
Al fine di ridurre le perdite per correnti parassite vengono utilizzate per il nucleo leghe speciali (ad esempio ferro-silicio ad elevata resistività) e viene adottato il processo di
\textbf{laminazione} con il nucleo in ferro sostituito da lamierini di ferro di ridotto spessore ($\delta \approx 1 \ mm$) fra loro isolati (vedi figura) per aumentare la resistenza e diminuire la sezione del flusso magnetico prodotto dalle correnti parassite riducendo quindi la f.e.m. indotta.
\begin{center}
    \includegraphics[scale=0.24]{Image/Laminazione_nucleo_ferro.png}
\end{center}


\subsection{Il sistema di potenza elettrico: produzione, trasmissione, distribuzione, utilizzo}
La \textbf{produzione} di potenza in una centrale elettrica generalmente è di $320-640 \ MVA$ in generatori di tensione trifase (alternatori) con valori efficaci della tensione concatenata di $10-25 \ kV$ (valore di media tensione MT).\\
La \textbf{trasmissione} fra la località della centrale e la zona del suo utilizzo copre distanze dell'ordine di 100-1000 km. La trasmissione avviene ad alta tensione (AT con $V_e = 400 \ kV$).\\
Nel luogo di utilizzo si ha prima la \textbf{distribuzione} dell'energia elettrica in media tensione (MT).\\
La corrente poi viene \textbf{utilizzata} a bassa tensione: BT con $V_e = 400 \ V$ in sistemi trifase
e $V_e = 220 \ V$ in sistemi monofase (questi ultimi sono ottenuti da trifase con neutro).\\
Il passaggio fra MT a AT dalla produzione alla trasmissione, fra AT a MT dalla trasmissione alla distribuzione e fra MT a BT fra distribuzione ed utilizzo avviene attraverso l'utilizzo di trasformatori.
\vspace*{0.2cm}\\
La \textbf{trasmissione} avviene in AT poiché a parità di potenza trasmessa $P=VI$ elevare la tensione riduce la corrente e la potenza dissipata per effetto Joule, dato che la potenza dissipata è $P= RI^2$.\\
Il passaggio da MT della produzione ad AT per la trasmissione, da AT ad MT per la distribuzione e quindi a BT per l'utilizzo è fatto con trasformatori trifase.













































































\end{document}